\chap{复数}

\section{介绍}
在这一章中,我们将发现没有实数根的方程是如何产生虚数$i$,它的平方是$-1$。这反过来又引导我们了解复数以及如何用代数方法处理它们。与四元数相关的许多特性都可以在复数中找到,这就是为什么它们值得仔细研究的原因。对此主题感兴趣的读者可能想看看作者的《计算机科学的虚数数学》(\textit{Imaginary Mathematics for Computer Science})一书\cite{bib3-1}。

\section{虚数}
虚数的发明是为了解决方程没有实根的问题,例如$x^{2}+16=0$。声明一个量$i$的存在性,使得$i^{2}=-1$,这个简单的思想允许我们将这个方程的解表示为
$$
x= \pm 4 i
$$
试图发现$i$到底是什么是没有意义的,$i$实际上只是一个平方为$-1$的数学对象。尽管如此,这是一个很棒的发明,并且确实适合于图形化的解释,我们将在下一章中对此进行研究。

1637年,法国数学家雷诺·笛卡尔(René Descartes, 1596-1650)发表了一篇论文《几何形状》\cite{bib3-2},他在文中指出,包含$\sqrt{-1}$的数字是“虚数”,几个世纪以来,这个标签一直被贴上。不幸的是,这是一个贬义的评论,并且$i$没有任何虚构的东西-它实际上只是一个平方为$-1$的东西,当嵌入代数中时,会产生一些惊人的模式。

关于如何称呼这组虚数,似乎没有达成任何共识——事实上,甚至有人认为虚数是不必要的。但是,如果您决定使用一个,则可能是$\mathbb{I}$或$ I \mathbb{R}$。因此,虚数可以定义为
$$
b i \in i \mathbb{R}, \quad i^{2}=-1
$$

\section{\boldmath$i$的平方}

由于$i^{2}=-1$,那么应该可以将$i$提高到其他幂。例如,
$$
i^{4}=i^{2} i^{2}=(-1)(-1)=1
$$
和
$$
i^{5}=i i^{4}=i
$$
因此,我们有这个序列:
\begin{center}
\begin{tabular}{lllllll}
\hline
$i^{0}$ & $i^{1}$ & $i^{2}$ & $i^{3}$ & $i^{4}$ & $i^{5}$ & $i^{6}$ \\
\hline
1 & $i$ & -1 & $-i$ & 1 & $i$ & -1 \\
\hline
\end{tabular}
\end{center}

循环模式$(1,i,-1,-i, 1, \ldots)$非常引人注目,它让我们想起了一个类似的模式$(x, y,-x,-y, x, \ldots)$,它是通过逆时针方向绕笛卡尔轴旋转产生的。这种相似性是不能忽视的,因为当实数轴与垂直的虚轴结合在一起时,就产生了所谓的复平面。但更多的内容在下一章。

以上顺序总结为:
$$
\begin{gathered}
i^{4 n}=1 \\
i^{4 n+1}=i \\
i^{4 n+2}=-1 \\
i^{4 n+3}=-i
\end{gathered}
$$
其中 $n \in \mathbb{N}$.

那么负幂呢?嗯,它们也是可能的。考虑$i^{-1}$,其计算如下:
$$
i^{-1}=\frac{1}{i}=\frac{1(-i)}{i(-i)}=\frac{-i}{1}=-i
$$
类似的
$$
i^{-2}=\frac{1}{i^{2}}=\frac{1}{-1}=-1
$$
和
$$
i^{-3}=i^{-1} i^{-2}=-i(-1)=i
$$
与负幂递增相关的数列为:
\begin{center}
\begin{tabular}{lllllll}
\hline
$i^{0}$ & $i^{-1}$ & $i^{-2}$ & $i^{-3}$ & $i^{-4}$ & $i^{-5}$ & $i^{-6}$ \\
\hline
1 & $-i$ & -1 & $i$ & 1 & $-i$ & -1 \\
\hline
\end{tabular}
\end{center}

这一次循环模式被反转为$(1,-i,-1, i, 1, \ldots)$,并且类似于模式$(x,-y,-x, y, x, \ldots)$,它是通过顺时针方向绕笛卡尔轴旋转产生的。

也许最奇怪的幂是它本身:$i^{i}$,它恰好等于$e^{-\pi / 2}= 0.207879576\dots$ \footnote{译者注,和网友讨论,并不止这一种结果,正确的结果应该如此:
\begin{align*}
    \begin{aligned}
        i &=  \cos (\frac{\pi}{2}+2k\pi)+i\sin(\frac{\pi}{2}+2k\pi)\\
        &=e^{i({\pi/2}+2k\pi)},\qquad k\in \mathbb{Z}\\
        i^i &= e^{-(\pi/2+2k\pi)},\qquad k\in \mathbb{Z}\\
    \end{aligned}
\end{align*}
}。在第四章作了说明。回顾了虚数$i$的某些特征之后,让我们来看看当它与实数结合时会发生什么。

\section{复数的定义}
根据定义,复数是实数和虚数的组合,表示为
$$
z=a+b i, \quad a, b \in \mathbb{R}, \quad i^{2}=-1
$$

复数的集合是$\mathbb{C}$,这允许我们将$z \in \mathbb{C}$中。例如,$3+4 i$是一个复数,其中3是实部,$ 4i $是虚部。以下均为复数:
$$
3, \quad 3+4 i, \quad-4-6 i, \quad 7 i, \quad 5.5+6.7 i
$$
实数也是复数——它只是没有虚部。这导致了实数集合是复数子集的想法,其表示为:
$$
\mathbb{R} \subset \mathbb{C}
$$
其中 $\subset$表示是后面几何的一个子集。

虽然有些数学家把$i$放在它的乘数$i 4$之前,但其他人把它放在乘数$4 i$之后,这是本书中使用的惯例。但是,当$i$与三角函数相关联时,最好将其放在函数之前,以避免与函数的角度混淆。例如,sin $\alpha i$表示角度是虚数,而$i \sin \alpha$表示$\sin \alpha$的值是虚数。因此,复数可以用各种方法构造:
$$
\sin \alpha+i \cos \beta, \quad 2-i \tan \alpha, \quad 23+x^{2} i
$$

一般来说,我们把一个复数写成$a+b i$,并遵从实代数的常规规则。我们所要记住的是,当我们遇到$i^{2}$时,它被$-1$替换。例如:
$$
\begin{aligned}
(2+3 i)(3+4 i) & =2 \times 3+2 \times 4 i+3 i \times 3+3 i \times 4 i \\
& =6+8 i+9 i+12 i^{2} \\
& =6+17 i-12 \\
& =-6+17 i .
\end{aligned}
$$

\subsection{复数的加减法}
给出两个复数
$$
\begin{aligned}
& z_{1}=a_{1}+b_{1} i \\
& z_{2}=a_{2}+b_{2} i
\end{aligned}
$$
然后
$$
z_{1} \pm z_{2}=\left(a_{1} \pm a_{2}\right)+\left(b_{1} \pm b_{2}\right) i
$$
其中实部和虚部分别相加或减去。只要$a_{1}, b_{1}, a_{2}, b_{2} \in \mathbb{R}$,这些操作是闭合的。
比如
$$
\begin{aligned}
z_{1} & =2+3 i \\
z_{2} & =4+2 i \\
z_{1}+z_{2} & =6+5 i \\
z_{1}-z_{2} & =-2+i .
\end{aligned}
$$

\subsection{复数乘以标量}
使用一般代数规则将一个复数乘以一个标量。例如,复数$a+b i$乘以标量$\lambda$,如下所示:
$$
\lambda(a+b i)=\lambda a+\lambda b i
$$
一个例子如下
$$
3(2+5 i)=6+15 i
$$

\subsection{复数的乘积}
给出两个复数
$$
\begin{aligned}
& z_{1}=a_{1}+b_{1} i \\
& z_{2}=a_{2}+b_{2} i
\end{aligned}
$$
它们的乘积是
$$
\begin{aligned}
z_{1} z_{2} & =\left(a_{1}+b_{1} i\right)\left(a_{2}+b_{2} i\right) \\
& =a_{1} a_{2}+a_{1} b_{2} i+b_{1} a_{2} i+b_{1} b_{2} i^{2} \\
& =\left(a_{1} a_{2}-b_{1} b_{2}\right)+\left(a_{1} b_{2}+b_{1} a_{2}\right) i
\end{aligned}
$$
这是另一个复数,确认运算是闭合的。例如:
$$
\begin{aligned}
z_{1} & =3+4 i \\
z_{2} & =3-2 i \\
z_{1} z_{2} & =(3+4 i)(3-2 i) \\
& =9-6 i+12 i-8 i^{2} \\
& =9+6 i+8 \\
& =17+6 i .
\end{aligned}
$$

注意,复数的加法、减法和乘法服从代数的一般公理。

\subsection{复数的平方}
给定一个复数$z$,其平方$z^{2}$为:
$$
\begin{aligned}
z & =a+b i \\
z^{2} & =(a+b i)(a+b i) \\
& =\left(a^{2}-b^{2}\right)+2 a b i
\end{aligned}
$$
示例:
$$
\begin{aligned}
z & =4+3 i \\
z^{2} & =(4+3 i)(4+3 i) \\
& =\left(4^{2}-3^{2}\right)+2 \times 4 \times 3 i \\
& =7+24 i .
\end{aligned}
$$

\subsection{复数的范数}
复数$z$的范数、模数或绝对值写成$|z|$,被定义为
$$
\begin{aligned}
z & =a+b i \\
|z| & =\sqrt{a^{2}+b^{2}}
\end{aligned}
$$
例如,$3+ 4i $的范数是5。当我们学习复数的极坐标表示时,我们会看到为什么会这样。

\subsection{复数的复共轭}
两个复数的乘积,只有虚部的符号不同,就得到一个特殊的结果:
$$
\begin{aligned}
(a+b i)(a-b i) & =a^{2}-a b i+a b i-b^{2} i^{2} \\
& =a^{2}+b^{2}
\end{aligned}
$$

这种类型的乘积总是得到一个实数,并用于解决两个复数的商。因为这个实值是一个非常有趣的结果,所以$a-b i$被称为$z=a+b i$的复共轭,它可以用一个条形 $\bar{z}$ 或一个星号 $z^{*}$ 来表示
$$
z z^{*}=a^{2}+b^{2}=|z|^{2} .
$$
举例
$$
\begin{aligned}
z & =3+4 i \\
z^{*} & =3-4 i \\
z z^{*} & =9+16=25 .
\end{aligned}
$$

\subsection{复数的商}
复数共轭为我们提供了一种将一个复数除以另一个复数的方法。例如,商
$$
\frac{a_{1}+b_{1} i}{a_{2}+b_{2} i}
$$
通过将分子和分母乘以分母的复共轭$a_{2}-b_{2} i$来求解,得到一个实数分母:
$$
\begin{aligned}
\frac{a_{1}+b_{1} i}{a_{2}+b_{2} i} & =\frac{\left(a_{1}+b_{1} i\right)\left(a_{2}-b_{2} i\right)}{\left(a_{2}+b_{2} i\right)\left(a_{2}-b_{2} i\right)} \\
& =\frac{a_{1} a_{2}-a_{1} b_{2} i+b_{1} a_{2} i-b_{1} b_{2} i^{2}}{a_{2}^{2}+b_{2}^{2}} \\
& =\left(\frac{a_{1} a_{2}+b_{1} b_{2}}{a_{2}^{2}+b_{2}^{2}}\right)+\left(\frac{b_{1} a_{2}-a_{1} b_{2}}{a_{2}^{2}+b_{2}^{2}}\right) i .
\end{aligned}
$$
举例,求解
$$
\frac{4+3 i}{3+4 i}
$$
上下同时乘以共轭复数 $3-4i$
$$
\begin{aligned}
\frac{4+3 i}{3+4 i} & =\frac{(4+3 i)(3-4 i)}{(3+4 i)(3-4 i)} \\
& =\frac{12-16 i+9 i-12 i^{2}}{25} \\
& =\frac{24}{25}-\frac{7}{25} i
\end{aligned}
$$

\subsection{复数的逆}
为了计算$z=a+b$的逆函数,我们先
$$
z^{-1}=\frac{1}{z} .
$$
上下同时乘以$z^{*}$就得到
$$
z^{-1}=\frac{z *}{z z^{*}}
$$
但我们之前已经证明了$z z^{*}=|z|^{2}$,因此,
$$
\begin{aligned}
z^{-1} & =\frac{z^{*}}{|z|^{2}} \\
& =\left(\frac{a}{a^{2}+b^{2}}\right)-\left(\frac{b}{a^{2}+b^{2}}\right) i
\end{aligned}
$$
举个例子,$3+ 4i $的逆是
$$
(3+4 i)^{-1}=\frac{3}{25}-\frac{4}{25} i
$$
让我们用$3+4 i$乘以它的逆来测试这个结果:
$$
(3+4 i)\left(\frac{3}{25}-\frac{4}{25} i\right)=\frac{9}{25}-\frac{12}{25} i+\frac{12}{25} i+\frac{16}{25}=1
$$
这证实了结果的正确性。

\subsection{\boldmath $\pm i$的平方根}
为了求出$\sqrt{i}$,我们假设根是复根。因此,我们先
$$
\begin{aligned}
\sqrt{i} & =a+b i \\
i & =(a+b i)(a+b i) \\
& =a^{2}+2 a b i-b^{2} \\
& =a^{2}-b^{2}+2 a b i
\end{aligned}
$$
等于实部和虚部
$$
\begin{aligned}
a^{2}-b^{2} & =0 \\
2 a b & =1 .
\end{aligned}
$$
由此我们可以推断
$$
a=b= \pm \frac{\sqrt{2}}{2}
$$
因此,根是
$$
\sqrt{i}= \pm \frac{\sqrt{2}}{2}(1+i)
$$
让我们通过平方每个根来测试这个结果,以确保答案是$i$:
$$
\left( \pm \frac{\sqrt{2}}{2}\right)^{2}(1+i)(1+i)=\frac{1}{2} 2 i=i
$$
为了求出$\sqrt{-i}$,我们假设根是复数。因此,我们先
$$
\begin{aligned}
\sqrt{-i} & =a+b i \\
-i & =(a+b i)(a+b i) \\
& =a^{2}+2 a b i-b^{2} \\
& =a^{2}-b^{2}+2 a b i
\end{aligned}
$$
等于实部和虚部
$$
\begin{aligned}
a^{2}-b^{2} & =0 \\
2 a b & =-1
\end{aligned}
$$
由此我们可以推断
$$
a=b= \pm \frac{\sqrt{2}}{2} i
$$
因此,根是
$$
\begin{aligned}
\sqrt{-i} & = \pm \frac{\sqrt{2}}{2} i(1+i) \\
& = \pm \frac{\sqrt{2}}{2}(-1+i) \\
& = \pm \frac{\sqrt{2}}{2}(1-i)
\end{aligned}
$$
让我们通过对每个根进行平方来测试这个结果,以确保答案是$-i$:
$$
\left( \pm \frac{\sqrt{2}}{2}\right)^{2}(1-i)(1-i)=-\frac{1}{2} 2 i=-i
$$
在下一章中,我们将使用这些根来研究复数的旋转性质。

\section{复数的域结构}
复数集合$\mathbb{C}$是一个域,因为它满足前面定义的域规则。

\section{有序对}
到目前为止,我们选择将复数表示为$a+b i$,这样可以区分实部和虚部。然而,有一件事我们不能假设,那就是实部总是在前面,虚部总是在后面,因为b i+a也是一个复数。因此,采用两个函数提取实、虚系数如下:
$$
\begin{aligned}
& \operatorname{Re}(a+b i)=a \\
& \operatorname{Im}(a+b i)=b
\end{aligned}
$$
这让我们想到了用有序对来表示一个复数,在有序对的情况下
$$
a+b i=(a, b)
$$
其中$b$跟在$a$后面以定义顺序。因此复数的集合$\mathbb{C}$等价于有序对$(a, b)$的集合$\mathbb{R}^{2}$。

把复数写成有序对是一个伟大的贡献,最早是由汉密尔顿(Hamilton)在1833年提出的。这样的符号非常简洁,没有任何想象的术语,可以在需要的时候添加。

\subsection{有序对的加法和减法}
给出两个复数:

$$
\begin{aligned}
& z_{1}=a_{1}+b_{1} i \\
& z_{2}=a_{2}+b_{2} i
\end{aligned}
$$
它们被写成有序对:
$$
\begin{aligned}
& z_{1}=\left(a_{1}, b_{1}\right) \\
& z_{2}=\left(a_{2}, b_{2}\right)
\end{aligned}
$$
且
$$
z_{1} \pm z_{2}=\left(a_{1} \pm a_{2}, b_{1} \pm b_{2}\right)
$$
其中两部分分别相加或相减。

举例
$$
\begin{aligned}
z_{1} & =2+3 i=(2,3) \\
z_{2} & =4+2 i=(4,2) \\
z_{1}+z_{2} & =(6,5) \\
z_{1}-z_{2} & =(-2,1) .
\end{aligned}
$$

\subsection{将有序对乘以标量}
我们已经看到如何将复数乘以标量,标量一定对有序对相同:
$$
\lambda(a, b)=(\lambda a, \lambda b)
$$
示例
$$
3(2,5)=(6,15)
$$

\subsection{有序对乘积}
给出两个复数
$$
\begin{aligned}
& z_{1}=a_{1}+b_{1} i \\
& z_{2}=a_{2}+b_{2} i
\end{aligned}
$$
它们的乘积是
$$
z_{1} z_{2}=\left(a_{1} a_{2}-b_{1} b_{2}\right)+\left(a_{1} b_{2}+b_{1} a_{2}\right) i
$$
它也必须适用于有序对:
$$
\begin{aligned}
z_{1} & =\left(a_{1}, b_{1}\right) \\
z_{2} & =\left(a_{2}, b_{2}\right) \\
z_{1} z_{2} & =\left(a_{1}, b_{1}\right)\left(a_{2}, b_{2}\right) \\
& =\left(a_{1} a_{2}-b_{1} b_{2}, a_{1} b_{2}+b_{1} a_{2}\right)
\end{aligned}
$$
举例
$$
\begin{aligned}
z_{1} & =(6,2) \\
z_{2} & =(4,3) \\
z_{1} z_{2} & =(6,2)(4,3) \\
& =(24-6,18+8) \\
& =(18,26) .
\end{aligned}
$$

\subsection{有序对平方}
复数的平方为:
$$
\begin{aligned}
z & =a+b i \\
z^{2} & =(a+b i)(a+b i) \\
& =\left(a^{2}-b^{2}\right)+2 a b i
\end{aligned}
$$
因此,有序对的平方为:
$$
\begin{aligned}
z & =(a, b) \\
z^{2} & =(a, b)(a, b) \\
& =\left(a^{2}-b^{2}, 2 a b\right)
\end{aligned}
$$
举例
$$
\begin{aligned}
z & =(4,3) \\
z^{2} & =(4,3)(4,3) \\
& =\left(4^{2}-3^{2}, 2 \times 4 \times 3\right) \\
& =(7,24)
\end{aligned}
$$
我们继续探索一个基于有序对的代数,它和复数代数是一样的。我们先写出
$$
\begin{aligned}
z & =(a, b) \\
& =(a, 0)+(0, b) \\
& =a(1,0)+b(0,1)
\end{aligned}
$$
它创建了单位有序对$(1,0)$和$(0,1)$。

现在我们来计算乘积$(1,0)(1,0)$
$$
\begin{aligned}
(1,0)(1,0) & =(1-0,0) \\
& =(1,0)
\end{aligned}
$$

这表明$(1,0)$表现得像实数1。即$(1,0)=1$。

接下来,让我们计算乘积$(0,1)(0,1)$:
$$
\begin{aligned}
(0,1)(0,1) & =(0-1,0) \\
& =(-1,0)
\end{aligned}
$$

这获得了实数 $-1$ :
$$
(0,1)^{2}=-1
$$
或
$$
(0,1)=\sqrt{-1} \text { 是一个虚数 }
$$
这意味着有序对$(a, b)$及其相关规则表示一个复数。即$(a, b) \equiv a+b i$。

\subsection{有序对范数}
有序对$z$的范数、模数或绝对值写成$|z|$,被定义为
$$
\begin{aligned}
z & =(a, b) \\
|z| & =\sqrt{a^{2}+b^{2}}
\end{aligned}
$$

比如,$(3,4)$的范数是$5$.

\subsection{有序对的复共轭}
$z=a+b i$的复共轭定义为$z^{*}=a-b i$,其在有序对中表示为$z^{*}=(a,-b)$:
$$
\begin{aligned}
z & =(a, b) \\
z^{*} & =(a,-b) \\
z z^{*} & =(a, b)(a,-b) \\
& =\left(a^{2}+b^{2}, b a-a b\right) \\
& =\left(a^{2}+b^{2}, 0\right) \\
& =a^{2}+b^{2}=|z|^{2} .
\end{aligned}
$$

\subsection{有序对的商}
解析$z_{1} / z_{2}$的技术是将表达式乘以$z_{2}^{*} / z_{2}^{*}$,使用有序对是
$$
\begin{aligned}
\frac{z_{1}}{z_{2}} & =\frac{\left(a_{1}, b_{1}\right)}{\left(a_{2}, b_{2}\right)} \\
& =\frac{\left(a_{1}, b_{1}\right)}{\left(a_{2}, b_{2}\right)} \frac{\left(a_{2},-b_{2}\right)}{\left(a_{2},-b_{2}\right)} \\
& =\frac{\left(a_{1} a_{2}+b_{1} b_{2}, b_{1} a_{2}-a_{1} b_{2}\right)}{\left(a_{2}^{2}+b_{2}^{2}, 0\right)} \\
& =\left(\frac{a_{1} a_{2}+b_{1} b_{2}}{a_{2}^{2}+b_{2}^{2}}, \frac{b_{1} a_{2}-a_{1} b_{2}}{a_{2}^{2}+b_{2}^{2}}\right) .
\end{aligned}
$$
比如,求解
$$
\frac{(4,3)}{(3,4)}
$$
上下同时乘以共轭复数$(3,-4)$
$$
\begin{aligned}
\frac{(4,3)}{(3,4)} & =\frac{(4,3)(3,-4)}{(3,4)(3,-4)} \\
& =\left(\frac{12+12}{25}, \frac{9-16}{25}\right) \\
& =\left(\frac{24}{25},-\frac{7}{25}\right) .
\end{aligned}
$$

\subsection{有序对的逆}
我们之前已经证明了$z^{-1}$是
$$
z^{-1}=\frac{z^{*}}{z z^{*}}=\frac{z^{*}}{|z|^{2}}
$$
这使用有序对表示是
$$
\begin{aligned}
z & =(a, b) \\
z^{-1} & =\frac{(a,-b)}{(a, b)(a,-b)} \\
& =\frac{(a,-b)}{\left(a^{2}+b^{2}, 0\right)} \\
& =\left(\frac{a}{a^{2}+b^{2}}, \frac{-b}{a^{2}+b^{2}}\right) .
\end{aligned}
$$
作为一个例子,$(3,4)$的逆是
$$
(3,4)^{-1}=\left(\frac{3}{25},-\frac{4}{25}\right)
$$
让我们用$(3,4)$乘以它的逆来测试这个结果:
$$
\begin{aligned}
(3,4)\left(\frac{3}{25},-\frac{4}{25}\right) & =\left(\frac{9}{25}+\frac{16}{25},-\frac{12}{25}+\frac{12}{25}\right) \\
& =(1,0) .
\end{aligned}
$$

\subsection{\boldmath $\pm i$的平方根}
为了求出$\sqrt{i}$,我们假设根是复根。因此,我们先
$$
\begin{aligned}
\sqrt{i} & =(a, b) \\
i & =(a, b)(a, b) \\
(0,1) & =\left(a^{2}-b^{2}, 2 a b\right)
\end{aligned}
$$
等于左序元素和右序元素
$$
\begin{aligned}
a^{2}-b^{2} & =0 \\
2 a b & =1 .
\end{aligned}
$$
由此我们可以推断
$$
a=b= \pm \frac{\sqrt{2}}{2}
$$
因此,根是
$$
\sqrt{i}= \pm \frac{\sqrt{2}}{2}(1,1)
$$
让我们通过平方每个根来测试这个结果,以确保答案是$i$:
$$
\left( \pm \frac{\sqrt{2}}{2}\right)^{2}(1,1)(1,1)=\frac{1}{2}(0,2)=(0,1)
$$
为了求出$\sqrt{-i}$,我们假设根是复数。因此,我们先
$$
\begin{aligned}
\sqrt{-i} & =(a, b) \\
-i & =(a, b)(a, b) \\
(0,-1) & =\left(a^{2}-b^{2}, 2 a b\right)
\end{aligned}
$$
等于左序元素和右序元素
$$
\begin{aligned}
a^{2}-b^{2} & =0 \\
2 a b & =-1
\end{aligned}
$$
由此我们可以推断
$$
\begin{aligned}
a=b & = \pm \frac{\sqrt{2}}{2} i \\
& = \pm \frac{\sqrt{2}}{2}(0,1)(1,1) \\
& = \pm \frac{\sqrt{2}}{2}(-1,1)
\end{aligned}
$$
因此,根是
$$
\sqrt{-i}= \pm \frac{\sqrt{2}}{2}(1,-1)
$$
让我们通过对每个根进行平方来测试这个结果,以确保答案是$-i$:
$$
\left( \pm \frac{\sqrt{2}}{2}\right)^{2}(1,-1)(1,-1)=\frac{1}{2}(0,-2)=(0,-1)
$$
从上面的定义可以明显看出,有序对为表示复数提供了另一种符号,其中虚特征嵌入到乘积公理中。我们还将使用有序对来定义具有三个虚项的四元数,当将它们合并到乘积公理中时,它们仍然是隐藏的。

\section{复数的矩阵表示}
由于四元数具有矩阵表示,也许我们应该研究复数的矩阵表示。

虽然我只是暗示了$ i $可以被看作某种旋转算子,但这是把它形象化的完美方式。在第4章中,我们发现将一个复数乘以$i$可以有效地逆时针旋转$90^{\circ}$。所以现在,它可以用$90^{\circ}$的旋转矩阵来表示:
$$
i \equiv\left[\begin{array}{cc}
\cos 90^{\circ} & -\sin 90^{\circ} \\
\sin 90^{\circ} & \cos 90^{\circ}
\end{array}\right]=\left[\begin{array}{cc}
0 & -1 \\
1 & 0
\end{array}\right]
$$
和$2 \times 2$的单位矩阵是
$$
\left[\begin{array}{ll}
1 & 0 \\
0 & 1
\end{array}\right]
$$
这允许我们将复数写成:
$$
\begin{aligned}
a+b i & =a\left[\begin{array}{ll}
1 & 0 \\
0 & 1
\end{array}\right]+b\left[\begin{array}{cc}
0 & -1 \\
1 & 0
\end{array}\right] \\
& =\left[\begin{array}{ll}
a & 0 \\
0 & a
\end{array}\right]+\left[\begin{array}{cc}
0 & -b \\
b & 0
\end{array}\right] \\
& =\left[\begin{array}{cc}
a & -b \\
b & a
\end{array}\right] .
\end{aligned}
$$
注意,这个矩阵运算表示 $i$ 平方到 $-1$:
$$
\begin{aligned}
{\left[\begin{array}{cc}
0 & -1 \\
1 & 0
\end{array}\right]\left[\begin{array}{cc}
0 & -1 \\
1 & 0
\end{array}\right] } & =\left[\begin{array}{cc}
-1 & 0 \\
0 & -1
\end{array}\right] \\
& =-1\left[\begin{array}{ll}
1 & 0 \\
0 & 1
\end{array}\right] .
\end{aligned}
$$

然而,我们也必须记住 $i^{2}=(-i)^{2}=-1$,这被解释为复平面上的逆时针和顺时针旋转。这一点得到以下方面的证实:
$$
\begin{aligned}
{\left[\begin{array}{cc}
0 & 1 \\
-1 & 0
\end{array}\right]\left[\begin{array}{cc}
0 & 1 \\
-1 & 0
\end{array}\right] } & =\left[\begin{array}{cc}
-1 & 0 \\
0 & -1
\end{array}\right] \\
& =-1\left[\begin{array}{ll}
1 & 0 \\
0 & 1
\end{array}\right] .
\end{aligned}
$$
现在让我们用矩阵符号来表示所有用于复数的算术运算。

\subsection{复数的加减法}
两个复数的加减法如下:
$$
\begin{aligned}
z_{1} & =a_{1}+b_{1} i \\
z_{2} & =a_{2}+b_{2} i \\
z_{1} & =\left[\begin{array}{cc}
a_{1} & -b_{1} \\
b_{1} & a_{1}
\end{array}\right] \\
z_{2} & =\left[\begin{array}{cc}
a_{2} & -b_{2} \\
b_{2} & a_{2}
\end{array}\right] \\
z_{1} \pm z_{2} & =\left[\begin{array}{cc}
a_{1} & -b_{1} \\
b_{1} & a_{1}
\end{array}\right] \pm\left[\begin{array}{cc}
a_{2} & -b_{2} \\
b_{2} & a_{2}
\end{array}\right] \\
& =\left[\begin{array}{ll}
a_{1} \pm a_{2} & -\left(b_{1} \pm b_{2}\right) \\
b_{1} \pm b_{2} & a_{1} \pm a_{2}
\end{array}\right] .
\end{aligned}
$$

举个例子
$$
\begin{aligned}
z_{1} & =2+3 i \\
z_{2} & =4+2 i \\
z_{1} & =\left[\begin{array}{cc}
2 & -3 \\
3 & 2
\end{array}\right] \\
z_{2} & =\left[\begin{array}{cc}
4 & -2 \\
2 & 4
\end{array}\right] \\
z_{1} \pm z_{2} & =\left[\begin{array}{cc}
2 & -3 \\
3 & 2
\end{array}\right] \pm\left[\begin{array}{cc}
4 & -2 \\
2 & 4
\end{array}\right] \\
z_{1}+z_{2} & =\left[\begin{array}{cc}
6 & -5 \\
5 & 6
\end{array}\right]=6+5 i \\
z_{1}-z_{2} & =\left[\begin{array}{cc}
-2 & -1 \\
1 & -2
\end{array}\right]=-2+i .
\end{aligned}
$$

\subsection{两个复数的乘积}
两个复数之积的计算方法如下:
$$
\begin{aligned}
z_{1} & =a_{1}+b_{1} i \\
z_{2} & =a_{2}+b_{2} i \\
z_{1} z_{2} & =\left[\begin{array}{cc}
a_{1} & -b_{1} \\
b_{1} & a_{1}
\end{array}\right]\left[\begin{array}{cc}
a_{2} & -b_{2} \\
b_{2} & a_{2}
\end{array}\right] \\
& =\left[\begin{array}{cc}
a_{1} a_{2}-b_{1} b_{2} & -\left(a_{1} b_{2}+b_{1} a_{2}\right) \\
a_{1} b_{2}+b_{1} a_{2} & a_{1} a_{2}-b_{1} b_{2}
\end{array}\right] .
\end{aligned}
$$
举个例子
$$
\begin{aligned}
z_{1} & =6+2 i \\
z_{2} & =4+3 i \\
z_{1} z_{2} & =\left[\begin{array}{cc}
6 & -2 \\
2 & 6
\end{array}\right]\left[\begin{array}{cc}
4 & -3 \\
3 & 4
\end{array}\right] \\
& =\left[\begin{array}{cc}
24-6 & -(18+8) \\
18+8 & 24-6
\end{array}\right] \\
& =\left[\begin{array}{cc}
18 & -26 \\
26 & 18
\end{array}\right] .
\end{aligned}
$$

\subsection{复数的范数平方}
范数的平方作为矩阵的行列式:
$$
\begin{aligned}
z & =a+b i \\
& =\left[\begin{array}{cc}
a & -b \\
b & a
\end{array}\right] \\
|z|^{2} & =a^{2}+b^{2}=\left|\begin{array}{cc}
a & -b \\
b & a
\end{array}\right| .
\end{aligned}
$$

\subsection{复数的复共轭}
复数的共轭复数是
$$
\begin{aligned}
z=a+b i & =\left[\begin{array}{cc}
a & -b \\
b & a
\end{array}\right] \\
z^{*}=a-b i & =\left[\begin{array}{cc}
a & b \\
-b & a
\end{array}\right] .
\end{aligned}
$$
乘积$z z^{*}=a^{2}+b^{2}$:
$$
\begin{aligned}
z z^{*} & =\left[\begin{array}{cc}
a & -b \\
b & a
\end{array}\right]\left[\begin{array}{cc}
a & b \\
-b & a
\end{array}\right] \\
& =\left[\begin{array}{cc}
a^{2}+b^{2} & 0 \\
0 & a^{2}+b^{2}
\end{array}\right] \\
& =\left(a^{2}+b^{2}\right)\left[\begin{array}{ll}
1 & 0 \\
0 & 1
\end{array}\right] .
\end{aligned}
$$
举个例子:
$$
\begin{aligned}
z & =3+4 i=\left[\begin{array}{cc}
3 & -4 \\
4 & 3
\end{array}\right] \\
z^{*} & =3-4 i=\left[\begin{array}{cc}
3 & 4 \\
-4 & 3
\end{array}\right] \\
z z^{*} & =\left[\begin{array}{cc}
3 & -4 \\
4 & 3
\end{array}\right]\left[\begin{array}{cc}
3 & 4 \\
-4 & 3
\end{array}\right]=\left[\begin{array}{cc}
25 & 0 \\
0 & 25
\end{array}\right] \\
& =25\left[\begin{array}{cc}
1 & 0 \\
0 & 1
\end{array}\right]
\end{aligned}
$$

\subsection{复数的逆}
$2 \times 2$矩阵$\mathbf{A}$的逆如下式给出
$$
\begin{aligned}
\mathbf{A} & =\left[\begin{array}{ll}
a_{11} & a_{12} \\
a_{21} & a_{22}
\end{array}\right] \\
\mathbf{A}^{-1} & =\frac{1}{a_{11} a_{22}-a_{12} a_{21}}\left[\begin{array}{cc}
a_{22} & -a_{12} \\
-a_{21} & a_{12}
\end{array}\right]
\end{aligned}
$$
因此,$z$ 的逆如下
$$
\begin{aligned}
z & =a+b i \\
z & =\left[\begin{array}{cc}
a & -b \\
b & a
\end{array}\right] \\
z^{-1} & =\frac{1}{a^{2}+b^{2}}\left[\begin{array}{cc}
a & b \\
-b & a
\end{array}\right] .
\end{aligned}
$$
举个例子:
$$
\begin{aligned}
z & =3+4 i \\
z & =\left[\begin{array}{cc}
3 & -4 \\
4 & 3
\end{array}\right] \\
z^{-1} & =\frac{1}{25}\left[\begin{array}{cc}
3 & 4 \\
-4 & 3
\end{array}\right] .
\end{aligned}
$$

\subsection{复数的商}
The quotient of two complex numbers is computed as follows:

$$
\begin{aligned}
z_{1} & =a_{1}+b_{1} i \\
z_{2} & =a_{2}+b_{2} i \\
\frac{z_{1}}{z_{2}} & =z_{1} z_{2}^{-1} \\
& =\left[\begin{array}{cc}
a_{1} & -b_{1} \\
b_{1} & a_{1}
\end{array}\right] \frac{1}{a_{2}^{2}+b_{2}^{2}}\left[\begin{array}{cc}
a_{2} & b_{2} \\
-b_{2} & a_{2}
\end{array}\right] \\
& =\frac{1}{a_{2}^{2}+b_{2}^{2}}\left[\begin{array}{cc}
a_{1} a_{2}+b_{1} b_{2} & -\left(b_{1} a_{2}-a_{1} b_{2}\right) \\
b_{1} a_{2}-a_{1} b_{2} & a_{1} a_{2}+b_{1} b_{2}
\end{array}\right] .
\end{aligned}
$$

For example:

$$
\begin{aligned}
z_{1} & =4+3 i \\
z_{2} & =3+4 i \\
\frac{z_{1}}{z_{2}} & =z_{1} z_{2}^{-1} \\
& =\left[\begin{array}{cc}
4 & -3 \\
3 & 4
\end{array}\right] \frac{1}{9+16}\left[\begin{array}{cc}
3 & 4 \\
-4 & 3
\end{array}\right] \\
& =\frac{1}{25}\left[\begin{array}{cc}
24 & 7 \\
-7 & 24
\end{array}\right] .
\end{aligned}
$$

\subsection{\boldmath $\pm i$的平方根}
To find $\sqrt{i}$ we assume that the roots are complex. Therefore, we start with

$$
\begin{aligned}
\sqrt{i} & =\left[\begin{array}{cc}
a & -b \\
b & a
\end{array}\right] \\
i & =\left[\begin{array}{cc}
a & -b \\
b & a
\end{array}\right]\left[\begin{array}{cc}
a & -b \\
b & a
\end{array}\right] \\
{\left[\begin{array}{cc}
0 & -1 \\
1 & 0
\end{array}\right] } & =\left[\begin{array}{cc}
a^{2}-b^{2} & -2 a b \\
2 a b & a^{2}-b^{2}
\end{array}\right]
\end{aligned}
$$

and equating left and right matrices we have

$$
\begin{aligned}
a^{2}-b^{2} & =0 \\
2 a b & =1 .
\end{aligned}
$$

From this we deduce that

$$
a=b= \pm \frac{\sqrt{2}}{2}
$$

Therefore, the roots are

$$
\sqrt{i}= \pm \frac{\sqrt{2}}{2}\left[\begin{array}{cc}
1 & -1 \\
1 & 1
\end{array}\right]
$$

Let's test this result by squaring each root to ensure the answer is $i$ :

$$
\left( \pm \frac{\sqrt{2}}{2}\right)^{2}\left[\begin{array}{cc}
1 & -1 \\
1 & 1
\end{array}\right]\left[\begin{array}{cc}
1 & -1 \\
1 & 1
\end{array}\right]=\frac{1}{2}\left[\begin{array}{cc}
0 & -2 \\
2 & 0
\end{array}\right]=i
$$

To find $\sqrt{-i}$ we assume that the roots are complex. Therefore, we start with

$$
\begin{aligned}
\sqrt{-i} & =\left[\begin{array}{cc}
a & -b \\
b & a
\end{array}\right] \\
-i & =\left[\begin{array}{cc}
a & -b \\
b & a
\end{array}\right]\left[\begin{array}{cc}
a & -b \\
b & a
\end{array}\right] \\
{\left[\begin{array}{cc}
0 & 1 \\
-1 & 0
\end{array}\right] } & =\left[\begin{array}{cc}
a^{2}-b^{2} & -2 a b \\
2 a b & a^{2}-b^{2}
\end{array}\right]
\end{aligned}
$$

and equating left and right matrices we have

$$
\begin{aligned}
a^{2}-b^{2} & =0 \\
2 a b & =-1
\end{aligned}
$$

From this we deduce that

$$
a=b= \pm \frac{\sqrt{2}}{2} i
$$

Therefore, the roots are

$$
\sqrt{-i}= \pm \frac{\sqrt{2}}{2}\left[\begin{array}{cc}
0 & -1 \\
1 & 0
\end{array}\right]\left[\begin{array}{cc}
1 & -1 \\
1 & 1
\end{array}\right]= \pm \frac{\sqrt{2}}{2}\left[\begin{array}{cc}
1 & 1 \\
-1 & 1
\end{array}\right]
$$

Let's test this result by squaring each root to ensure the answer is $i$ :

$$
\left( \pm \frac{\sqrt{2}}{2}\right)^{2}\left[\begin{array}{cc}
1 & 1 \\
-1 & 1
\end{array}\right]\left[\begin{array}{cc}
1 & 1 \\
-1 & 1
\end{array}\right]=\frac{1}{2}\left[\begin{array}{cc}
0 & 2 \\
-2 & 0
\end{array}\right]=-i
$$

\section{总结}
We have shown in this chapter that the set of complex numbers is a field, as they satisfy the requirement for closure, associativity, distributivity, an identity element, and an inverse. We have also shown that there is a one-to-one correspondence between a complex number and an ordered pair, and that a complex number can be represented as a matrix, which permits us to compute all complex number operations as matrix operations or ordered pairs.

If this the first time you have come across complex numbers you probably will have found them strange on the one hand, and amazing on the other. Simply by declaring the existence of $i$ that squares to -1 , opens up a new number system that unifies large areas of mathematics.

\subsection{定义总结}
\subsubsection{定义}
$i \mathbb{R}$ is the set of imaginary numbers: $i b \in i \mathbb{R}, \quad i^{2}=-1$.

\subsubsection*{复数}

Real unit: 1

Imaginary unit: $i$.

\subsubsection*{有序对}

$$
\begin{aligned}
& 1=(1,0) \\
& i=(0,1) .
\end{aligned}
$$

\subsubsection*{矩阵}

$$
\begin{aligned}
1 & =\left[\begin{array}{ll}
1 & 0 \\
0 & 1
\end{array}\right] \\
i & =\left[\begin{array}{cc}
0 & -1 \\
1 & 0
\end{array}\right] .
\end{aligned}
$$

$\mathbb{C}$ is the set of complex numbers: $z=a+i b, \quad a \in \mathbb{R}, \quad i b \in i \mathbb{R}, \quad z \in \mathbb{C}$.

\subsubsection*{复数}
$$
z=a+b i
$$

\subsubsection*{有序对}
$$
z=(a, b)
$$

\subsubsection*{矩阵}

$$
a+b i=\left[\begin{array}{cc}
a & -b \\
b & a
\end{array}\right]
$$

\subsubsection{加减法}
\subsubsection*{复数}

$$
\begin{aligned}
z_{1} & =a_{1}+b_{1} i \\
z_{2} & =a_{2}+b_{2} i \\
z_{1} \pm z_{2} & =a_{1} \pm a_{2}+\left(b_{1} \pm b_{2}\right) i
\end{aligned}
$$

\subsubsection*{有序对}
$$
\begin{aligned}
z_{1} & =\left(a_{1}, b_{1}\right) \\
z_{2} & =\left(a_{2}, b_{2}\right) \\
z_{1} \pm z_{2} & =\left(a_{1} \pm a_{2}, b_{1} \pm b_{2}\right) .
\end{aligned}
$$

\subsubsection*{矩阵}
$$
\begin{aligned}
z_{1} & =\left[\begin{array}{ll}
a_{1} & -b_{1} \\
b_{1} & a_{1}
\end{array}\right] \\
z_{2} & =\left[\begin{array}{cc}
a_{2} & -b_{2} \\
b_{2} & a_{2}
\end{array}\right] \\
z_{1} \pm z_{2} & =\left[\begin{array}{cc}
a_{1} & -b_{1} \\
b_{1} & a_{1}
\end{array}\right] \pm\left[\begin{array}{cc}
a_{2} & -b_{2} \\
b_{2} & a_{2}
\end{array}\right] \\
& =\left[\begin{array}{cc}
a_{1} \pm a_{2} & -\left(b_{1} \pm b_{2}\right) \\
b_{1} \pm b_{2} & a_{1} \pm a_{2}
\end{array}\right] .
\end{aligned}
$$

\subsubsection{乘法}
\subsubsection*{复数}

$$
\begin{aligned}
z_{1} & =a_{1}+b_{1} i \\
z_{2} & =a_{2}+b_{2} i \\
z_{1} z_{2} & =\left(a_{1}+b_{1} i\right)\left(a_{2}+b_{2} i\right) \\
& =\left(a_{1} a_{2}-b_{1} b_{2}\right)+\left(a_{1} b_{2}+b_{1} a_{2}\right) i
\end{aligned}
$$

\subsubsection*{有序对}
$$
\begin{aligned}
z_{1} & =\left(a_{1}, b_{1}\right) \\
z_{2} & =\left(a_{2}, b_{2}\right) \\
z_{1} z_{2} & =\left(a_{1}, b_{1}\right)\left(a_{2}, b_{2}\right) \\
& =\left(a_{1} a_{2}-b_{1} b_{2}, a_{1} b_{2}+b_{1} a_{2}\right) .
\end{aligned}
$$

\subsubsection*{矩阵}
$$
\begin{aligned}
z_{1} & =\left[\begin{array}{ll}
a_{1} & -b_{1} \\
b_{1} & a_{1}
\end{array}\right] \\
z_{2} & =\left[\begin{array}{cc}
a_{2} & -b_{2} \\
b_{2} & a_{2}
\end{array}\right]
\end{aligned}
$$

$$
\begin{aligned}
z_{1} z_{2} & =\left[\begin{array}{cc}
a_{1} & -b_{1} \\
b_{1} & a_{1}
\end{array}\right]\left[\begin{array}{cc}
a_{2} & -b_{2} \\
b_{2} & a_{2}
\end{array}\right] \\
& =\left[\begin{array}{cc}
a_{1} a_{2}-b_{1} b_{2} & -\left(a_{1} b_{2}+b_{1} a_{2}\right) \\
a_{1} b_{2}+b_{1} a_{2} & a_{1} a_{2}-b_{1} b_{2}
\end{array}\right]
\end{aligned}
$$

\subsubsection{平方}
\subsubsection*{复数}
$$
\begin{aligned}
z & =a+b i \\
z^{2} & =(a+b i)(a+b i) \\
& =\left(a^{2}-b^{2}\right)+2 a b i
\end{aligned}
$$

\subsubsection*{有序对}
$$
\begin{aligned}
z & =(a, b) \\
z^{2} & =(a, b)(a, b) \\
& =\left(a^{2}-b^{2}, 2 a b\right)
\end{aligned}
$$

\subsubsection*{矩阵}

$$
\begin{aligned}
z & =\left[\begin{array}{cc}
a & -b \\
b & a
\end{array}\right] \\
z^{2} & =\left[\begin{array}{cc}
a & -b \\
b & a
\end{array}\right]\left[\begin{array}{cc}
a & -b \\
b & a
\end{array}\right] \\
& =\left[\begin{array}{cc}
a^{2}-b^{2} & -2 a b \\
2 a b & a^{2}-b^{2}
\end{array}\right] .
\end{aligned}
$$

\subsubsection{范数}

\subsubsection*{复数}

$$
\begin{aligned}
z & =a+b i \\
|z| & =\sqrt{a^{2}+b^{2}}
\end{aligned}
$$

\subsubsection*{有序对}
$$
\begin{aligned}
z & =(a, b) \\
|z| & =\sqrt{a^{2}+b^{2}}
\end{aligned}
$$

\subsubsection*{矩阵}
$$
\begin{aligned}
z & =\left[\begin{array}{cc}
a & -b \\
b & a
\end{array}\right] \\
|z|^{2} & =\left|\begin{array}{cc}
a & -b \\
b & a
\end{array}\right| .
\end{aligned}
$$

\subsubsection{复共轭}
\subsubsection*{复数}

$$
\begin{aligned}
z & =a+b i \\
z^{*} & =a-b i
\end{aligned}
$$

\subsubsection*{有序对}
$$
\begin{aligned}
z & =(a, b) \\
z^{*} & =(a,-b) .
\end{aligned}
$$

\subsubsection*{矩阵}

$$
\begin{aligned}
z & =\left[\begin{array}{cc}
a & -b \\
b & a
\end{array}\right] \\
z^{*} & =\left[\begin{array}{cc}
a & b \\
-b & a
\end{array}\right] .
\end{aligned}
$$

\subsubsection{逆}
\subsubsection*{复数}

$$
\begin{aligned}
z & =a+b i \\
z^{-1} & =\frac{z^{*}}{|z|^{2}} \\
& =\left(\frac{a}{a^{2}+b^{2}}\right)-\left(\frac{b}{a^{2}+b^{2}}\right) i
\end{aligned}
$$

\subsubsection*{有序对}
$$
\begin{aligned}
z & =(a, b) \\
z^{*} & =(a,-b) \\
|z|^{2} & =a^{2}+b^{2}
\end{aligned}
$$

$$
\begin{aligned}
\frac{1}{z} & =z^{-1}=\frac{z^{*}}{|z|^{2}} \\
& =\left(\frac{a}{a^{2}+b^{2}}, \frac{-b}{a^{2}+b^{2}}\right)
\end{aligned}
$$

\subsubsection*{矩阵}
$$
\begin{aligned}
z & =\left[\begin{array}{cc}
a & -b \\
b & a
\end{array}\right] \\
z^{*} & =\left[\begin{array}{cc}
a & b \\
-b & a
\end{array}\right] \\
|z|^{2} & =a^{2}+b^{2} \\
\frac{1}{z} & =z^{-1}=\frac{z^{*}}{|z|^{2}} \\
& =\frac{1}{a^{2}+b^{2}}\left[\begin{array}{cc}
a & b \\
-b & a
\end{array}\right] .
\end{aligned}
$$

\subsubsection{商}
\subsubsection*{复数}

$$
\begin{aligned}
z_{1} & =a_{1}+b_{1} i \\
z_{2} & =a_{2}+b_{2} i \\
\frac{z_{1}}{z_{2}} & =\frac{a_{1}+b_{1} i}{a_{2}+b_{2} i} \\
& =\left(\frac{a_{1} a_{2}+b_{1} b_{2}}{a_{2}^{2}+b_{2}^{2}}\right)+\left(\frac{b_{1} a_{2}-a_{1} b_{2}}{a_{2}^{2}+b_{2}^{2}}\right) i .
\end{aligned}
$$

\subsubsection*{有序对}
$$
\begin{aligned}
z_{1} & =\left(a_{1}, b_{1}\right) \\
z_{2} & =\left(a_{2}, b_{2}\right) \\
\frac{z_{1}}{z_{2}} & =\frac{\left(a_{1}, b_{1}\right)}{\left(a_{2}, b_{2}\right)} \\
& =\left(\frac{a_{1} a_{2}+b_{1} b_{2}}{a_{2}^{2}+b_{2}^{2}}, \frac{b_{1} a_{2}-a_{1} b_{2}}{a_{2}^{2}+b_{2}^{2}}\right) .
\end{aligned}
$$

\subsubsection*{矩阵}
$$
\begin{aligned}
z_{1} & =\left[\begin{array}{cc}
a_{1} & -b_{1} \\
b_{1} & a_{1}
\end{array}\right] \\
z_{2} & =\left[\begin{array}{cc}
a_{2} & -b_{2} \\
b_{2} & a_{2}
\end{array}\right] \\
\frac{z_{1}}{z_{2}} & =z_{1} z_{2}^{-1} \\
& =\frac{1}{a_{2}^{2}+b_{2}^{2}}\left[\begin{array}{cc}
a_{1} & -b_{1} \\
b_{1} & a_{1}
\end{array}\right]\left[\begin{array}{cc}
a_{2} & b_{2} \\
-b_{2} & a_{2}
\end{array}\right] \\
& =\frac{1}{a_{2}^{2}+b_{2}^{2}}\left[\begin{array}{ll}
a_{1} a_{2}+b_{1} b_{2} & -\left(b_{1} a_{2}-a_{1} b_{2}\right) \\
b_{1} a_{2}-a_{1} b_{2} & a_{1} a_{2}+b_{1} b_{2}
\end{array}\right] .
\end{aligned}
$$

\subsubsection{\boldmath $\pm i$的平方根}
\subsubsection*{复数}
$$
\begin{aligned}
\sqrt{i} & = \pm \frac{\sqrt{2}}{2}(1+i) \\
\sqrt{-i} & = \pm \frac{\sqrt{2}}{2}(1-i)
\end{aligned}
$$

\subsubsection*{有序对}
$$
\begin{aligned}
\sqrt{i} & = \pm \frac{\sqrt{2}}{2}(1,1) \\
\sqrt{-i} & = \pm \frac{\sqrt{2}}{2}(1,-1) .
\end{aligned}
$$

\subsubsection*{矩阵}
$$
\begin{aligned}
\sqrt{i} & = \pm \frac{\sqrt{2}}{2}\left[\begin{array}{cc}
1 & -1 \\
1 & 1
\end{array}\right] \\
\sqrt{-i} & = \pm \frac{\sqrt{2}}{2}\left[\begin{array}{cc}
1 & 1 \\
-1 & 1
\end{array}\right] .
\end{aligned}
$$

\section{样例}
Here are some further worked examples that employ the ideas described above. In some cases a test is included to confirm the result.

\begin{Li}
复数加减法
$z_{1}$ 和 $z_{2}$相加。

复数

$$
\begin{aligned}
z_{1} & =12+6 i \\
z_{2} & =10-4 i \\
z_{1}+z_{2} & =22+2 i \\
z_{1}-z_{2} & =2+10 i
\end{aligned}
$$

有序对

$$
\begin{aligned}
z_{1} & =(12,6) \\
z_{2} & =(10,-4) \\
z_{1}+z_{2} & =(12,6)+(10,-4)=(22,2) \\
z_{1}-z_{2} & =(12,6)-(10,-4)=(2,10) .
\end{aligned}
$$

矩阵

$$
\begin{aligned}
z_{1} & =\left[\begin{array}{cc}
12 & -6 \\
6 & 12
\end{array}\right] \\
z_{2}= & {\left[\begin{array}{cc}
10 & 4 \\
-4 & 10
\end{array}\right] } \\
z_{1}+z_{2} & =\left[\begin{array}{cc}
12 & -6 \\
6 & 12
\end{array}\right]+\left[\begin{array}{cc}
10 & 4 \\
-4 & 10
\end{array}\right]=\left[\begin{array}{cc}
22 & -2 \\
2 & 22
\end{array}\right] \\
z_{1}-z_{2} & =\left[\begin{array}{cc}
12 & -6 \\
6 & 12
\end{array}\right]-\left[\begin{array}{cc}
10 & 4 \\
-4 & 10
\end{array}\right]=\left[\begin{array}{cc}
2 & -10 \\
10 & 2
\end{array}\right] .
\end{aligned}
$$

\end{Li}

\subsubsection{Product of Complex Numbers}
Compute the product $z_{1} z_{2}$.

\section{Complex Number}
$$
\begin{aligned}
z_{1} & =12+6 i \\
z_{2} & =10-4 i \\
z_{1} z_{2} & =(12+6 i)(10-4 i) \\
& =144+12 i
\end{aligned}
$$

\section{Ordered Pair}
$$
\begin{aligned}
z_{1} & =(12,6) \\
z_{2} & =(10,-4) \\
z_{1} z_{2} & =(12,6)(10,-4) \\
& =(120+24,-48+60) \\
& =(144,12) .
\end{aligned}
$$

\section{Matrix}
$$
\begin{aligned}
z_{1} & =\left[\begin{array}{cc}
12 & -6 \\
6 & 12
\end{array}\right] \\
z_{2} & =\left[\begin{array}{cc}
10 & 4 \\
-4 & 10
\end{array}\right] \\
z_{1} z_{2} & =\left[\begin{array}{cc}
12 & -6 \\
6 & 12
\end{array}\right]\left[\begin{array}{cc}
10 & 4 \\
-4 & 10
\end{array}\right]=\left[\begin{array}{cc}
144 & -12 \\
12 & 144
\end{array}\right] .
\end{aligned}
$$

\subsubsection{Multiplying a Complex Number by $i$}
Multiply $z_{1}$ by $i$.

Complex Number

$$
\begin{aligned}
z_{1} & =12+6 i \\
z_{1} i & =(12+6 i) i \\
& =-6+12 i
\end{aligned}
$$

\section{Ordered Pair}
$$
\begin{aligned}
z_{1} & =(12,6) \\
i & =(0,1) \\
z_{1} i & =(12,6)(0,1) \\
& =(-6,12) .
\end{aligned}
$$

\section{Matrix}
$$
\begin{aligned}
z_{1} & =\left[\begin{array}{cc}
12 & -6 \\
6 & 12
\end{array}\right] \\
i= & {\left[\begin{array}{cc}
0 & -1 \\
1 & 0
\end{array}\right] } \\
z_{1} z_{2} & =\left[\begin{array}{cc}
12 & -6 \\
6 & 12
\end{array}\right]\left[\begin{array}{cc}
0 & -1 \\
1 & 0
\end{array}\right]=\left[\begin{array}{cc}
-6 & -12 \\
12 & -6
\end{array}\right] .
\end{aligned}
$$

\subsubsection{The Norm of a Complex Number}
Compute the norm of $z_{1}$.

Complex Number

$$
\begin{aligned}
z_{1} & =12+6 i \\
\left|z_{1}\right| & =\sqrt{12^{2}+6^{2}} \approx 13.416
\end{aligned}
$$

Ordered Pair

$$
\begin{aligned}
z_{1} & =(12,6) \\
\left|z_{1}\right| & =\sqrt{12^{2}+6^{2}} \approx 13.416 .
\end{aligned}
$$

Matrix

$$
\begin{aligned}
z_{1} & =\left[\begin{array}{cc}
12 & -6 \\
6 & 12
\end{array}\right] \\
\left|z_{1}\right| & =\left|\begin{array}{cc}
12 & -6 \\
6 & 12
\end{array}\right|=\sqrt{12^{2}+6^{2}} \approx 13.416
\end{aligned}
$$

\subsubsection{The Complex Conjugate of a Complex Number}
Compute the complex conjugate of the following.

\section{Complex Number}
$$
\begin{aligned}
(2+3 i)^{*} & =2-3 i \\
1^{*} & =1 \\
i^{*} & =-i
\end{aligned}
$$

\section{Ordered Pair}
$$
\begin{aligned}
& (2,3)^{*}=(2,-3) \\
& (1,0)^{*}=\left(\begin{array}{ll}
1, & 0
\end{array}\right) \\
& (0,1)^{*}=\left(\begin{array}{ll}
0,-1
\end{array}\right)
\end{aligned}
$$

\section{Matrix}
$$
\begin{aligned}
z & =\left[\begin{array}{cc}
2 & -3 \\
3 & 2
\end{array}\right] \\
z^{*} & =\left[\begin{array}{cc}
2 & 3 \\
-3 & 2
\end{array}\right] \\
1 & =\left[\begin{array}{ll}
1 & 0 \\
0 & 1
\end{array}\right] \\
1^{*} & =\left[\begin{array}{ll}
1 & 0 \\
0 & 1
\end{array}\right] \\
i & =\left[\begin{array}{cc}
0 & -1 \\
1 & 0
\end{array}\right] \\
i^{*} & =\left[\begin{array}{cc}
0 & 1 \\
-1 & 0
\end{array}\right] .
\end{aligned}
$$

\subsubsection{The Quotient of Two Complex Numbers}
Compute the quotient $(2+3 i) /(3+4 i)$.

\section{Complex Number}
$$
\begin{aligned}
\frac{2+3 i}{3+4 i} & =\frac{(2+3 i)}{(3+4 i)} \frac{(3-4 i)}{(3-4 i)} \\
& =\frac{6-8 i+9 i+12}{25} \\
& =\frac{18}{25}+\frac{1}{25} i
\end{aligned}
$$

Test

$$
\begin{aligned}
(3+4 i)\left(\frac{18}{25}+\frac{1}{25} i\right) & =\frac{54}{25}+\frac{3}{25} i+\frac{72}{25} i-\frac{4}{25} \\
& =2+3 i .
\end{aligned}
$$

\section{Ordered Pair}
$$
\begin{aligned}
\frac{(2,3)}{(3,4)} & =\frac{(2,3)}{(3,4)} \frac{(3,-4)}{(3,-4)} \\
& =\frac{(6+12,1)}{(9+16,0)} \\
& =\left(\frac{18}{25}, \frac{1}{25}\right) .
\end{aligned}
$$

Matrix

$$
\begin{aligned}
z_{1} & =\left[\begin{array}{cc}
2 & -3 \\
3 & 2
\end{array}\right] \\
z_{2} & =\left[\begin{array}{cc}
3 & -4 \\
4 & 3
\end{array}\right] \\
\frac{z_{1}}{z_{2}} & =z_{1} z_{2}^{-1} \\
& =\frac{1}{25}\left[\begin{array}{cc}
2 & -3 \\
3 & 2
\end{array}\right]\left[\begin{array}{cc}
3 & 4 \\
-4 & 3
\end{array}\right] \\
& =\frac{1}{25}\left[\begin{array}{cc}
18 & -1 \\
1 & 18
\end{array}\right] .
\end{aligned}
$$

\subsubsection{Divide a Complex Number by $i$}
Divide $2+3 i$ by $i$.

\section{Complex Number}
$$
\begin{aligned}
\frac{2+3 i}{0+i} & =\frac{(2+3 i)}{(0+i)} \frac{(0-i)}{(0-i)} \\
& =\frac{-2 i+3}{1} \\
& =3-2 i
\end{aligned}
$$

Test

$$
i(3-2 i)=2+3 i
$$

\section{Ordered Pair}
$$
\begin{aligned}
\frac{(2,3)}{(0,1)} & =\frac{(2,3)}{(0,1)} \frac{(0,-1)}{(0,-1)} \\
& =\frac{(3,-2)}{(1,0)} \\
& =(3,-2) .
\end{aligned}
$$

Matrix

$$
\begin{aligned}
z & =\left[\begin{array}{cc}
2 & -3 \\
3 & 2
\end{array}\right] \\
i & =\left[\begin{array}{cc}
0 & -1 \\
1 & 0
\end{array}\right] \\
i^{-1} & =\left[\begin{array}{cc}
0 & 1 \\
-1 & 0
\end{array}\right] \\
z i^{-1} & =\left[\begin{array}{ll}
2 & -3 \\
3 & 2
\end{array}\right]\left[\begin{array}{cc}
0 & 1 \\
-1 & 0
\end{array}\right]=\left[\begin{array}{cc}
3 & 2 \\
-2 & 3
\end{array}\right] .
\end{aligned}
$$

\subsubsection{Divide a Complex Number by $-i$}
Divide $2+3 i$ by $-i$.

Complex Number

$$
\begin{aligned}
\frac{2+3 i}{0-i} & =\frac{(2+3 i)}{(0-i)} \frac{(0+i)}{(0+i)} \\
& =\frac{2 i-3}{1} \\
& =-3+2 i
\end{aligned}
$$

Test

$$
-i(-3+2 i)=2+3 i
$$

\section{Ordered Pair}
$$
\begin{aligned}
\frac{(2,3)}{(0,-1)} & =\frac{(2,3)}{(0,-1)} \frac{(0,1)}{(0,1)} \\
& =\frac{(-3,2)}{1} \\
& =(-3,2)
\end{aligned}
$$

\section{Matrix}
$$
\begin{aligned}
z & =\left[\begin{array}{cc}
2 & -3 \\
3 & 2
\end{array}\right] \\
-i & =\left[\begin{array}{cc}
0 & 1 \\
-1 & 0
\end{array}\right] \\
-i^{-1} & =\left[\begin{array}{cc}
0 & -1 \\
1 & 0
\end{array}\right] \\
z\left(-i^{-1}\right) & =\left[\begin{array}{cc}
2 & -3 \\
3 & 2
\end{array}\right]\left[\begin{array}{cc}
0 & -1 \\
1 & 0
\end{array}\right]=\left[\begin{array}{cc}
-3 & -2 \\
2 & -3
\end{array}\right] .
\end{aligned}
$$

\subsubsection{The Inverse of a Complex Number}
Compute the inverse of $2+3 i$.

\section{Complex Number}
$$
\begin{aligned}
\frac{1}{2+3 i} & =\frac{1}{(2+3 i)} \frac{(2-3 i)}{(2-3 i)} \\
& =\frac{2-3 i}{13} \\
& =\frac{2}{13}-\frac{3}{13} i .
\end{aligned}
$$

\section{Ordered Pair}
$$
\begin{aligned}
\frac{1}{(2,3)} & =\frac{1}{(2,3)} \frac{(2,-3)}{(2,-3)} \\
& =\frac{(2,-3)}{13} \\
& =\left(\frac{2}{13},-\frac{3}{13}\right) .
\end{aligned}
$$

Matrix

$$
\begin{aligned}
z & =\left[\begin{array}{lc}
2 & -3 \\
3 & 2
\end{array}\right] \\
z^{-1} & =\frac{1}{13}\left[\begin{array}{cc}
2 & 3 \\
-3 & 2
\end{array}\right] .
\end{aligned}
$$

\subsubsection{The Inverse of $i$}
Compute the inverse of $i$.

\section{Complex Number}
$$
\begin{aligned}
\frac{1}{0+i} & =\frac{1}{(0+i)} \frac{(0-i)}{(0-i)} \\
& =\frac{-i}{1}=-i
\end{aligned}
$$

\section{Ordered Pair}
$$
\begin{aligned}
\frac{1}{(0,1)} & =\frac{1}{(0,1)} \frac{(0,-1)}{(0,-1)} \\
& =\frac{(0,-1)}{(1,0)}=(0,-1)=-i .
\end{aligned}
$$

Matrix

$$
\begin{aligned}
i & =\left[\begin{array}{cc}
0 & -1 \\
1 & 0
\end{array}\right] \\
i^{-1} & =\left[\begin{array}{cc}
0 & 1 \\
-1 & 0
\end{array}\right]=-i .
\end{aligned}
$$

\subsubsection{The Inverse of $-i$}
Compute the inverse of $-i$.

\section{Complex Number}
$$
\begin{aligned}
\frac{1}{0-i} & =\frac{1}{(0-i)} \frac{(0+i)}{(0+i)} \\
& =\frac{i}{1}=i
\end{aligned}
$$

\section{Ordered Pair}
$$
\begin{aligned}
\frac{1}{(0,-1)} & =\frac{1}{(0,-1)} \frac{(0,1)}{(0,1)} \\
& =\frac{(0,1)}{(1,0)}=(0,1)=i .
\end{aligned}
$$

\section{Matrix}
$$
\begin{aligned}
-i & =\left[\begin{array}{cc}
0 & 1 \\
-1 & 0
\end{array}\right] \\
-i^{-1} & =\left[\begin{array}{cc}
0 & -1 \\
1 & 0
\end{array}\right]=i .
\end{aligned}
$$

\begin{thebibliography}{99}
  \bibitem{bib3-1} Vince, J.: Imaginary Mathematics for Computer Science. Springer, Berlin (2018). ISBN 978-3319-94636-8
  \bibitem{bib3-2} Descartes, R.: La Géométrie (1637) (There is an English translation by Michael Mahoney) Dover, New York (1979)
\end{thebibliography}
