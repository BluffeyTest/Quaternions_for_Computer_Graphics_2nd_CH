\chap{复数}

\section{介绍}
In this chapter we discover how equations that have no real roots give rise to imaginary $i$ which squares to -1 . This, in turn, leads us to complex numbers and how they are manipulated algebraically. Many of the qualities associated with quaternions are found in complex numbers, which is why they are worthy of close examination. Readers interested in this subject may want to examine the author's book Imaginary Mathematics for Computer Science [1].

\section{虚数}
Imaginary numbers were invented to resolve problems where an equation has no real roots, such as $x^{2}+16=0$. The simple idea of declaring the existence of a quantity $i$, such that $i^{2}=-1$, permits us to express the solution to this equation as

$$
x= \pm 4 i
$$

It is pointless trying to discover what $i$ really is, $i$ really is just a mathematical object that squares to -1 . Nevertheless, it is a wonderful invention and does lend itself to a graphical interpretation, which we investigate in the next chapter.

In 1637 the French mathematician René Descartes (1596-1650), published $L a$ Géométrie [2], in which he stated that numbers incorporating $\sqrt{-1}$ were 'imaginary', and for centuries this label has stuck. Unfortunately, it was a derogatory remark, and there is nothing imaginary about $i$-it really is just something that squares to -1 , and when embedded within algebra creates some amazing patterns.

There doesn't appear to be any consensus concerning what to call the set of imaginary numbers-in fact it is even argued that one is unnecessary. However, if you decide on using one, the possibilities are $\mathbb{I}$ or $i \mathbb{R}$. Consequently, an imaginary number can be defined as

$$
b i \in i \mathbb{R}, \quad i^{2}=-1
$$

\section{\boldmath$i$的平方}
As $i^{2}=-1$ then it should be possible to raise $i$ to other powers. For example,

$$
i^{4}=i^{2} i^{2}=(-1)(-1)=1
$$

and

$$
i^{5}=i i^{4}=i
$$

Therefore, we have the sequence:

\begin{center}
\begin{tabular}{lllllll}
\hline
$i^{0}$ & $i^{1}$ & $i^{2}$ & $i^{3}$ & $i^{4}$ & $i^{5}$ & $i^{6}$ \\
\hline
1 & $i$ & -1 & $-i$ & 1 & $i$ & -1 \\
\hline
\end{tabular}
\end{center}

The cyclic pattern $(1, i,-1,-i, 1, \ldots)$ is quite striking, and reminds us of a similar pattern $(x, y,-x,-y, x, \ldots)$ which is generated by rotating around the Cartesian axes in a counter-clockwise direction. Such a similarity cannot be ignored, for when the real number line is combined with a vertical imaginary axis it gives rise to what is called the complex plane. But more of this in the following chapter.

The above sequence is summarised as:

$$
\begin{gathered}
i^{4 n}=1 \\
i^{4 n+1}=i \\
i^{4 n+2}=-1 \\
i^{4 n+3}=-i
\end{gathered}
$$

where $n \in \mathbb{N}$.

But what about negative powers? Well, they, too, are also possible. Consider $i^{-1}$, which is evaluated as follows:

$$
i^{-1}=\frac{1}{i}=\frac{1(-i)}{i(-i)}=\frac{-i}{1}=-i
$$

Similarly

$$
i^{-2}=\frac{1}{i^{2}}=\frac{1}{-1}=-1
$$

and

$$
i^{-3}=i^{-1} i^{-2}=-i(-1)=i
$$

The sequence associated with increasing negative powers is:

\begin{center}
\begin{tabular}{lllllll}
\hline
$i^{0}$ & $i^{-1}$ & $i^{-2}$ & $i^{-3}$ & $i^{-4}$ & $i^{-5}$ & $i^{-6}$ \\
\hline
1 & $-i$ & -1 & $i$ & 1 & $-i$ & -1 \\
\hline
\end{tabular}
\end{center}

This time the cyclic pattern is reversed to $(1,-i,-1, i, 1, \ldots)$ and is similar to the pattern $(x,-y,-x, y, x, \ldots)$ which is generated by rotating around the Cartesian axes in a clockwise direction.

Perhaps the strangest power of all is itself: $i^{i}$, which happens to equal $e^{-\pi / 2}=$ $0.207879576 .$. , and is explained in Chap. 4. Having reviewed certain features of imaginary $i$, let's discover what happens when it's combined with real numbers.

\section{复数的定义}
By definition, a complex number is the combination of a real number and an imaginary number, and is expressed as

$$
z=a+b i, \quad a, b \in \mathbb{R}, \quad i^{2}=-1
$$

The set of complex numbers is $\mathbb{C}$, which permits us to write $z \in \mathbb{C}$. For example, $3+4 i$ is a complex number where 3 is the real part and $4 i$ is the imaginary part. The following are all complex numbers:

$$
3, \quad 3+4 i, \quad-4-6 i, \quad 7 i, \quad 5.5+6.7 i
$$

A real number is also a complex number-it just has no imaginary part. This leads to the idea that the set of real numbers is a subset of complex numbers, which is expressed as:

$$
\mathbb{R} \subset \mathbb{C}
$$

where $\subset$ means is a subset of.

Although some mathematicians place $i$ before its multiplier: $i 4$, others place it after the multiplier: $4 i$, which is the convention used in this book. However, when $i$ is associated with trigonometric functions, it is good practice to place it before the function to avoid any confusion with the function's angle. For example, sin $\alpha i$ could imply that the angle is imaginary, whereas $i \sin \alpha$ implies that the value of $\sin \alpha$ is imaginary. Therefore, a complex number can be constructed in all sorts of ways:

$$
\sin \alpha+i \cos \beta, \quad 2-i \tan \alpha, \quad 23+x^{2} i
$$

In general, we write a complex number as $a+b i$ and subject it to the normal rules of real algebra. All that we have to remember is that whenever we encounter $i^{2}$ it is replaced by -1 . For example:

$$
\begin{aligned}
(2+3 i)(3+4 i) & =2 \times 3+2 \times 4 i+3 i \times 3+3 i \times 4 i \\
& =6+8 i+9 i+12 i^{2} \\
& =6+17 i-12 \\
& =-6+17 i .
\end{aligned}
$$

\subsection{复数的加减法}
Given two complex numbers:

$$
\begin{aligned}
& z_{1}=a_{1}+b_{1} i \\
& z_{2}=a_{2}+b_{2} i
\end{aligned}
$$

then,

$$
z_{1} \pm z_{2}=\left(a_{1} \pm a_{2}\right)+\left(b_{1} \pm b_{2}\right) i
$$

where the real and imaginary parts are added or subtracted, respectively. The operations are closed, so long as $a_{1}, b_{1}, a_{2}, b_{2} \in \mathbb{R}$.

For example:

$$
\begin{aligned}
z_{1} & =2+3 i \\
z_{2} & =4+2 i \\
z_{1}+z_{2} & =6+5 i \\
z_{1}-z_{2} & =-2+i .
\end{aligned}
$$

\subsection{复数乘以标量}
A complex number is multiplied by a scalar using normal algebraic rules. For example, the complex number $a+b i$ is multiplied by the scalar $\lambda$ as follows:

$$
\lambda(a+b i)=\lambda a+\lambda b i
$$

an example is

$$
3(2+5 i)=6+15 i
$$

\subsection{复数的乘积}
Given two complex numbers:

$$
\begin{aligned}
& z_{1}=a_{1}+b_{1} i \\
& z_{2}=a_{2}+b_{2} i
\end{aligned}
$$

their product is

$$
\begin{aligned}
z_{1} z_{2} & =\left(a_{1}+b_{1} i\right)\left(a_{2}+b_{2} i\right) \\
& =a_{1} a_{2}+a_{1} b_{2} i+b_{1} a_{2} i+b_{1} b_{2} i^{2} \\
& =\left(a_{1} a_{2}-b_{1} b_{2}\right)+\left(a_{1} b_{2}+b_{1} a_{2}\right) i
\end{aligned}
$$

which is another complex number and confirms that the operation is closed. For example:

$$
\begin{aligned}
z_{1} & =3+4 i \\
z_{2} & =3-2 i \\
z_{1} z_{2} & =(3+4 i)(3-2 i) \\
& =9-6 i+12 i-8 i^{2} \\
& =9+6 i+8 \\
& =17+6 i .
\end{aligned}
$$

Note that the addition, subtraction and multiplication of complex numbers obey the normal axioms of algebra.

\subsection{复数的平方}
Given a complex number $z$, its square $z^{2}$ is given by:

$$
\begin{aligned}
z & =a+b i \\
z^{2} & =(a+b i)(a+b i) \\
& =\left(a^{2}-b^{2}\right)+2 a b i
\end{aligned}
$$

For example:

$$
\begin{aligned}
z & =4+3 i \\
z^{2} & =(4+3 i)(4+3 i) \\
& =\left(4^{2}-3^{2}\right)+2 \times 4 \times 3 i \\
& =7+24 i .
\end{aligned}
$$

\subsection{复数的范数}
The norm, modulus or absolute value of a complex number $z$ is written $|z|$ and by definition is

$$
\begin{aligned}
z & =a+b i \\
|z| & =\sqrt{a^{2}+b^{2}}
\end{aligned}
$$

For example, the norm of $3+4 i$ is 5 . We'll see why this is so when we cover the polar representation of a complex number.

\subsection{复数的复共轭}
The product of two complex numbers, where the only difference between them is the sign of the imaginary part, gives rise to a special result:

$$
\begin{aligned}
(a+b i)(a-b i) & =a^{2}-a b i+a b i-b^{2} i^{2} \\
& =a^{2}+b^{2}
\end{aligned}
$$

This type of product always results in a real quantity and is used to resolve the quotient of two complex numbers. Because this real value is such an interesting result, $a-b i$ is called the complex conjugate of $z=a+b i$, and is written either with a bar $\bar{z}$, or an asterisk $z^{*}$, and implies that

$$
z z^{*}=a^{2}+b^{2}=|z|^{2} .
$$

For example:

$$
\begin{aligned}
z & =3+4 i \\
z^{*} & =3-4 i \\
z z^{*} & =9+16=25 .
\end{aligned}
$$

\subsection{复数的商}
The complex conjugate provides us with a mechanism to divide one complex number by another. For instance, the quotient

$$
\frac{a_{1}+b_{1} i}{a_{2}+b_{2} i}
$$

is resolved by multiplying the numerator and denominator by the denominator's complex conjugate $a_{2}-b_{2} i$ to create a real denominator:

$$
\begin{aligned}
\frac{a_{1}+b_{1} i}{a_{2}+b_{2} i} & =\frac{\left(a_{1}+b_{1} i\right)\left(a_{2}-b_{2} i\right)}{\left(a_{2}+b_{2} i\right)\left(a_{2}-b_{2} i\right)} \\
& =\frac{a_{1} a_{2}-a_{1} b_{2} i+b_{1} a_{2} i-b_{1} b_{2} i^{2}}{a_{2}^{2}+b_{2}^{2}} \\
& =\left(\frac{a_{1} a_{2}+b_{1} b_{2}}{a_{2}^{2}+b_{2}^{2}}\right)+\left(\frac{b_{1} a_{2}-a_{1} b_{2}}{a_{2}^{2}+b_{2}^{2}}\right) i .
\end{aligned}
$$

For example, to evaluate

$$
\frac{4+3 i}{3+4 i}
$$

we multiply top and bottom by the complex conjugate $3-4 i$ :

$$
\begin{aligned}
\frac{4+3 i}{3+4 i} & =\frac{(4+3 i)(3-4 i)}{(3+4 i)(3-4 i)} \\
& =\frac{12-16 i+9 i-12 i^{2}}{25} \\
& =\frac{24}{25}-\frac{7}{25} i
\end{aligned}
$$

\subsection{复数的逆}
To compute the inverse of $z=a+b i$ we start with

$$
z^{-1}=\frac{1}{z} .
$$

Multiplying top and bottom by $z^{*}$ we have

$$
z^{-1}=\frac{z *}{z z^{*}}
$$

But we have previously shown that $z z^{*}=|z|^{2}$, therefore,

$$
\begin{aligned}
z^{-1} & =\frac{z^{*}}{|z|^{2}} \\
& =\left(\frac{a}{a^{2}+b^{2}}\right)-\left(\frac{b}{a^{2}+b^{2}}\right) i
\end{aligned}
$$

As an example, the inverse of $3+4 i$ is

$$
(3+4 i)^{-1}=\frac{3}{25}-\frac{4}{25} i
$$

Let's test this result by multiplying $3+4 i$ by its inverse:

$$
(3+4 i)\left(\frac{3}{25}-\frac{4}{25} i\right)=\frac{9}{25}-\frac{12}{25} i+\frac{12}{25} i+\frac{16}{25}=1
$$

which confirms the correctness of the result.

\subsection{\boldmath $\pm i$的平方根}
To find $\sqrt{i}$ we assume that the roots are complex. Therefore, we start with

$$
\begin{aligned}
\sqrt{i} & =a+b i \\
i & =(a+b i)(a+b i) \\
& =a^{2}+2 a b i-b^{2} \\
& =a^{2}-b^{2}+2 a b i
\end{aligned}
$$

and equating real and imaginary parts we have

$$
\begin{aligned}
a^{2}-b^{2} & =0 \\
2 a b & =1 .
\end{aligned}
$$

From this we deduce that

$$
a=b= \pm \frac{\sqrt{2}}{2}
$$

Therefore, the roots are

$$
\sqrt{i}= \pm \frac{\sqrt{2}}{2}(1+i)
$$

Let's test this result by squaring each root to ensure the answer is $i$ :

$$
\left( \pm \frac{\sqrt{2}}{2}\right)^{2}(1+i)(1+i)=\frac{1}{2} 2 i=i
$$

To find $\sqrt{-i}$ we assume that the roots are complex. Therefore, we start with

$$
\begin{aligned}
\sqrt{-i} & =a+b i \\
-i & =(a+b i)(a+b i) \\
& =a^{2}+2 a b i-b^{2} \\
& =a^{2}-b^{2}+2 a b i
\end{aligned}
$$

and equating real and imaginary parts we have

$$
\begin{aligned}
a^{2}-b^{2} & =0 \\
2 a b & =-1
\end{aligned}
$$

From this we deduce that

$$
a=b= \pm \frac{\sqrt{2}}{2} i
$$

Therefore, the roots are

$$
\begin{aligned}
\sqrt{-i} & = \pm \frac{\sqrt{2}}{2} i(1+i) \\
& = \pm \frac{\sqrt{2}}{2}(-1+i) \\
& = \pm \frac{\sqrt{2}}{2}(1-i)
\end{aligned}
$$

Let's test this result by squaring each root to ensure the answer is $-i$ :

$$
\left( \pm \frac{\sqrt{2}}{2}\right)^{2}(1-i)(1-i)=-\frac{1}{2} 2 i=-i
$$

We use these roots in the next chapter to investigate the rotational properties of complex numbers.

\section{复数的域结构}
The set of complex numbers $\mathbb{C}$ is a field, because it satisfies the previously defined rules for a field.

\section{有序对}
So far, we have chosen to express a complex number as $a+b i$ where we can distinguish between the real and imaginary parts. However, one thing we cannot assume is that the real part is always first, and the imaginary part second, because $b i+a$ is also a complex number. Consequently, two functions are employed to extract the real and imaginary coefficients as follows:

$$
\begin{aligned}
& \operatorname{Re}(a+b i)=a \\
& \operatorname{Im}(a+b i)=b
\end{aligned}
$$

and leads us to the idea of representing a complex number by an ordered pair where order is guaranteed:

$$
a+b i=(a, b)
$$

where $b$ follows $a$ to define the order. Thus the set $\mathbb{C}$ of complex numbers is equivalent to the set $\mathbb{R}^{2}$ of ordered pairs $(a, b)$.

Writing a complex number as an ordered pair was a great contribution, and first made by Hamilton in 1833. Such notation is very succinct and free from any imaginary term, which can be added whenever required.

\subsection{有序对的加法和减法}
Given two complex numbers:

$$
\begin{aligned}
& z_{1}=a_{1}+b_{1} i \\
& z_{2}=a_{2}+b_{2} i
\end{aligned}
$$

they are written as ordered pairs:

$$
\begin{aligned}
& z_{1}=\left(a_{1}, b_{1}\right) \\
& z_{2}=\left(a_{2}, b_{2}\right)
\end{aligned}
$$

and

$$
z_{1} \pm z_{2}=\left(a_{1} \pm a_{2}, b_{1} \pm b_{2}\right)
$$

where the two parts are added or subtracted, respectively.

For example:

$$
\begin{aligned}
z_{1} & =2+3 i=(2,3) \\
z_{2} & =4+2 i=(4,2) \\
z_{1}+z_{2} & =(6,5) \\
z_{1}-z_{2} & =(-2,1) .
\end{aligned}
$$

\subsection{将有序对乘以标量}
We have already seen how a complex number is multiplied by a scalar, which must be the same as ordered pairs:

$$
\lambda(a, b)=(\lambda a, \lambda b)
$$

An example is

$$
3(2,5)=(6,15)
$$

\subsection{有序对乘积}
Given two complex numbers:

$$
\begin{aligned}
& z_{1}=a_{1}+b_{1} i \\
& z_{2}=a_{2}+b_{2} i
\end{aligned}
$$

their product is

$$
z_{1} z_{2}=\left(a_{1} a_{2}-b_{1} b_{2}\right)+\left(a_{1} b_{2}+b_{1} a_{2}\right) i
$$

which must also work with ordered pairs:

$$
\begin{aligned}
z_{1} & =\left(a_{1}, b_{1}\right) \\
z_{2} & =\left(a_{2}, b_{2}\right) \\
z_{1} z_{2} & =\left(a_{1}, b_{1}\right)\left(a_{2}, b_{2}\right) \\
& =\left(a_{1} a_{2}-b_{1} b_{2}, a_{1} b_{2}+b_{1} a_{2}\right)
\end{aligned}
$$

For example:

$$
\begin{aligned}
z_{1} & =(6,2) \\
z_{2} & =(4,3) \\
z_{1} z_{2} & =(6,2)(4,3) \\
& =(24-6,18+8) \\
& =(18,26) .
\end{aligned}
$$

\subsection{有序对平方}
The square of a complex number is given by:

$$
\begin{aligned}
z & =a+b i \\
z^{2} & =(a+b i)(a+b i) \\
& =\left(a^{2}-b^{2}\right)+2 a b i
\end{aligned}
$$

Therefore, the square of an ordered pair is:

$$
\begin{aligned}
z & =(a, b) \\
z^{2} & =(a, b)(a, b) \\
& =\left(a^{2}-b^{2}, 2 a b\right)
\end{aligned}
$$

For example:

$$
\begin{aligned}
z & =(4,3) \\
z^{2} & =(4,3)(4,3) \\
& =\left(4^{2}-3^{2}, 2 \times 4 \times 3\right) \\
& =(7,24)
\end{aligned}
$$

Let's continue to develop an algebra based upon ordered pairs that is identical to the algebra of complex numbers. We start by writing

$$
\begin{aligned}
z & =(a, b) \\
& =(a, 0)+(0, b) \\
& =a(1,0)+b(0,1)
\end{aligned}
$$

which creates the unit ordered pairs $(1,0)$ and $(0,1)$.

Now let's compute the product $(1,0)(1,0)$ :

$$
\begin{aligned}
(1,0)(1,0) & =(1-0,0) \\
& =(1,0)
\end{aligned}
$$

which shows that $(1,0)$ behaves like the real number 1 . i.e. $(1,0)=1$.

Next, let's compute the product $(0,1)(0,1)$ :

$$
\begin{aligned}
(0,1)(0,1) & =(0-1,0) \\
& =(-1,0)
\end{aligned}
$$

which is the real number -1 :

$$
(0,1)^{2}=-1
$$

or

$$
(0,1)=\sqrt{-1} \text { and is imaginary }
$$

This means that the ordered pair $(a, b)$, together with its associated rules, represents a complex number. i.e. $(a, b) \equiv a+b i$.

\subsection{有序对范数}
The norm, modulus or absolute value of an ordered pair $z$ is written $|z|$ and by definition is

$$
\begin{aligned}
z & =(a, b) \\
|z| & =\sqrt{a^{2}+b^{2}}
\end{aligned}
$$

For example, the norm of $(3,4)$ is 5.

\subsection{有序对的复共轭}
The complex conjugate of $z=a+b i$ is defined as $z^{*}=a-b i$, which in terms of an ordered pair is $z^{*}=(a,-b)$ :

$$
\begin{aligned}
z & =(a, b) \\
z^{*} & =(a,-b) \\
z z^{*} & =(a, b)(a,-b) \\
& =\left(a^{2}+b^{2}, b a-a b\right) \\
& =\left(a^{2}+b^{2}, 0\right) \\
& =a^{2}+b^{2}=|z|^{2} .
\end{aligned}
$$

\subsection{有序对的商}
The technique for resolving $z_{1} / z_{2}$ is to multiply the expression by $z_{2}^{*} / z_{2}^{*}$, which using ordered pairs is

$$
\begin{aligned}
\frac{z_{1}}{z_{2}} & =\frac{\left(a_{1}, b_{1}\right)}{\left(a_{2}, b_{2}\right)} \\
& =\frac{\left(a_{1}, b_{1}\right)}{\left(a_{2}, b_{2}\right)} \frac{\left(a_{2},-b_{2}\right)}{\left(a_{2},-b_{2}\right)} \\
& =\frac{\left(a_{1} a_{2}+b_{1} b_{2}, b_{1} a_{2}-a_{1} b_{2}\right)}{\left(a_{2}^{2}+b_{2}^{2}, 0\right)} \\
& =\left(\frac{a_{1} a_{2}+b_{1} b_{2}}{a_{2}^{2}+b_{2}^{2}}, \frac{b_{1} a_{2}-a_{1} b_{2}}{a_{2}^{2}+b_{2}^{2}}\right) .
\end{aligned}
$$

For example, to evaluate

$$
\frac{(4,3)}{(3,4)}
$$

we multiply top and bottom by the complex conjugate $(3,-4)$ :

$$
\begin{aligned}
\frac{(4,3)}{(3,4)} & =\frac{(4,3)(3,-4)}{(3,4)(3,-4)} \\
& =\left(\frac{12+12}{25}, \frac{9-16}{25}\right) \\
& =\left(\frac{24}{25},-\frac{7}{25}\right) .
\end{aligned}
$$

\subsection{有序对的逆}
We have previously shown that $z^{-1}$ is

$$
z^{-1}=\frac{z^{*}}{z z^{*}}=\frac{z^{*}}{|z|^{2}}
$$

which using ordered pairs is

$$
\begin{aligned}
z & =(a, b) \\
z^{-1} & =\frac{(a,-b)}{(a, b)(a,-b)} \\
& =\frac{(a,-b)}{\left(a^{2}+b^{2}, 0\right)} \\
& =\left(\frac{a}{a^{2}+b^{2}}, \frac{-b}{a^{2}+b^{2}}\right) .
\end{aligned}
$$

As an illustration, the inverse of $(3,4)$ is

$$
(3,4)^{-1}=\left(\frac{3}{25},-\frac{4}{25}\right)
$$

Let's test this result by multiplying $(3,4)$ by its inverse:

$$
\begin{aligned}
(3,4)\left(\frac{3}{25},-\frac{4}{25}\right) & =\left(\frac{9}{25}+\frac{16}{25},-\frac{12}{25}+\frac{12}{25}\right) \\
& =(1,0) .
\end{aligned}
$$

\subsection{\boldmath $\pm i$的平方根}
To find $\sqrt{i}$ we assume that the roots are complex. Therefore, we start with

$$
\begin{aligned}
\sqrt{i} & =(a, b) \\
i & =(a, b)(a, b) \\
(0,1) & =\left(a^{2}-b^{2}, 2 a b\right)
\end{aligned}
$$

and equating left and right ordered elements we have

$$
\begin{aligned}
a^{2}-b^{2} & =0 \\
2 a b & =1 .
\end{aligned}
$$

From this we deduce that

$$
a=b= \pm \frac{\sqrt{2}}{2}
$$

Therefore, the roots are

$$
\sqrt{i}= \pm \frac{\sqrt{2}}{2}(1,1)
$$

Let's test this result by squaring each root to ensure the answer is $i$ :

$$
\left( \pm \frac{\sqrt{2}}{2}\right)^{2}(1,1)(1,1)=\frac{1}{2}(0,2)=(0,1)
$$

To find $\sqrt{-i}$ we assume that the roots are complex. Therefore, we start with

$$
\begin{aligned}
\sqrt{-i} & =(a, b) \\
-i & =(a, b)(a, b) \\
(0,-1) & =\left(a^{2}-b^{2}, 2 a b\right)
\end{aligned}
$$

and equating left and right ordered elements we have

$$
\begin{aligned}
a^{2}-b^{2} & =0 \\
2 a b & =-1
\end{aligned}
$$

From this we deduce that

$$
\begin{aligned}
a=b & = \pm \frac{\sqrt{2}}{2} i \\
& = \pm \frac{\sqrt{2}}{2}(0,1)(1,1) \\
& = \pm \frac{\sqrt{2}}{2}(-1,1)
\end{aligned}
$$

Therefore, the roots are

$$
\sqrt{-i}= \pm \frac{\sqrt{2}}{2}(1,-1)
$$

Let's test this result by squaring each root to ensure the answer is $-i$ :

$$
\left( \pm \frac{\sqrt{2}}{2}\right)^{2}(1,-1)(1,-1)=\frac{1}{2}(0,-2)=(0,-1)
$$

It is obvious from the above definitions that ordered pairs provide an alternative notation for expressing complex numbers, where the imaginary feature is embedded within the product axiom. We will also use ordered pairs to define a quaternion with three imaginary terms, which when incorporated within the product axiom remain hidden.

\section{复数的矩阵表示}
As quaternions have a matrix representation, perhaps we should investigate the matrix representation for a complex number.

Although I have only hinted that $i$ can be regarded as some sort of rotational operator, this is the perfect way of visualising it. In Chap. 4 we discover that multiplying a complex number by $i$ effectively rotates the number $90^{\circ}$ anti-clockwise. So for the moment, it can be represented by a rotation matrix of $90^{\circ}$ :

$$
i \equiv\left[\begin{array}{cc}
\cos 90^{\circ} & -\sin 90^{\circ} \\
\sin 90^{\circ} & \cos 90^{\circ}
\end{array}\right]=\left[\begin{array}{cc}
0 & -1 \\
1 & 0
\end{array}\right]
$$

and the $2 \times 2$ identity matrix is

$$
\left[\begin{array}{ll}
1 & 0 \\
0 & 1
\end{array}\right]
$$

This permits us to write a complex number as:

$$
\begin{aligned}
a+b i & =a\left[\begin{array}{ll}
1 & 0 \\
0 & 1
\end{array}\right]+b\left[\begin{array}{cc}
0 & -1 \\
1 & 0
\end{array}\right] \\
& =\left[\begin{array}{ll}
a & 0 \\
0 & a
\end{array}\right]+\left[\begin{array}{cc}
0 & -b \\
b & 0
\end{array}\right] \\
& =\left[\begin{array}{cc}
a & -b \\
b & a
\end{array}\right] .
\end{aligned}
$$

Note that the matrix representing $i$ squares to -1 :

$$
\begin{aligned}
{\left[\begin{array}{cc}
0 & -1 \\
1 & 0
\end{array}\right]\left[\begin{array}{cc}
0 & -1 \\
1 & 0
\end{array}\right] } & =\left[\begin{array}{cc}
-1 & 0 \\
0 & -1
\end{array}\right] \\
& =-1\left[\begin{array}{ll}
1 & 0 \\
0 & 1
\end{array}\right] .
\end{aligned}
$$

However, we must also remember that $i^{2}=(-i)^{2}=-1$, which is interpreted as anti-clockwise and clockwise rotations in the complex plane. This is confirmed by:

$$
\begin{aligned}
{\left[\begin{array}{cc}
0 & 1 \\
-1 & 0
\end{array}\right]\left[\begin{array}{cc}
0 & 1 \\
-1 & 0
\end{array}\right] } & =\left[\begin{array}{cc}
-1 & 0 \\
0 & -1
\end{array}\right] \\
& =-1\left[\begin{array}{ll}
1 & 0 \\
0 & 1
\end{array}\right] .
\end{aligned}
$$

Now let's employ matrix notation for all the arithmetic operations used for complex numbers.

\subsection{复数的加减法}
Two complex numbers are added or subtracted as follows:

$$
\begin{aligned}
z_{1} & =a_{1}+b_{1} i \\
z_{2} & =a_{2}+b_{2} i \\
z_{1} & =\left[\begin{array}{cc}
a_{1} & -b_{1} \\
b_{1} & a_{1}
\end{array}\right] \\
z_{2} & =\left[\begin{array}{cc}
a_{2} & -b_{2} \\
b_{2} & a_{2}
\end{array}\right] \\
z_{1} \pm z_{2} & =\left[\begin{array}{cc}
a_{1} & -b_{1} \\
b_{1} & a_{1}
\end{array}\right] \pm\left[\begin{array}{cc}
a_{2} & -b_{2} \\
b_{2} & a_{2}
\end{array}\right] \\
& =\left[\begin{array}{ll}
a_{1} \pm a_{2} & -\left(b_{1} \pm b_{2}\right) \\
b_{1} \pm b_{2} & a_{1} \pm a_{2}
\end{array}\right] .
\end{aligned}
$$

For example:

$$
\begin{aligned}
z_{1} & =2+3 i \\
z_{2} & =4+2 i \\
z_{1} & =\left[\begin{array}{cc}
2 & -3 \\
3 & 2
\end{array}\right] \\
z_{2} & =\left[\begin{array}{cc}
4 & -2 \\
2 & 4
\end{array}\right] \\
z_{1} \pm z_{2} & =\left[\begin{array}{cc}
2 & -3 \\
3 & 2
\end{array}\right] \pm\left[\begin{array}{cc}
4 & -2 \\
2 & 4
\end{array}\right] \\
z_{1}+z_{2} & =\left[\begin{array}{cc}
6 & -5 \\
5 & 6
\end{array}\right]=6+5 i \\
z_{1}-z_{2} & =\left[\begin{array}{cc}
-2 & -1 \\
1 & -2
\end{array}\right]=-2+i .
\end{aligned}
$$

\subsection{两个复数的乘积}
The product of two complex numbers is computed as follows:

$$
\begin{aligned}
z_{1} & =a_{1}+b_{1} i \\
z_{2} & =a_{2}+b_{2} i \\
z_{1} z_{2} & =\left[\begin{array}{cc}
a_{1} & -b_{1} \\
b_{1} & a_{1}
\end{array}\right]\left[\begin{array}{cc}
a_{2} & -b_{2} \\
b_{2} & a_{2}
\end{array}\right] \\
& =\left[\begin{array}{cc}
a_{1} a_{2}-b_{1} b_{2} & -\left(a_{1} b_{2}+b_{1} a_{2}\right) \\
a_{1} b_{2}+b_{1} a_{2} & a_{1} a_{2}-b_{1} b_{2}
\end{array}\right] .
\end{aligned}
$$

For example:

$$
\begin{aligned}
z_{1} & =6+2 i \\
z_{2} & =4+3 i \\
z_{1} z_{2} & =\left[\begin{array}{cc}
6 & -2 \\
2 & 6
\end{array}\right]\left[\begin{array}{cc}
4 & -3 \\
3 & 4
\end{array}\right] \\
& =\left[\begin{array}{cc}
24-6 & -(18+8) \\
18+8 & 24-6
\end{array}\right] \\
& =\left[\begin{array}{cc}
18 & -26 \\
26 & 18
\end{array}\right] .
\end{aligned}
$$

\subsection{复数的范数平方}
The square of the norm is as the determinant of the matrix:

$$
\begin{aligned}
z & =a+b i \\
& =\left[\begin{array}{cc}
a & -b \\
b & a
\end{array}\right] \\
|z|^{2} & =a^{2}+b^{2}=\left|\begin{array}{cc}
a & -b \\
b & a
\end{array}\right| .
\end{aligned}
$$

\subsection{复数的复共轭}
The complex conjugate of a complex number is

$$
\begin{aligned}
z=a+b i & =\left[\begin{array}{cc}
a & -b \\
b & a
\end{array}\right] \\
z^{*}=a-b i & =\left[\begin{array}{cc}
a & b \\
-b & a
\end{array}\right] .
\end{aligned}
$$

The product $z z^{*}=a^{2}+b^{2}$ :

$$
\begin{aligned}
z z^{*} & =\left[\begin{array}{cc}
a & -b \\
b & a
\end{array}\right]\left[\begin{array}{cc}
a & b \\
-b & a
\end{array}\right] \\
& =\left[\begin{array}{cc}
a^{2}+b^{2} & 0 \\
0 & a^{2}+b^{2}
\end{array}\right] \\
& =\left(a^{2}+b^{2}\right)\left[\begin{array}{ll}
1 & 0 \\
0 & 1
\end{array}\right] .
\end{aligned}
$$

For example:

$$
\begin{aligned}
z & =3+4 i=\left[\begin{array}{cc}
3 & -4 \\
4 & 3
\end{array}\right] \\
z^{*} & =3-4 i=\left[\begin{array}{cc}
3 & 4 \\
-4 & 3
\end{array}\right] \\
z z^{*} & =\left[\begin{array}{cc}
3 & -4 \\
4 & 3
\end{array}\right]\left[\begin{array}{cc}
3 & 4 \\
-4 & 3
\end{array}\right]=\left[\begin{array}{cc}
25 & 0 \\
0 & 25
\end{array}\right] \\
& =25\left[\begin{array}{cc}
1 & 0 \\
0 & 1
\end{array}\right]
\end{aligned}
$$

\subsection{复数的逆}
The inverse of $2 \times 2$ matrix $\mathbf{A}$ is given by

$$
\begin{aligned}
\mathbf{A} & =\left[\begin{array}{ll}
a_{11} & a_{12} \\
a_{21} & a_{22}
\end{array}\right] \\
\mathbf{A}^{-1} & =\frac{1}{a_{11} a_{22}-a_{12} a_{21}}\left[\begin{array}{cc}
a_{22} & -a_{12} \\
-a_{21} & a_{12}
\end{array}\right]
\end{aligned}
$$

therefore, the inverse of $z$ is given by

$$
\begin{aligned}
z & =a+b i \\
z & =\left[\begin{array}{cc}
a & -b \\
b & a
\end{array}\right] \\
z^{-1} & =\frac{1}{a^{2}+b^{2}}\left[\begin{array}{cc}
a & b \\
-b & a
\end{array}\right] .
\end{aligned}
$$

For example:

$$
\begin{aligned}
z & =3+4 i \\
z & =\left[\begin{array}{cc}
3 & -4 \\
4 & 3
\end{array}\right] \\
z^{-1} & =\frac{1}{25}\left[\begin{array}{cc}
3 & 4 \\
-4 & 3
\end{array}\right] .
\end{aligned}
$$

\subsection{复数的商}
The quotient of two complex numbers is computed as follows:

$$
\begin{aligned}
z_{1} & =a_{1}+b_{1} i \\
z_{2} & =a_{2}+b_{2} i \\
\frac{z_{1}}{z_{2}} & =z_{1} z_{2}^{-1} \\
& =\left[\begin{array}{cc}
a_{1} & -b_{1} \\
b_{1} & a_{1}
\end{array}\right] \frac{1}{a_{2}^{2}+b_{2}^{2}}\left[\begin{array}{cc}
a_{2} & b_{2} \\
-b_{2} & a_{2}
\end{array}\right] \\
& =\frac{1}{a_{2}^{2}+b_{2}^{2}}\left[\begin{array}{cc}
a_{1} a_{2}+b_{1} b_{2} & -\left(b_{1} a_{2}-a_{1} b_{2}\right) \\
b_{1} a_{2}-a_{1} b_{2} & a_{1} a_{2}+b_{1} b_{2}
\end{array}\right] .
\end{aligned}
$$

For example:

$$
\begin{aligned}
z_{1} & =4+3 i \\
z_{2} & =3+4 i \\
\frac{z_{1}}{z_{2}} & =z_{1} z_{2}^{-1} \\
& =\left[\begin{array}{cc}
4 & -3 \\
3 & 4
\end{array}\right] \frac{1}{9+16}\left[\begin{array}{cc}
3 & 4 \\
-4 & 3
\end{array}\right] \\
& =\frac{1}{25}\left[\begin{array}{cc}
24 & 7 \\
-7 & 24
\end{array}\right] .
\end{aligned}
$$

\subsection{\boldmath $\pm i$的平方根}
To find $\sqrt{i}$ we assume that the roots are complex. Therefore, we start with

$$
\begin{aligned}
\sqrt{i} & =\left[\begin{array}{cc}
a & -b \\
b & a
\end{array}\right] \\
i & =\left[\begin{array}{cc}
a & -b \\
b & a
\end{array}\right]\left[\begin{array}{cc}
a & -b \\
b & a
\end{array}\right] \\
{\left[\begin{array}{cc}
0 & -1 \\
1 & 0
\end{array}\right] } & =\left[\begin{array}{cc}
a^{2}-b^{2} & -2 a b \\
2 a b & a^{2}-b^{2}
\end{array}\right]
\end{aligned}
$$

and equating left and right matrices we have

$$
\begin{aligned}
a^{2}-b^{2} & =0 \\
2 a b & =1 .
\end{aligned}
$$

From this we deduce that

$$
a=b= \pm \frac{\sqrt{2}}{2}
$$

Therefore, the roots are

$$
\sqrt{i}= \pm \frac{\sqrt{2}}{2}\left[\begin{array}{cc}
1 & -1 \\
1 & 1
\end{array}\right]
$$

Let's test this result by squaring each root to ensure the answer is $i$ :

$$
\left( \pm \frac{\sqrt{2}}{2}\right)^{2}\left[\begin{array}{cc}
1 & -1 \\
1 & 1
\end{array}\right]\left[\begin{array}{cc}
1 & -1 \\
1 & 1
\end{array}\right]=\frac{1}{2}\left[\begin{array}{cc}
0 & -2 \\
2 & 0
\end{array}\right]=i
$$

To find $\sqrt{-i}$ we assume that the roots are complex. Therefore, we start with

$$
\begin{aligned}
\sqrt{-i} & =\left[\begin{array}{cc}
a & -b \\
b & a
\end{array}\right] \\
-i & =\left[\begin{array}{cc}
a & -b \\
b & a
\end{array}\right]\left[\begin{array}{cc}
a & -b \\
b & a
\end{array}\right] \\
{\left[\begin{array}{cc}
0 & 1 \\
-1 & 0
\end{array}\right] } & =\left[\begin{array}{cc}
a^{2}-b^{2} & -2 a b \\
2 a b & a^{2}-b^{2}
\end{array}\right]
\end{aligned}
$$

and equating left and right matrices we have

$$
\begin{aligned}
a^{2}-b^{2} & =0 \\
2 a b & =-1
\end{aligned}
$$

From this we deduce that

$$
a=b= \pm \frac{\sqrt{2}}{2} i
$$

Therefore, the roots are

$$
\sqrt{-i}= \pm \frac{\sqrt{2}}{2}\left[\begin{array}{cc}
0 & -1 \\
1 & 0
\end{array}\right]\left[\begin{array}{cc}
1 & -1 \\
1 & 1
\end{array}\right]= \pm \frac{\sqrt{2}}{2}\left[\begin{array}{cc}
1 & 1 \\
-1 & 1
\end{array}\right]
$$

Let's test this result by squaring each root to ensure the answer is $i$ :

$$
\left( \pm \frac{\sqrt{2}}{2}\right)^{2}\left[\begin{array}{cc}
1 & 1 \\
-1 & 1
\end{array}\right]\left[\begin{array}{cc}
1 & 1 \\
-1 & 1
\end{array}\right]=\frac{1}{2}\left[\begin{array}{cc}
0 & 2 \\
-2 & 0
\end{array}\right]=-i
$$

\section{总结}
We have shown in this chapter that the set of complex numbers is a field, as they satisfy the requirement for closure, associativity, distributivity, an identity element, and an inverse. We have also shown that there is a one-to-one correspondence between a complex number and an ordered pair, and that a complex number can be represented as a matrix, which permits us to compute all complex number operations as matrix operations or ordered pairs.

If this the first time you have come across complex numbers you probably will have found them strange on the one hand, and amazing on the other. Simply by declaring the existence of $i$ that squares to -1 , opens up a new number system that unifies large areas of mathematics.

\subsection{定义总结}
\subsubsection{定义}
$i \mathbb{R}$ is the set of imaginary numbers: $i b \in i \mathbb{R}, \quad i^{2}=-1$.

\subsubsection*{复数}

Real unit: 1

Imaginary unit: $i$.

\subsubsection*{有序对}

$$
\begin{aligned}
& 1=(1,0) \\
& i=(0,1) .
\end{aligned}
$$

\subsubsection*{矩阵}

$$
\begin{aligned}
1 & =\left[\begin{array}{ll}
1 & 0 \\
0 & 1
\end{array}\right] \\
i & =\left[\begin{array}{cc}
0 & -1 \\
1 & 0
\end{array}\right] .
\end{aligned}
$$

$\mathbb{C}$ is the set of complex numbers: $z=a+i b, \quad a \in \mathbb{R}, \quad i b \in i \mathbb{R}, \quad z \in \mathbb{C}$.

\subsubsection*{复数}
$$
z=a+b i
$$

\subsubsection*{有序对}
$$
z=(a, b)
$$

\subsubsection*{矩阵}

$$
a+b i=\left[\begin{array}{cc}
a & -b \\
b & a
\end{array}\right]
$$

\subsubsection{加减法}
\subsubsection*{复数}

$$
\begin{aligned}
z_{1} & =a_{1}+b_{1} i \\
z_{2} & =a_{2}+b_{2} i \\
z_{1} \pm z_{2} & =a_{1} \pm a_{2}+\left(b_{1} \pm b_{2}\right) i
\end{aligned}
$$

\subsubsection*{有序对}
$$
\begin{aligned}
z_{1} & =\left(a_{1}, b_{1}\right) \\
z_{2} & =\left(a_{2}, b_{2}\right) \\
z_{1} \pm z_{2} & =\left(a_{1} \pm a_{2}, b_{1} \pm b_{2}\right) .
\end{aligned}
$$

\subsubsection*{矩阵}
$$
\begin{aligned}
z_{1} & =\left[\begin{array}{ll}
a_{1} & -b_{1} \\
b_{1} & a_{1}
\end{array}\right] \\
z_{2} & =\left[\begin{array}{cc}
a_{2} & -b_{2} \\
b_{2} & a_{2}
\end{array}\right] \\
z_{1} \pm z_{2} & =\left[\begin{array}{cc}
a_{1} & -b_{1} \\
b_{1} & a_{1}
\end{array}\right] \pm\left[\begin{array}{cc}
a_{2} & -b_{2} \\
b_{2} & a_{2}
\end{array}\right] \\
& =\left[\begin{array}{cc}
a_{1} \pm a_{2} & -\left(b_{1} \pm b_{2}\right) \\
b_{1} \pm b_{2} & a_{1} \pm a_{2}
\end{array}\right] .
\end{aligned}
$$

\subsubsection{乘法}
\subsubsection*{复数}

$$
\begin{aligned}
z_{1} & =a_{1}+b_{1} i \\
z_{2} & =a_{2}+b_{2} i \\
z_{1} z_{2} & =\left(a_{1}+b_{1} i\right)\left(a_{2}+b_{2} i\right) \\
& =\left(a_{1} a_{2}-b_{1} b_{2}\right)+\left(a_{1} b_{2}+b_{1} a_{2}\right) i
\end{aligned}
$$

\subsubsection*{有序对}
$$
\begin{aligned}
z_{1} & =\left(a_{1}, b_{1}\right) \\
z_{2} & =\left(a_{2}, b_{2}\right) \\
z_{1} z_{2} & =\left(a_{1}, b_{1}\right)\left(a_{2}, b_{2}\right) \\
& =\left(a_{1} a_{2}-b_{1} b_{2}, a_{1} b_{2}+b_{1} a_{2}\right) .
\end{aligned}
$$

\subsubsection*{矩阵}
$$
\begin{aligned}
z_{1} & =\left[\begin{array}{ll}
a_{1} & -b_{1} \\
b_{1} & a_{1}
\end{array}\right] \\
z_{2} & =\left[\begin{array}{cc}
a_{2} & -b_{2} \\
b_{2} & a_{2}
\end{array}\right]
\end{aligned}
$$

$$
\begin{aligned}
z_{1} z_{2} & =\left[\begin{array}{cc}
a_{1} & -b_{1} \\
b_{1} & a_{1}
\end{array}\right]\left[\begin{array}{cc}
a_{2} & -b_{2} \\
b_{2} & a_{2}
\end{array}\right] \\
& =\left[\begin{array}{cc}
a_{1} a_{2}-b_{1} b_{2} & -\left(a_{1} b_{2}+b_{1} a_{2}\right) \\
a_{1} b_{2}+b_{1} a_{2} & a_{1} a_{2}-b_{1} b_{2}
\end{array}\right]
\end{aligned}
$$

\subsubsection{平方}
\subsubsection*{复数}
$$
\begin{aligned}
z & =a+b i \\
z^{2} & =(a+b i)(a+b i) \\
& =\left(a^{2}-b^{2}\right)+2 a b i
\end{aligned}
$$

\subsubsection*{有序对}
$$
\begin{aligned}
z & =(a, b) \\
z^{2} & =(a, b)(a, b) \\
& =\left(a^{2}-b^{2}, 2 a b\right)
\end{aligned}
$$

\subsubsection*{矩阵}

$$
\begin{aligned}
z & =\left[\begin{array}{cc}
a & -b \\
b & a
\end{array}\right] \\
z^{2} & =\left[\begin{array}{cc}
a & -b \\
b & a
\end{array}\right]\left[\begin{array}{cc}
a & -b \\
b & a
\end{array}\right] \\
& =\left[\begin{array}{cc}
a^{2}-b^{2} & -2 a b \\
2 a b & a^{2}-b^{2}
\end{array}\right] .
\end{aligned}
$$

\subsubsection{范数}

\subsubsection*{复数}

$$
\begin{aligned}
z & =a+b i \\
|z| & =\sqrt{a^{2}+b^{2}}
\end{aligned}
$$

\subsubsection*{有序对}
$$
\begin{aligned}
z & =(a, b) \\
|z| & =\sqrt{a^{2}+b^{2}}
\end{aligned}
$$

\subsubsection*{矩阵}
$$
\begin{aligned}
z & =\left[\begin{array}{cc}
a & -b \\
b & a
\end{array}\right] \\
|z|^{2} & =\left|\begin{array}{cc}
a & -b \\
b & a
\end{array}\right| .
\end{aligned}
$$

\subsubsection{复共轭}
\subsubsection*{复数}

$$
\begin{aligned}
z & =a+b i \\
z^{*} & =a-b i
\end{aligned}
$$

\subsubsection*{有序对}
$$
\begin{aligned}
z & =(a, b) \\
z^{*} & =(a,-b) .
\end{aligned}
$$

\subsubsection*{矩阵}

$$
\begin{aligned}
z & =\left[\begin{array}{cc}
a & -b \\
b & a
\end{array}\right] \\
z^{*} & =\left[\begin{array}{cc}
a & b \\
-b & a
\end{array}\right] .
\end{aligned}
$$

\subsubsection{逆}
\subsubsection*{复数}

$$
\begin{aligned}
z & =a+b i \\
z^{-1} & =\frac{z^{*}}{|z|^{2}} \\
& =\left(\frac{a}{a^{2}+b^{2}}\right)-\left(\frac{b}{a^{2}+b^{2}}\right) i
\end{aligned}
$$

\subsubsection*{有序对}
$$
\begin{aligned}
z & =(a, b) \\
z^{*} & =(a,-b) \\
|z|^{2} & =a^{2}+b^{2}
\end{aligned}
$$

$$
\begin{aligned}
\frac{1}{z} & =z^{-1}=\frac{z^{*}}{|z|^{2}} \\
& =\left(\frac{a}{a^{2}+b^{2}}, \frac{-b}{a^{2}+b^{2}}\right)
\end{aligned}
$$

\subsubsection*{矩阵}
$$
\begin{aligned}
z & =\left[\begin{array}{cc}
a & -b \\
b & a
\end{array}\right] \\
z^{*} & =\left[\begin{array}{cc}
a & b \\
-b & a
\end{array}\right] \\
|z|^{2} & =a^{2}+b^{2} \\
\frac{1}{z} & =z^{-1}=\frac{z^{*}}{|z|^{2}} \\
& =\frac{1}{a^{2}+b^{2}}\left[\begin{array}{cc}
a & b \\
-b & a
\end{array}\right] .
\end{aligned}
$$

\subsubsection{商}
\subsubsection*{复数}

$$
\begin{aligned}
z_{1} & =a_{1}+b_{1} i \\
z_{2} & =a_{2}+b_{2} i \\
\frac{z_{1}}{z_{2}} & =\frac{a_{1}+b_{1} i}{a_{2}+b_{2} i} \\
& =\left(\frac{a_{1} a_{2}+b_{1} b_{2}}{a_{2}^{2}+b_{2}^{2}}\right)+\left(\frac{b_{1} a_{2}-a_{1} b_{2}}{a_{2}^{2}+b_{2}^{2}}\right) i .
\end{aligned}
$$

\subsubsection*{有序对}
$$
\begin{aligned}
z_{1} & =\left(a_{1}, b_{1}\right) \\
z_{2} & =\left(a_{2}, b_{2}\right) \\
\frac{z_{1}}{z_{2}} & =\frac{\left(a_{1}, b_{1}\right)}{\left(a_{2}, b_{2}\right)} \\
& =\left(\frac{a_{1} a_{2}+b_{1} b_{2}}{a_{2}^{2}+b_{2}^{2}}, \frac{b_{1} a_{2}-a_{1} b_{2}}{a_{2}^{2}+b_{2}^{2}}\right) .
\end{aligned}
$$

\subsubsection*{矩阵}
$$
\begin{aligned}
z_{1} & =\left[\begin{array}{cc}
a_{1} & -b_{1} \\
b_{1} & a_{1}
\end{array}\right] \\
z_{2} & =\left[\begin{array}{cc}
a_{2} & -b_{2} \\
b_{2} & a_{2}
\end{array}\right] \\
\frac{z_{1}}{z_{2}} & =z_{1} z_{2}^{-1} \\
& =\frac{1}{a_{2}^{2}+b_{2}^{2}}\left[\begin{array}{cc}
a_{1} & -b_{1} \\
b_{1} & a_{1}
\end{array}\right]\left[\begin{array}{cc}
a_{2} & b_{2} \\
-b_{2} & a_{2}
\end{array}\right] \\
& =\frac{1}{a_{2}^{2}+b_{2}^{2}}\left[\begin{array}{ll}
a_{1} a_{2}+b_{1} b_{2} & -\left(b_{1} a_{2}-a_{1} b_{2}\right) \\
b_{1} a_{2}-a_{1} b_{2} & a_{1} a_{2}+b_{1} b_{2}
\end{array}\right] .
\end{aligned}
$$

\subsubsection{Square root of $\pm i$}
\subsubsection{Complex Number}
$$
\begin{aligned}
\sqrt{i} & = \pm \frac{\sqrt{2}}{2}(1+i) \\
\sqrt{-i} & = \pm \frac{\sqrt{2}}{2}(1-i)
\end{aligned}
$$

\subsubsection{Ordered Pair}
$$
\begin{aligned}
\sqrt{i} & = \pm \frac{\sqrt{2}}{2}(1,1) \\
\sqrt{-i} & = \pm \frac{\sqrt{2}}{2}(1,-1) .
\end{aligned}
$$

\subsubsection{Matrix}
$$
\begin{aligned}
\sqrt{i} & = \pm \frac{\sqrt{2}}{2}\left[\begin{array}{cc}
1 & -1 \\
1 & 1
\end{array}\right] \\
\sqrt{-i} & = \pm \frac{\sqrt{2}}{2}\left[\begin{array}{cc}
1 & 1 \\
-1 & 1
\end{array}\right] .
\end{aligned}
$$

\section{样例}
Here are some further worked examples that employ the ideas described above. In some cases a test is included to confirm the result.

\subsection{复数加减法}
Add and subtract $z_{1}$ and $z_{2}$.

Complex Number

$$
\begin{aligned}
z_{1} & =12+6 i \\
z_{2} & =10-4 i \\
z_{1}+z_{2} & =22+2 i \\
z_{1}-z_{2} & =2+10 i
\end{aligned}
$$

\section{Ordered Pair}
$$
\begin{aligned}
z_{1} & =(12,6) \\
z_{2} & =(10,-4) \\
z_{1}+z_{2} & =(12,6)+(10,-4)=(22,2) \\
z_{1}-z_{2} & =(12,6)-(10,-4)=(2,10) .
\end{aligned}
$$

Matrix

$$
\begin{aligned}
z_{1} & =\left[\begin{array}{cc}
12 & -6 \\
6 & 12
\end{array}\right] \\
z_{2}= & {\left[\begin{array}{cc}
10 & 4 \\
-4 & 10
\end{array}\right] } \\
z_{1}+z_{2} & =\left[\begin{array}{cc}
12 & -6 \\
6 & 12
\end{array}\right]+\left[\begin{array}{cc}
10 & 4 \\
-4 & 10
\end{array}\right]=\left[\begin{array}{cc}
22 & -2 \\
2 & 22
\end{array}\right] \\
z_{1}-z_{2} & =\left[\begin{array}{cc}
12 & -6 \\
6 & 12
\end{array}\right]-\left[\begin{array}{cc}
10 & 4 \\
-4 & 10
\end{array}\right]=\left[\begin{array}{cc}
2 & -10 \\
10 & 2
\end{array}\right] .
\end{aligned}
$$

\subsubsection{Product of Complex Numbers}
Compute the product $z_{1} z_{2}$.

\section{Complex Number}
$$
\begin{aligned}
z_{1} & =12+6 i \\
z_{2} & =10-4 i \\
z_{1} z_{2} & =(12+6 i)(10-4 i) \\
& =144+12 i
\end{aligned}
$$

\section{Ordered Pair}
$$
\begin{aligned}
z_{1} & =(12,6) \\
z_{2} & =(10,-4) \\
z_{1} z_{2} & =(12,6)(10,-4) \\
& =(120+24,-48+60) \\
& =(144,12) .
\end{aligned}
$$

\section{Matrix}
$$
\begin{aligned}
z_{1} & =\left[\begin{array}{cc}
12 & -6 \\
6 & 12
\end{array}\right] \\
z_{2} & =\left[\begin{array}{cc}
10 & 4 \\
-4 & 10
\end{array}\right] \\
z_{1} z_{2} & =\left[\begin{array}{cc}
12 & -6 \\
6 & 12
\end{array}\right]\left[\begin{array}{cc}
10 & 4 \\
-4 & 10
\end{array}\right]=\left[\begin{array}{cc}
144 & -12 \\
12 & 144
\end{array}\right] .
\end{aligned}
$$

\subsubsection{Multiplying a Complex Number by $i$}
Multiply $z_{1}$ by $i$.

Complex Number

$$
\begin{aligned}
z_{1} & =12+6 i \\
z_{1} i & =(12+6 i) i \\
& =-6+12 i
\end{aligned}
$$

\section{Ordered Pair}
$$
\begin{aligned}
z_{1} & =(12,6) \\
i & =(0,1) \\
z_{1} i & =(12,6)(0,1) \\
& =(-6,12) .
\end{aligned}
$$

\section{Matrix}
$$
\begin{aligned}
z_{1} & =\left[\begin{array}{cc}
12 & -6 \\
6 & 12
\end{array}\right] \\
i= & {\left[\begin{array}{cc}
0 & -1 \\
1 & 0
\end{array}\right] } \\
z_{1} z_{2} & =\left[\begin{array}{cc}
12 & -6 \\
6 & 12
\end{array}\right]\left[\begin{array}{cc}
0 & -1 \\
1 & 0
\end{array}\right]=\left[\begin{array}{cc}
-6 & -12 \\
12 & -6
\end{array}\right] .
\end{aligned}
$$

\subsubsection{The Norm of a Complex Number}
Compute the norm of $z_{1}$.

Complex Number

$$
\begin{aligned}
z_{1} & =12+6 i \\
\left|z_{1}\right| & =\sqrt{12^{2}+6^{2}} \approx 13.416
\end{aligned}
$$

Ordered Pair

$$
\begin{aligned}
z_{1} & =(12,6) \\
\left|z_{1}\right| & =\sqrt{12^{2}+6^{2}} \approx 13.416 .
\end{aligned}
$$

Matrix

$$
\begin{aligned}
z_{1} & =\left[\begin{array}{cc}
12 & -6 \\
6 & 12
\end{array}\right] \\
\left|z_{1}\right| & =\left|\begin{array}{cc}
12 & -6 \\
6 & 12
\end{array}\right|=\sqrt{12^{2}+6^{2}} \approx 13.416
\end{aligned}
$$

\subsubsection{The Complex Conjugate of a Complex Number}
Compute the complex conjugate of the following.

\section{Complex Number}
$$
\begin{aligned}
(2+3 i)^{*} & =2-3 i \\
1^{*} & =1 \\
i^{*} & =-i
\end{aligned}
$$

\section{Ordered Pair}
$$
\begin{aligned}
& (2,3)^{*}=(2,-3) \\
& (1,0)^{*}=\left(\begin{array}{ll}
1, & 0
\end{array}\right) \\
& (0,1)^{*}=\left(\begin{array}{ll}
0,-1
\end{array}\right)
\end{aligned}
$$

\section{Matrix}
$$
\begin{aligned}
z & =\left[\begin{array}{cc}
2 & -3 \\
3 & 2
\end{array}\right] \\
z^{*} & =\left[\begin{array}{cc}
2 & 3 \\
-3 & 2
\end{array}\right] \\
1 & =\left[\begin{array}{ll}
1 & 0 \\
0 & 1
\end{array}\right] \\
1^{*} & =\left[\begin{array}{ll}
1 & 0 \\
0 & 1
\end{array}\right] \\
i & =\left[\begin{array}{cc}
0 & -1 \\
1 & 0
\end{array}\right] \\
i^{*} & =\left[\begin{array}{cc}
0 & 1 \\
-1 & 0
\end{array}\right] .
\end{aligned}
$$

\subsubsection{The Quotient of Two Complex Numbers}
Compute the quotient $(2+3 i) /(3+4 i)$.

\section{Complex Number}
$$
\begin{aligned}
\frac{2+3 i}{3+4 i} & =\frac{(2+3 i)}{(3+4 i)} \frac{(3-4 i)}{(3-4 i)} \\
& =\frac{6-8 i+9 i+12}{25} \\
& =\frac{18}{25}+\frac{1}{25} i
\end{aligned}
$$

Test

$$
\begin{aligned}
(3+4 i)\left(\frac{18}{25}+\frac{1}{25} i\right) & =\frac{54}{25}+\frac{3}{25} i+\frac{72}{25} i-\frac{4}{25} \\
& =2+3 i .
\end{aligned}
$$

\section{Ordered Pair}
$$
\begin{aligned}
\frac{(2,3)}{(3,4)} & =\frac{(2,3)}{(3,4)} \frac{(3,-4)}{(3,-4)} \\
& =\frac{(6+12,1)}{(9+16,0)} \\
& =\left(\frac{18}{25}, \frac{1}{25}\right) .
\end{aligned}
$$

Matrix

$$
\begin{aligned}
z_{1} & =\left[\begin{array}{cc}
2 & -3 \\
3 & 2
\end{array}\right] \\
z_{2} & =\left[\begin{array}{cc}
3 & -4 \\
4 & 3
\end{array}\right] \\
\frac{z_{1}}{z_{2}} & =z_{1} z_{2}^{-1} \\
& =\frac{1}{25}\left[\begin{array}{cc}
2 & -3 \\
3 & 2
\end{array}\right]\left[\begin{array}{cc}
3 & 4 \\
-4 & 3
\end{array}\right] \\
& =\frac{1}{25}\left[\begin{array}{cc}
18 & -1 \\
1 & 18
\end{array}\right] .
\end{aligned}
$$

\subsubsection{Divide a Complex Number by $i$}
Divide $2+3 i$ by $i$.

\section{Complex Number}
$$
\begin{aligned}
\frac{2+3 i}{0+i} & =\frac{(2+3 i)}{(0+i)} \frac{(0-i)}{(0-i)} \\
& =\frac{-2 i+3}{1} \\
& =3-2 i
\end{aligned}
$$

Test

$$
i(3-2 i)=2+3 i
$$

\section{Ordered Pair}
$$
\begin{aligned}
\frac{(2,3)}{(0,1)} & =\frac{(2,3)}{(0,1)} \frac{(0,-1)}{(0,-1)} \\
& =\frac{(3,-2)}{(1,0)} \\
& =(3,-2) .
\end{aligned}
$$

Matrix

$$
\begin{aligned}
z & =\left[\begin{array}{cc}
2 & -3 \\
3 & 2
\end{array}\right] \\
i & =\left[\begin{array}{cc}
0 & -1 \\
1 & 0
\end{array}\right] \\
i^{-1} & =\left[\begin{array}{cc}
0 & 1 \\
-1 & 0
\end{array}\right] \\
z i^{-1} & =\left[\begin{array}{ll}
2 & -3 \\
3 & 2
\end{array}\right]\left[\begin{array}{cc}
0 & 1 \\
-1 & 0
\end{array}\right]=\left[\begin{array}{cc}
3 & 2 \\
-2 & 3
\end{array}\right] .
\end{aligned}
$$

\subsubsection{Divide a Complex Number by $-i$}
Divide $2+3 i$ by $-i$.

Complex Number

$$
\begin{aligned}
\frac{2+3 i}{0-i} & =\frac{(2+3 i)}{(0-i)} \frac{(0+i)}{(0+i)} \\
& =\frac{2 i-3}{1} \\
& =-3+2 i
\end{aligned}
$$

Test

$$
-i(-3+2 i)=2+3 i
$$

\section{Ordered Pair}
$$
\begin{aligned}
\frac{(2,3)}{(0,-1)} & =\frac{(2,3)}{(0,-1)} \frac{(0,1)}{(0,1)} \\
& =\frac{(-3,2)}{1} \\
& =(-3,2)
\end{aligned}
$$

\section{Matrix}
$$
\begin{aligned}
z & =\left[\begin{array}{cc}
2 & -3 \\
3 & 2
\end{array}\right] \\
-i & =\left[\begin{array}{cc}
0 & 1 \\
-1 & 0
\end{array}\right] \\
-i^{-1} & =\left[\begin{array}{cc}
0 & -1 \\
1 & 0
\end{array}\right] \\
z\left(-i^{-1}\right) & =\left[\begin{array}{cc}
2 & -3 \\
3 & 2
\end{array}\right]\left[\begin{array}{cc}
0 & -1 \\
1 & 0
\end{array}\right]=\left[\begin{array}{cc}
-3 & -2 \\
2 & -3
\end{array}\right] .
\end{aligned}
$$

\subsubsection{The Inverse of a Complex Number}
Compute the inverse of $2+3 i$.

\section{Complex Number}
$$
\begin{aligned}
\frac{1}{2+3 i} & =\frac{1}{(2+3 i)} \frac{(2-3 i)}{(2-3 i)} \\
& =\frac{2-3 i}{13} \\
& =\frac{2}{13}-\frac{3}{13} i .
\end{aligned}
$$

\section{Ordered Pair}
$$
\begin{aligned}
\frac{1}{(2,3)} & =\frac{1}{(2,3)} \frac{(2,-3)}{(2,-3)} \\
& =\frac{(2,-3)}{13} \\
& =\left(\frac{2}{13},-\frac{3}{13}\right) .
\end{aligned}
$$

Matrix

$$
\begin{aligned}
z & =\left[\begin{array}{lc}
2 & -3 \\
3 & 2
\end{array}\right] \\
z^{-1} & =\frac{1}{13}\left[\begin{array}{cc}
2 & 3 \\
-3 & 2
\end{array}\right] .
\end{aligned}
$$

\subsubsection{The Inverse of $i$}
Compute the inverse of $i$.

\section{Complex Number}
$$
\begin{aligned}
\frac{1}{0+i} & =\frac{1}{(0+i)} \frac{(0-i)}{(0-i)} \\
& =\frac{-i}{1}=-i
\end{aligned}
$$

\section{Ordered Pair}
$$
\begin{aligned}
\frac{1}{(0,1)} & =\frac{1}{(0,1)} \frac{(0,-1)}{(0,-1)} \\
& =\frac{(0,-1)}{(1,0)}=(0,-1)=-i .
\end{aligned}
$$

Matrix

$$
\begin{aligned}
i & =\left[\begin{array}{cc}
0 & -1 \\
1 & 0
\end{array}\right] \\
i^{-1} & =\left[\begin{array}{cc}
0 & 1 \\
-1 & 0
\end{array}\right]=-i .
\end{aligned}
$$

\subsubsection{The Inverse of $-i$}
Compute the inverse of $-i$.

\section{Complex Number}
$$
\begin{aligned}
\frac{1}{0-i} & =\frac{1}{(0-i)} \frac{(0+i)}{(0+i)} \\
& =\frac{i}{1}=i
\end{aligned}
$$

\section{Ordered Pair}
$$
\begin{aligned}
\frac{1}{(0,-1)} & =\frac{1}{(0,-1)} \frac{(0,1)}{(0,1)} \\
& =\frac{(0,1)}{(1,0)}=(0,1)=i .
\end{aligned}
$$

\section{Matrix}
$$
\begin{aligned}
-i & =\left[\begin{array}{cc}
0 & 1 \\
-1 & 0
\end{array}\right] \\
-i^{-1} & =\left[\begin{array}{cc}
0 & -1 \\
1 & 0
\end{array}\right]=i .
\end{aligned}
$$

\section{References}
\begin{enumerate}
  \item Vince, J.: Imaginary Mathematics for Computer Science. Springer, Berlin (2018). ISBN 978-3319-94636-8

  \item Descartes, R.: La Géométrie (1637) (There is an English translation by Michael Mahoney) Dover, New York (1979)

\end{enumerate}
