\pdfbookmark{第二版译序}{2ndxu}
\section*{第二版译序} 

\pdfbookmark{第一版译序}{1stxu}
\section*{第一版译序} 
译者是一名图像算法工程师,在学习 SLAM 的时候第一次接触到四元数,草草了解一点点,然后在三维重建的项目中,遇到点云显示的问题,也需要用四元数来处理旋转进行解决,恰好在查找资料的过程中发现了本书,大家也都推荐此书是一本讲解四元数很透彻的书,但是令我感到惊奇的是,如此一本好书居然目前为止尚未有中文版出版,故此,我一边学习此书一边翻译下来。

读完此书后,我的感受是,这本书讲解很透彻,循循善诱,基本上不用依赖什么大学数学知识就可以理解,甚至,讲解细致程度可以让小学生也能懂。没有多少过于复杂的定理堆彻,基本会慢慢引入概念,然后一步步引导推导,推导完成后再进行总结,最后再用例题来验证掌握程度,加深印象,巩固掌握。

当然,译者也是第一次系统学习四元数,可能对其中有些术语表达不准确;也有一些句子感觉上翻译得不是很对,很拗口,难以理解;也有一些词语为了统一,可能在局部的地方并不是那么准确,比如“ inverse ”,统一翻译为逆,而不是倒数。如果有发现翻译或原文有什么问题,有更好的翻译,或者公式上面有上面疏漏,欢迎随时发邮件指正勘误。

希望各位学习四元数的旅程能如译者第一次学习本书时一样轻松愉快!

\hfill 郭飞  Bluffey@163.com

\hfill 2023年2月17日 于 成都 

\newpage
\pdfbookmark{第二版序}{2ndxu}
\section*{第二版序}
当我在学习成为一名电气工程师的时候,我遇到了复数,它们被用来用$j$算子表示非相电压和电流。我相信用的是字母$j$,而不是$i$,因为后者代表电流。所以从我学习的一开始,我脑海中就有一个清晰的画面,想象中的单位是一个旋转算子,它可以在时间上推进或延迟电量。

当事情的发展决定了我要从事计算机编程而不是电子工程的职业时,我并不需要复数,直到 Mandlebrot 对分形的研究出现。但这只是一个暂时的阶段,我从来没有需要在我的计算机图形软件中使用复数。然而,在1986年,当我加入飞行模拟行业时,我看到了一份关于四元数的内部报告,它被用来控制模拟飞机的旋转方向。

我还记得当时我被四元数弄糊涂了,因为它们包含了太多的虚数。然而,经过大量的研究,我开始了解它们是什么,但不知道它们是如何工作的。与此同时,我开始对数学的哲学方面感兴趣,并试图通过伯特兰·罗素(Bertrand Russell)的著作来理解数学的“真正意义”。因此,像$i$这样的概念是一个智力上的挑战。

现在,我已经习惯了虚数$i$只不过是一个符号,并且在代数的上下文中允许定义$i^{2}=-1$。我认为,试图发现其存在的深层意义是徒劳的。然而,它是数学中一个令人惊奇的对象,我经常想是否会有类似的对象等待被发明。

当我开始写关于计算机图形学的数学书籍时,我学习了复分析,以便有信心地写
复数。就在那时,我发现了向量和四元数发明背后的历史事件,主要是通过 Michael Crowe的优秀著作:《向量分析的历史\footnote{ A History of Vector Analysis}》。这本书让我认识到了解数学发明的方式和原因的重要性。。然后我读到了 Simon Altmann 的书:《旋转、四元数和双群\footnote{Rotations, Quaternions, and Double Groups} 》,该书提供了有关十九世纪四元数衰落的更多信息。Altmann 非常热情地为奥林德•罗德里格斯(Olinde Rodrigues)的数学工作争取认可,他发表了一个公式,与汉密尔顿的四元数产生的公式非常相似。罗德里格斯的这篇论文的重要之处在于,它比汉密尔顿在1843年发明四元数早了三年。然而,罗德里格斯并没有发明四元数代数——这个奖项应该属于汉密尔顿——但他确实理解了半角在三角函数中的重要性,三角函数用于使点绕任意轴旋转。

任何使用过欧拉变换的人都会意识到它的缺点,尤其是它的致命弱点:万向锁。因此,任何可以围绕任意轴旋转点的设备都是程序员工具箱中受欢迎的补充。在平面和空间中有许多旋转点和帧的技术,我在我的书中有一些详细的介绍:《计算机图形的旋转变换\footnote{Rotation Transforms for Computer Graphics}》。那本书还涵盖了 Euler-Rodrigues 参数化和四元数,但只有在提交手稿出版后,我才决定写这本书,专门讨论四元数,它们是如何和为什么被发明的,以及它们在计算机图形学中的应用。

在研究这本书的同时,阅读 William Rowan Hamilton 和他的朋友p·g·Tait的一些早期书籍和论文是非常有教育意义的。我现在明白了要完全理解四元数的重要性以及如何利用它们是多么困难。当时,没有围绕任意轴旋转点的主要需求;然而,需要一个数学系统来处理向量。最后,四元数不再流行,慢慢地淡出了人们的视线。尽管如此,对 Hamilton 来说,表示矢量并对其进行算术运算的能力是一项重大成就,尽管乔赛亚·吉布斯(Josiah Gibbs)有远见地创建了一个简单可行的代数框架。

我希望这第二版能如预期的那样改进原书。我已经复习了所有的文档和代数符号,并重新设计了所有的图形,包括颜色。我还对每章的参考书目进行了本地化调整,并扩大了索引。具体来说,我引入了一章来描述三元组如何导致四元数的发明。最后,我扩展了对Olinde Rodrigues的参考和关于插值的部分。

四元数有不同的表示方式,但我最喜欢的是有序对,这是我在 Simon Altmann 的书中发现的。我绝不认为自己是四元数方面的权威。我只是想交流一下我是如何理解他们的,希望对你有用。

这本书现在分为九章。第一章和最后一章介绍和总结了这本书,其余七章涵盖以下主题。第2章是关于数集和代数,并回顾符号和语言相关的书的其余部分。书中有关于数集、公理、有序对、群、环和域的章节。这为读者准备了非交换四元数乘积,以及为什么四元数被描述为除法环。第3章回顾了复数,并展示了如何将它们表示为有序对和矩阵。第4章继续这个主题,介绍复平面并展示复数的旋转特征。它还为读者准备了19世纪初提出的问题:是否存在一个复数的三维等价物?第5章追溯了四元数在三元组的中间阶段的出现,并展示了汉密尔顿如何引入第三个假想的$k$,以及如何将其与$i$和$j$整合。

第6章通过描述 Hamilton 的发明来回答这个问题:四元数和相关代数。我加入了一些历史信息,以便读者理解汉密尔顿作品的重要性。虽然有序对是主要的表示法,但我也包括了矩阵表示法。

为了让读者为四元数的旋转质量做好准备,第7章回顾了3D旋转变换,特别是欧拉角和万向节锁。我还开发了一个矩阵,用于使用向量和矩阵变换围绕任意轴旋转一个点。

第8章是本书的重点,描述了四元数如何围绕任意轴旋转向量。本章以一些历史信息开始,并解释不同的四元数乘积如何旋转点。虽然四元数很容易使用它们的复数形式或有序对符号来实现,但它们也有矩阵形式,这是从第一原理发展而来的。本章继续介绍关于偏移轴旋转,旋转参考帧,插值四元数,以及在四元数和旋转矩阵之间转换的部分。每章包含许多实际的例子,以显示如何评估方程,并在相关的地方,每章末尾会给出进一步的例子。

准备第二版是一个非常愉快的经历,我相信你也会喜欢阅读它,并从它的页面中发现一些新的东西。

我要感谢米德尔塞克斯大学荣誉读者Tony Crilly博士,他阅读了原稿,并纠正和澄清了我的注释和解释。我完全相信他的数学知识,我很感激他的建议和专业知识。然而,我仍然要为我可能犯的代数错误负全部责任。

我还要感谢Patrick Riley教授,他阅读了一些手稿的早期草稿,并提出了一些关于四元数的有趣的技术问题。这些问题使我意识到我对四元数的一些描述需要进一步澄清,希望这些描述已经被纠正了。

我不确定这是否是我的最后一本书。如果是的话,我要感谢英国施普林格计算机科学编辑Helen Desmond,感谢她在过去几年的专业支持。如果这不是我的最后一本书,那么我期待着与她在另一个项目上再次合作。



Breinton, UK \hfill John Vince


\newpage
\pdfbookmark{第一版序}{diyixu}
\section*{第一版序}

%\bookmark[page=6,level=0]{序}
% \bookmark[dest=\HyperLocalCurrentHref, level=0]{序}

50多年前,当我正在学习成为一名电气工程师时,我遇到了复数,它们被用来用$j$操作符表示异相电压和电流。我相信使用的是字母$j$,而不是$ i $,因为后者代表电流。因此,从我学习的一开始,我就在脑海中清晰地描绘了一个假想的单位,它是一个旋转算子,可以在时间上推进或延缓电量。


当事情决定我要从事计算机编程而不是电气工程的职业时,我不需要复数,直到Mandlebrot关于分形的工作出现。但那只是暂时的阶段,我从来都不需要在我的任何电脑图形软件中使用复数。然而在1986年,当我加入飞行模拟行业时,我看到了一份关于四元数的内部报告,它被用于控制模拟飞机的旋转方向。

我仍然记得我对四元数的困惑,仅仅是因为它们包含了很多虚数。然而,经过大量研究,我开始了解它们是什么,但不知道它们是如何工作的。与此同时,我对数学的哲学方面产生了兴趣,并试图通过伯特兰·罗素的著作来理解数学的“真正意义”。因此,像$i$这样的概念是一个智力挑战。

我现在对虚数$ i $不过是一个符号的想法感到满意,并且在代数的上下文中允许定义$ i ^{2}=-1$。我相信,试图发现它存在的更深层次的意义是徒劳的。尽管如此,它在数学中是一个令人惊叹的对象,我经常想是否会有类似的对象等待被发明出来。

当我开始写关于计算机图形学的数学书籍时,我学习了复分析,以便有信心地写复数。就在那时,我发现了向量和四元数发明背后的历史事件,主要是通过Michael Crowe的优秀著作《向量分析的历史\footnote{ A History of Vector Analysis}》。这本书让我认识到理解数学发明如何以及为什么发生的重要性。


最近,我读到了Simon Altmann的书《旋转、四元数和双群\footnote{Rotations, Quaternions, and Double Groups} 》,这本书提供了关于十九世纪四元数消亡\footnote{demise,感觉不应该译为消亡}的进一步信息。Altmann非常热衷于确保Olinde Rodrigues的数学工作得到认可,后者发表了一个与汉密尔顿四元数生成的公式非常相似的公式。罗德里格斯发表论文的一个重要方面是,它比汉密尔顿在1843年发明四元数早了三年。然而,罗德里格斯并没有发明四元数代数——这个奖必须颁给汉密尔顿——但他确实理解三角函数中半角的重要性,三角函数用于围绕任意轴旋转点。

任何使用过欧拉变换的人都会意识到它的缺点,尤其是它的阿喀琉斯之踵:万向节锁。因此,任何可以围绕任意轴旋转点的设备都是程序员工具包中受欢迎的补充。在平面和空间中有许多旋转点和帧的技术,我在我的书《计算机图形的旋转变换\footnote{Rotation Transforms for Computer Graphics}》中详细介绍了这些技术。那本书还涵盖了欧拉-罗德里格斯参数化和四元数,但只是在提交了手稿准备出版之后,我才决定写这本专门关于四元数的书,以及它们是如何和为什么被发明的,以及它们在计算机图形学中的应用。

在研究这本书的同时,阅读William Rowan Hamilton和他的朋友P.G. Tait的一些早期书籍和论文非常有启发性。我现在明白了完全理解四元数的意义是多么困难,以及如何利用它们。当时,没有围绕任意轴旋转点的主要需求;然而,需要一个数学系统来处理矢量。最后,四元数不再是当月的宠儿,慢慢地淡出了人们的视线。尽管如此,能够表示向量并对它们进行算术操作是汉密尔顿的一个重大成就,尽管乔赛亚·吉布斯(Josiah Gibbs)有远见地创建了一个简单可行的代数框架。

在这本书中,我试图描述一些围绕四元数发明的历史,以及对四元数代数的描述。我绝不认为自己是四元数方面的权威。我只是想谈谈我是如何理解它们的,希望对你们有用。四元数有不同的表示方式,但我最喜欢的是有序对,这是我在Simon Altmann的书中发现的。

这本书分为八章。第一章和最后一章是对本书的介绍和总结,共六章,内容包括以下主题。关于数集和代数的第二章回顾了与本书其余部分相关的符号和语言。有关于数集、公理、有序对、群、环和域的章节。这为读者准备了非交换四元数积,以及为什么四元数被描述为除法环。

第三章回顾了复数,并展示了复数如何表示为有序对和矩阵。第四章继续这一主题,介绍了复平面,并展示了复数的旋转特征。它还为读者准备了一个在19世纪早期提出的问题:是否有一个相当于复数的$3 D$ ?

第五章通过描述汉密尔顿的发明来回答这个问题:四元数及其相关代数。我在书中加入了一些历史信息,以便读者了解Hamilton著作的重要性。虽然有序对是表示法的主要形式,但我也包括了矩阵表示法。

为了让读者了解四元数的旋转特性,第6章回顾了3D旋转变换,特别是欧拉角和万向节锁。我还开发了一个矩阵,用于使用向量和矩阵变换围绕任意轴旋转一个点。

第7章是本书的重点,描述了四元数如何围绕任意轴旋转向量。本章以一些历史信息开始,并解释了不同的四元数乘积如何旋转点。虽然四元数很容易使用它们的复杂形式或有序对符号来实现,但它们也有一个矩阵形式,这是从第一性原理发展而来的。本章继续讨论特征值、特征向量、围绕偏移轴旋转、旋转参照系、插值四元数以及四元数和旋转矩阵之间的转换。

每一章都有很多实际的例子来说明如何计算方程,每章末尾会给出进一步的例子。

写这本书是一段非常愉快的经历,我相信你也会喜欢读它,并从书中发现一些新的东西。

我要感谢米德尔塞克斯大学荣誉读者Tony Crilly博士,他阅读了一份手稿草稿,并纠正和澄清了我的注释和解释。托尼在我的书《计算机图形的旋转变换》中执行了相同的任务。我绝对信任他的数学知识,我感谢他的建议和专业知识。然而,我仍然为我可能犯的任何代数错误承担全部责任。

我还要感谢Patrick Riley教授,他阅读了手稿的一些早期草稿,并提出了一些关于四元数的有趣技术问题。这样的问题让我意识到我对四元数的一些描述需要进一步澄清,希望这些描述已经得到纠正。

我现在已经在我的三本书中使用了\LaTeX$ 2\varepsilon$ ,并且对它的符号有了信心。尽管如此,我仍然需要致电施普林格的技术支持团队,感谢他们的帮助。

我不确定这是否是我的最后一本书。如果是的话,我要感谢贝弗利·福特,计算机科学的编辑主任,和海伦·德斯蒙德,施普林格英国计算机科学的副主编,他们在过去几年的专业支持。如果这不是我的最后一本书,那么我期待着在另一个项目上与他们再次合作。


Ringwood, UK  \hfill                                               John Vince
