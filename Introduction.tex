
\chap{介绍}
\section{旋转变换}
在计算机图形学中,我们使用变换来修改对象或虚拟摄像机的位置和方向。这种变换通常包括:缩放、平移和旋转。前两个变换很简单,但是旋转确实会引起问题。这是因为我们通常从围绕$x-, y$ -和$z$-轴的单独旋转中构造一个旋转变换。尽管这样的转变行之有效,但还远远不够完美。真正需要的是一种直观、简单和准确的技术。

多年来,旋转变换包括方向余弦,欧拉角,欧拉-罗德里格斯参数化,四元数和多向量。最后两种技术是最新的,并且与历史相关。然而,这本书的主题是四元数,以及它们如何在计算机图形学中使用。


\section{目标读者}
这本书的目标读者学习或工作在计算机图形学和需要一个四元数的概述。他们可能就是我在互联网论坛上遇到过的询问四元数、它们如何工作以及如何编码的人。希望这本书能回答大部分问题。


\section{本书的宗旨和目标}
这本书的主要目的是向读者介绍四元数的主题,以及它们如何用于围绕任意轴旋转点。第二个目标是让读者意识到所有数学发现背后的人的层面。就我个人而言,我认为我们绝不能忽视这样一个事实:数学家也是人。尽管他们可能被赋予了非凡的数学技能,但他们恋爱、结婚、成家、死亡,留下了一座令人惊叹的知识大厦,我们都从中受益。

鼓励那些对数学的人类维度感兴趣的读者阅读四元数的发明,并发现爱、兴趣、坚韧、灵感和奉献是如何导致一项重大数学发现的。迈克尔·克罗的《矢量分析的历史》\cite{bib1-1}是一本必不可少的书。这提供了导致四元数发明的事件的彻底分析,以及向量分析的出现。第二本书是Simon Altmann的《旋转、四元数和双群\cite{bib1-2}》,除了提供了四元数代数的现代分析外,还介绍了Olinde Rodrigues,他比通常被公认为四元数之父的William Rowan Hamilton早三年发明了一些与四元数相关的数学。

Simon Altmann对四元数代数的分析对我自己对四元数的看法产生了深远的影响,我试图在接下来的章节中传达这一点。特别地,我采用了使用有序对来表示四元数的思想。

本书的首要目标是使读者能够设计和编码四元数算法。读完这本书后,这应该是一个简单的练习,尽管我们处理的是一个四维数学对象。

\section{数学技术}
一旦读者理解了四元数,就会认为它们很简单。然而,如果这是你第一次遇到他们,他们可能会显得很奇怪。但像所有事物一样,熟悉带来理解和信心。

为了描述四元数,我需要用到一点三角学,一些向量理论和矩阵代数。由于四元数被描述为“超复数”,因此有一章是关于复数的。


\section{本书中的假设}
很多时候,从事计算机图形工作的人——比如我自己——没有机会学习技术文献中经常使用的数学水平。因此,我故意在写这本书的时候,温和地介绍了从复杂代数到四元数代数的形式数学符号。


\begin{thebibliography}{99}
    \bibitem{bib1-1} Crowe, M.J.: A History of Vector Analysis. Dover, New York (1994)
    \bibitem{bib1-2}Altmann, S.L.: Rotations, Quaternions and Double Groups. Dover, New York (1986) ISBN-13:
    978-0-486-44518-2
\end{thebibliography}