
\chap{四元数代数}
\section{介绍}
本章包含进一步的历史背景,四元数的发明,并涵盖四元数代数的演变。我展示了如何通过将四元数视为有序对来极大地简化四元数代数,并提供了加法、减法、实数、纯四元数和单位四元数的示例。在定义了复共轭、范数、四元数积、平方和逆之后,我将展示如何用矩阵表示四元数。本章最后总结了重要的定义和几个工作实例。



\section{一些历史}
Hamilton 定义了一个四元数$q$,它的相关规则为
$$
    q=s+i a+j b+k c, \quad s, a, b, c \in \mathbb{R}
$$
其中,
$$
    \begin{gathered}
        i^{2}=j^{2}=k^{2}=i j k=-1 \\
        i j=k, \quad j k=i, \quad k i=j \\
        j i=-k, k j=-i, i k=-j
    \end{gathered}
$$
引用\cite{bib6-1,bib6-2,bib6-3} 但我们倾向于将四元数写成
$$
    q=s+a i+b j+c k
$$
从 Hamilton 的规则中观察$i  j$的出现是如何被$k$所取代的。额外的虚数k项是循环模式$i j=k, j k=i$和$k i=j$的关键,它们非常类似于两个单位笛卡尔矢量的叉乘:
$$
    \mathbf{i} \times \mathbf{j}=\mathbf{k}, \quad \mathbf{j} \times \mathbf{k}=\mathbf{i}, \quad \mathbf{k} \times \mathbf{i}=\mathbf{j} .
$$
事实上,这种相似性并非巧合,因为 Hamilton 也发明了标量和向量乘积。然而,尽管四元数提供了一个描述向量的代数框架,人们必须承认,在 Hamilton 之前,向量已经被研究了很多年。

Hamilton 还发现$i, j, k$项可以表示三个笛卡尔单位向量$\mathbf{i}, \mathbf{j}$和$\mathbf{k}$,它们必须具有虚数性质。例如$\mathbf{i}^{2}=-1$,等等,这对于一些数学家和科学家来说不太好,他们怀疑是否有必要涉及这么多虚项。

Hamilton 寻找复数的三维等效物的动机部分是代数的,部分是几何的。因为如果一个复数是由有序对表示的,并且能够以$90^{\circ}$旋转平面上的点,那么也许一个三元组可以以$90^{\circ}$旋转空间中的点。最后,一个三元组必须被一个四元组-一个四元数取代。

我们可以从两个角度来看待 Hamilton 的规则。首先,它们是三个虚项组合的代数结果。第二,它们反映了一个潜在的空间几何结构。后一种解释被P. G. Tait(p.g.泰特)所采用,并在他的《四元数的基本论述》一书中进行了概述。Tait的方法假设三个单位向量$\mathbf{i}, \mathbf{j}$, $\mathbf{k}$分别与$x$-, $y$-, $z$-轴对齐:

\begin{CJK}{UTF8}{gkai}
    $\mathbf{i}$与$\mathbf{j}$相乘的结果或$\mathbf{ij}$定义为$\mathbf{j}$在垂直于$\mathbf{i}$的平面上沿正方向转动一个直角,换句话说,$\mathbf{i}$对$\mathbf{j}$的操作将其翻转过来,使其与$\mathbf{k}$重合;因此简而言之$\mathbf{i} \mathbf{j}=\mathbf{k}$。

    为了保持一致,必须承认,如果$\mathbf{i}$不是对$\mathbf{j}$进行操作,而是对平面$y z$中垂直于$\mathbf{i}$的任何其他单位向量进行操作,它将使其在同一方向上经过一个直角,因此$\mathbf{i k}$只能是$-\mathbf{j}$。

    将我们引用$\mathbf{i}$来说明的定义扩展到其他单位向量,很明显,$\mathbf{j}$对$\mathbf{k}$的运算必须使其转回$\mathbf{i}$,或$\mathbf{j} \mathbf{k}=\mathbf{i}$。\cite{bib6-4}
\end{CJK}

其解释如图\ref{fig:6-1} a-d所示。图\ref{fig:6-1}a显示了 $\mathbf{i}$, $\mathbf{j}$, $\mathbf{k}$的原始对齐方式。图\ref{fig:6-1}b显示了 $\mathbf{j}$ 转动 $\mathbf{k}$ 的效果。图\ref{fig:6-1}c显示了 $\mathbf{k}$ 转动 $\mathbf{i}$,图6.1d显示了 $\mathbf{i}$ 转动 $\mathbf{j}$。

\begin{figure}[htbp]
    \centering
    \subfigure[]{\includegraphics[max width=0.3\textwidth]{2023_04_20_41f1ceac5a31dc7d1b59g-089(1)}}
    \subfigure[]{\includegraphics[max width=0.3\textwidth]{2023_04_20_41f1ceac5a31dc7d1b59g-089(2)}}\\
    \subfigure[]{\includegraphics[max width=0.3\textwidth]{2023_04_20_41f1ceac5a31dc7d1b59g-089}}
    \subfigure[]{\includegraphics[max width=0.3\textwidth]{2023_04_20_41f1ceac5a31dc7d1b59g-089(3)}}
    \caption[short]{ 解释乘积$\mathbf{j k,  ki, ij}$}
    \label{fig:6-1}
\end{figure}


到目前为止,还没有提到虚数——我们只有:
$$
    \begin{aligned}
         & \mathbf{i j}=\mathbf{k}, \quad \mathbf{j k}=\mathbf{i}, \quad \mathbf{k i}=\mathbf{j} \\
         & \mathbf{j i}=-\mathbf{k}, \mathbf{k j}=-\mathbf{i}, \mathbf{i k}=-\mathbf{j} .
    \end{aligned}
$$

如果我们假设这些向量服从代数的分配律和结合律,它们的虚性质就暴露出来了。例如:
$$
    \mathbf{i j}=\mathbf{k}
$$

然后以$\mathbf{i}$相乘:
$$
    \mathbf{i i j}=\mathbf{i k}=-\mathbf{j}
$$

因此,
$$
    \mathbf{i i}=\mathbf{i}^{2}=-1
$$

同样,我们可以展示 $\mathbf{j}^{2}=\mathbf{k}^{2}=-1$.

接下来:

$$
    \mathbf{i} \mathbf{j} \mathbf{k}=\mathbf{i}(\mathbf{j} \mathbf{k})=\mathbf{i} \mathbf{i}=\mathbf{i}^{2}=-1
$$

因此,仅仅通过声明叉积的作用, Hamilton 的规则就出现了,带有所有虚数的特征。它还提出了下列意见:

\begin{CJK}{UTF8}{gkai}
    Servois 于 1813 年在热尔贡(Gergonne)的《年鉴》上发表了一个非常奇特的推测,这是迄今为止发现的唯一一个包含有对四元数的最微小的预想痕迹的推测。 为了把平面的$a+b \sqrt{-1}$形式推广到空间,他通过类比写出了空间有向单位线的形式
    $$
        p \cos \alpha+q \cos \beta+r \cos \gamma
    $$

    在这里,$\alpha, \beta, \gamma$分别是它对三个轴的倾斜角。他很容易意识到$p, q, r$必须是非实数:但是,他问道,“它们是否可以化简为一般形式 $A+B\sqrt{-1}$?” 这不可能是答案。事实上,它们就是四元数微积分中的 $\mathbf{i}, \mathbf{j}, \mathbf{k}$。\cite{bib6-4}

\end{CJK}

因此,法国数学家弗朗索瓦-约瑟夫·塞尔瓦(François-Joseph Servois,1768-1847)也曾非常接近发现四元数。此外, Tait 和 Hamilton 显然都不知道 Rodrigues 发表的论文。

而且还不止于此。杰出的德国数学家卡尔·弗里德里希·高斯(Carl Friedrich Gauss, 1777-1855)非常谨慎,对发表任何过于革命性的东西都感到紧张,以防他被同行的数学家嘲笑。他的日记显示,他比尼古拉·伊万诺维奇·罗巴切夫斯基(Nikolai Ivanovich Lobachevsky)更早发现了非欧几里得几何。1819年,他在日记中的一个简短笔记中\cite{bib6-5}透露,他已经确定了一种求两个四元数$(a, b, c, d)$和$(\alpha, \beta, \gamma, \delta)$乘积的方法:
$$
    \begin{aligned}
        (A, B, C, D)= & (a, b, c, d)(\alpha, \beta, \gamma, \delta)                               \\
        =             & (a \alpha-b \beta-c \gamma-d \delta, a \beta+b \alpha-c \delta+d \gamma   \\
                      & a \gamma+b \delta+c \alpha-d \beta, a \delta-b \gamma+c \beta+d \alpha) .
    \end{aligned}
$$
乍一看,这个结果看起来不像四元数乘积,但如果我们将四元数的第二个和第三个坐标转置,并将它们视为四元数,我们有:

$$
    \begin{aligned}
        (A, B, C, D) & =(a+c i+b j+d k)(\alpha+\gamma i+\beta j+\delta k)                                    \\
                     & =a \alpha-c \gamma-b \beta-d \delta+a(\gamma i+\beta j+\delta k)                      \\
                     & =+\alpha(c i+b j+d k),(b \delta-d \beta) i+(d \gamma-c \delta) j+(c \beta-b \gamma) k
    \end{aligned}
$$
这和 Hamilton 的四元数乘积是一样的!此外, Gauss 还意识到乘积是不可交换的。然而,他没有发表他的发现,而是让 Hamilton 自己发明了四元数,发表了他的结果,并获得了荣誉。

在1881年和1884年,耶鲁大学的约西亚·威拉德·吉布斯(Josiah Willard Gibbs)为他的学生印刷了他关于矢量分析的课堂笔记。吉布斯切断了四元数的实部和矢量部之间的“脐带”,并将三维矢量作为一个独立的对象提出,没有任何虚数的内涵。 Gibbs 还采纳了德国数学家赫尔曼·格莱斯曼(Hermann Günter Grassmann, 1809-1877)的思想,后者自1832年以来一直在发展自己的向量系统思想。Gibbs 还使用四元数乘积的相关部分定义了标量和向量乘积。最后,在1901年, Gibbs 的学生埃德温·比德威尔·威尔逊(Edwin Bidwell Wilson)以书的形式出版了吉布斯的笔记:《矢量分析》\cite{bib6-6},其中包含了今天使用的符号。

四元数代数无疑是虚数,然而,仅仅通过分离矢量部分并忽略虚数规则, Gibbs 就揭示了数学的一个新分支,即矢量分析。Hamilton 和他的支持者无法说服同行四元数可以表示矢量,最终, Gibbs 的符号赢得了胜利,四元数也逐渐淡出了人们的视野。

近年来,四元数被飞行模拟行业重新发现,最近又被计算机图形学界重新发现,并被用于围绕任意轴旋转矢量。在此期间,许多人都有机会对四元数进行研究,并提出了利用其特性的新方法。


让我们看看四元数 $q$ 的三种写法:
\begin{align}
    q= & s+x i+y j+z k                                                                        \\
    q= & s+\mathbf{v}                                                                         \\
    q= & {[s, \mathbf{v}] }                                                                   \\
       & \text { 其中 } s, x, y, z \in \mathbb{R}, \quad \mathbf{v} \in \mathbb{R}^{3} \notag \\
       & \text { 和 } i^{2}=j^{2}=k^{2}=-1 .\notag
\end{align}

区别是相当微妙的。在(6.1)中,我们有 Hamilton 的原始定义及其虚数术语和相关规则。在(6.2)中,' $+$ '符号用于向向量添加标量,这看起来很奇怪,但却有效。在(6.3)中,我们有一个由标量和向量组成的有序对。

现在你可能会想:同一个物体怎么可能有三种不同的定义?我认为你可以随便叫一个对象,只要它们在代数上是相同的。例如,用矩阵表示法表示一组线性式,得到的结果与普通式相同。因此,两种表示法是同样有效的。

虽然我在其他出版物中使用了(6.1)和(6.2)中的符号,但在本书中我使用的是有序对。因此,我们需要证明的是, Hamilton 对四元数的原始定义(6.1),包括它的标量和三个虚数项,可以被一个由标量和一个“现代”向量组成的有序对(6.3)所取代。

\section{四元数定义}

让我们从两个四元数$q_{a}$和$q_{b}$ à la Hamilton开始:\footnote{译注,原句是:Let's start with two quaternions $q_{a}$ and $q_{b}$ à la Hamilton:,感觉原句有错误排过来的字符。 }

$$
    \begin{aligned}
         & q_{a}=s_{a}+x_{a} i+y_{a} j+z_{a} k \\
         & q_{b}=s_{b}+x_{b} i+y_{b} j+z_{b} k
    \end{aligned}
$$
和强制性规则:
$$
    \begin{gathered}
        i^{2}=j^{2}=k^{2}=i j k=-1 \\
        i j=k, \quad j k=i, \quad k i=j \\
        j i=-k, k j=-i, i k=-j .
    \end{gathered}
$$
我们的目标是证明 $q_{a}$ 和 $q_{b}$ 也可以用有序对来表示
$$
    \begin{aligned}
        q_{a} & =\left[s_{a}, \mathbf{a}\right]                                                                                       \\
        q_{b} & =\left[s_{b}, \mathbf{b}\right], \quad s_{a}, s_{b} \in \mathbb{R}, \quad \mathbf{a}, \mathbf{b} \in \mathbb{R}^{3} .
    \end{aligned}
$$
四元数乘积 $q_{a} q_{b}$ 展开为
\begin{align}
    \begin{aligned}
        q_{a} q_{b}=\left[s_{a}, \mathbf{a}\right]\left[s_{b}, \mathbf{b}\right]= & {\left[s_{a}+x_{a} i+y_{a} j+z_{a} k\right]\left[s_{b}+x_{b} i+y_{b} j+z_{b} k\right] } \\
        =                                                                         & {\left[\left(s_{a} s_{b}-x_{a} x_{b}-y_{a} y_{b}-z_{a} z_{b}\right)\right.}             \\
                                                                                  & +\left(s_{a} x_{b}+s_{b} x_{a}+y_{a} z_{b}-y_{b} z_{a}\right) i                         \\
                                                                                  & +\left(s_{a} y_{b}+s_{b} y_{a}+z_{a} x_{b}-z_{b} x_{a}\right) j                         \\
                                                                                  & \left.+\left(s_{a} z_{b}+s_{b} z_{a}+x_{a} y_{b}-x_{b} y_{a}\right) k\right] .
    \end{aligned}
    \label{ep:6.4}
\end{align}

式 (6.4) 采用了另一种四元数的形式,证实了四元数乘积是封闭的。

在这个阶段, Hamilton 把虚项 $i,j,k$ 转化为单位直角坐标向量 $\mathbf{i}, \mathbf{j}, \mathbf{k}$ 并把 (6.4) 转化为向量形式。这种方法的问题在于向量保留了它们的虚根。Simon Altmann 的建议是用有序对代替虚数:
$$
    i=[0, \mathbf{i}], \quad j=[0, \mathbf{j}], \quad k=[0, \mathbf{k}]
$$
它们本身就是四元数,叫做四元数单位。

用四元数单位来定义一个四元数和用单位笛卡尔向量来定义一个向量是完全一样的。此外,它允许向量在没有任何想象关联的情况下存在。

让我们将这些四元数单位与 $[1, \mathbf{0}]=1$ 一起代入 (6.4) 中:

\begin{align}
    \begin{aligned}
        {\left[s_{a}, \mathbf{a}\right]\left[s_{b}, \mathbf{b}\right]=} & {\left[\left(s_{a} s_{b}-x_{a} x_{b}-y_{a} y_{b}-z_{a} z_{b}\right)[1, \mathbf{0}]\right.}  \\
                                                                        & +\left(s_{a} x_{b}+s_{b} x_{a}+y_{a} z_{b}-y_{b} z_{a}\right)[0, \mathbf{i}]                \\
                                                                        & +\left(s_{a} y_{b}+s_{b} y_{a}+z_{a} x_{b}-z_{b} x_{a}\right)[0, \mathbf{j}]                \\
                                                                        & \left.+\left(s_{a} z_{b}+s_{b} z_{a}+x_{a} y_{b}-x_{b} y_{a}\right)[0, \mathbf{k}]\right] .
    \end{aligned}
\end{align}
接下来,我们利用之前定义的规则展开 (6.5):
\begin{align}
    \begin{aligned}
        {\left[s_{a}, \mathbf{a}\right]\left[s_{b}, \mathbf{b}\right]=} & {\left[\left[s_{a} s_{b}-x_{a} x_{b}-y_{a} y_{b}-z_{a} z_{b}, \mathbf{0}\right]\right.}                \\
                                                                        & +\left[0,\left(s_{a} x_{b}+s_{b} x_{a}+y_{a} z_{b}-y_{b} z_{a}\right) \mathbf{i}\right]                \\
                                                                        & +\left[0,\left(s_{a} y_{b}+s_{b} y_{a}+z_{a} x_{b}-z_{b} x_{a}\right) \mathbf{j}\right]                \\
                                                                        & \left.+\left[0,\left(s_{a} z_{b}+s_{b} z_{a}+x_{a} y_{b}-x_{b} y_{a}\right) \mathbf{k}\right]\right] .
    \end{aligned}
\end{align}
纵向扫描 (6.6) 可以发现一些隐藏向量:
\begin{align}
    \begin{aligned}
        {\left[s_{a}, \mathbf{a}\right]\left[s_{b}, \mathbf{b}\right]=} & {\left[\left[s_{a} s_{b}-x_{a} x_{b}-y_{a} y_{b}-z_{a} z_{b}, \mathbf{0}\right]\right.}                                                                                      \\
                                                                        & +\left[0, s_{a}\left(x_{b} \mathbf{i}+y_{b} \mathbf{j}+z_{b} \mathbf{k}\right)+s_{b}\left(x_{a} \mathbf{i}+y_{a} \mathbf{j}+z_{a} \mathbf{k}\right)\right.                   \\
                                                                        & \left.\left.+\left(y_{a} z_{b}-y_{b} z_{a}\right) \mathbf{i}+\left(z_{a} x_{b}-z_{b} x_{a}\right) \mathbf{j}+\left(x_{a} y_{b}-x_{b} y_{a}\right) \mathbf{k}\right]\right] .
    \end{aligned}
\end{align}
式 (6.7) 包含两个有序对,现在可以将它们合并起来:
\begin{align}
    \begin{aligned}
        {\left[s_{a}, \mathbf{a}\right]\left[s_{b}, \mathbf{b}\right]=} & {\left[s_{a} s_{b}-x_{a} x_{b}-y_{a} y_{b}-z_{a} z_{b},\right.}                                                                                                 \\
                                                                        & +s_{a}\left(x_{b} \mathbf{i}+y_{b} \mathbf{j}+z_{b} \mathbf{k}\right)+s_{b}\left(x_{a} \mathbf{i}+y_{a} \mathbf{j}+z_{a} \mathbf{k}\right)                      \\
                                                                        & \left.+\left(y_{a} z_{b}-y_{b} z_{a}\right) \mathbf{i}+\left(z_{a} x_{b}-z_{b} x_{a}\right) \mathbf{j}+\left(x_{a} y_{b}-x_{b} y_{a}\right) \mathbf{k}\right] .
    \end{aligned}
\end{align}
如果我们令
$$
    \begin{aligned}
         & \mathbf{a}=x_{a} \mathbf{i}+y_{a} \mathbf{j}+z_{a} \mathbf{k} \\
         & \mathbf{b}=x_{b} \mathbf{i}+y_{b} \mathbf{j}+z_{b} \mathbf{k}
    \end{aligned}
$$

将它们代入 (6.8) 即可得:

\begin{align}
    \left[s_{a}, \mathbf{a}\right]\left[s_{b}, \mathbf{b}\right]=\left[s_{a} s_{b}-\mathbf{a} \cdot \mathbf{b}, s_{a} \mathbf{b}+s_{b} \mathbf{a}+\mathbf{a} \times \mathbf{b}\right]
\end{align}
它定义了四元数乘积。

从现在开始,我们不必担心 Hamilton 的规则,因为它们嵌入在挨(6.9)中。此外,我
们的向量没有虚数的关联。

虽然 Rodrigues 没有使用(6.9)中使用的 Gibbs 向量符号,但他设法计算出等效的代数
表达式,这是一项成就。

\subsection{四元数单位}
使用(6.9),我们可以通过对四元数单位进行平方来检查四元数单位是否为虚数:
$$
    \begin{aligned}
        i     & =[0, \mathbf{i}]                                             \\
        i^{2} & =[0, \mathbf{i}][0, \mathbf{i}]                              \\
              & =[\mathbf{i} \cdot \mathbf{i}, \mathbf{i} \times \mathbf{i}] \\
              & =[-1, \mathbf{0}]
    \end{aligned}
$$
这是一个实数四元数,相当于-1,确认$[0,\mathbf{i}]$是虚数。使用类似的展开,我们可以证明$[0,\mathbf{j}]$和$[0,\mathbf{k}]$具有相同的属性。现在我们来计算乘积 $i j$,  $j k$ 和 $k i$
$$
    \begin{aligned}
        i j & =[0, \mathbf{i}][0, \mathbf{j}]                               \\
            & =[-\mathbf{i} \cdot \mathbf{j}, \mathbf{i} \times \mathbf{j}] \\
            & =[0, \mathbf{k}]
    \end{aligned}
$$
也就是四元数单位 $k$。
$$
    \begin{aligned}
        j k & =[0, \mathbf{j}][0, \mathbf{k}]                               \\
            & =[-\mathbf{j} \cdot \mathbf{k}, \mathbf{j} \times \mathbf{k}] \\
            & =[0, \mathbf{i}]
    \end{aligned}
$$
也就是四元数单位 $i$.
$$
    \begin{aligned}
        k i & =[0, \mathbf{k}][0, \mathbf{i}]                               \\
            & =[-\mathbf{k} \cdot \mathbf{i}, \mathbf{k} \times \mathbf{i}] \\
            & =[0, \mathbf{j}]
    \end{aligned}
$$
也就是四元数单位 $j$.

接下来,让我们确认 $i j k=-1$:
$$
    \begin{aligned}
        \text { ijk } & =[0, \mathbf{i}][0, \mathbf{j}][0, \mathbf{k}]                \\
                      & =[0, \mathbf{k}][0, \mathbf{k}]                               \\
                      & =[-\mathbf{k} \cdot \mathbf{k}, \mathbf{k} \times \mathbf{k}] \\
                      & =[-1, \mathbf{0}]
    \end{aligned}
$$
它是一个实数四元数,等于-1,证实了$i j k=-1$。

因此,有序对的符号支持所有的 Hamilton 规则。然而,最后一个二重乘积假设四元数是结合的。我们再检查一下证明$(ij) k=i(j k)$
$$
    \begin{aligned}
        i(j k) & =[0, \mathbf{i}][0, \mathbf{j}][0, \mathbf{k}]                \\
               & =[0, \mathbf{i}][0, \mathbf{i}]                               \\
               & =[-\mathbf{i} \cdot \mathbf{i}, \mathbf{i} \times \mathbf{i}] \\
               & =[-1, \mathbf{0}]
    \end{aligned}
$$
这是正确的。

\subsection{四元数乘积示例}
虽然我们还没有发现如何使用四元数来旋转向量,但让我们通过评估一个例子来关注它们的代数特征。
$$
    \begin{aligned}
         & q_{a}=[1,2 \mathbf{i}+3 \mathbf{j}+4 \mathbf{k}] \\
         & q_{b}=[2,3 \mathbf{i}+4 \mathbf{j}+5 \mathbf{k}]
    \end{aligned}
$$
乘积 $q_{a} q_{b}$ 是
$$
    \begin{aligned}
        q_{a} q_{b}= & {[1,2 \mathbf{i}+3 \mathbf{j}+4 \mathbf{k}][2,3 \mathbf{i}+4 \mathbf{j}+5 \mathbf{k}] }                    \\
        =            & {[1 \times 2-(2 \times 3+3 \times 4+4 \times 5),}                                                          \\
                     & 1(3 \mathbf{i}+4 \mathbf{j}+5 \mathbf{k})+2(2 \mathbf{i}+3 \mathbf{j}+4 \mathbf{k})                        \\
                     & +(3 \times 5-4 \times 4) \mathbf{i}-(2 \times 5-4 \times 3) \mathbf{j}+(2 \times 4-3 \times 3) \mathbf{k}] \\
        =            & {[-36,7 \mathbf{i}+10 \mathbf{j}+13 \mathbf{k}-\mathbf{i}+2 \mathbf{j}-\mathbf{k}] }                       \\
        =            & {[-36,6 \mathbf{i}+12 \mathbf{j}+12 \mathbf{k}] }
    \end{aligned}
$$
这是表示另一个四元数的有序对。


在证明了 Hamilton 的虚符号有一个等价向量,并且可以表示为有序对之后,我们继续使用这个符号并描述四元数的其他特征。注意,我们可以放弃 Hamilton 的规则,因为它们嵌入在四元数乘积的定义中,并将在以下定义中出现。

\section{代数定义}
一个四元数是有序对:
$$
    q=[s, \mathbf{v}], \quad s \in \mathbb{R}, \quad \mathbf{v} \in \mathbb{R}^{3}
$$
如果我们用它的分量来表示$\mathbf{v}$,我们有
$$
    q=[s, x \mathbf{i}+y \mathbf{j}+z \mathbf{k}], \quad s, x, y, z \in \mathbb{R}
$$

\section{四元数加减法}
加减法使用以下规则:
$$
    \begin{aligned}
        q_{a}           & =\left[s_{a}, \mathbf{a}\right]                            \\
        q_{b}           & =\left[s_{b}, \mathbf{b}\right]                            \\
        q_{a} \pm q_{b} & =\left[s_{a} \pm s_{b}, \mathbf{a} \pm \mathbf{b}\right] .
    \end{aligned}
$$
示例:
$$
    \begin{aligned}
        q_{a}       & =[0.5,2 \mathbf{i}+3 \mathbf{j}-4 \mathbf{k}]   \\
        q_{b}       & =[0.1,4 \mathbf{i}+5 \mathbf{j}+6 \mathbf{k}]   \\
        q_{a}+q_{b} & =[0.6,6 \mathbf{i}+8 \mathbf{j}+2 \mathbf{k}]   \\
        q_{a}-q_{b} & =[0.4,-2 \mathbf{i}-2 \mathbf{j}-10 \mathbf{k}]
    \end{aligned}
$$

\section{实四元数}
实四元数有一个零向量项:
$$
    q=[s, \mathbf{0}]
$$
两个实四元数的乘积是
$$
    \begin{aligned}
        q_{a}       & =\left[s_{a}, \mathbf{0}\right]                               \\
        q_{b}       & =\left[s_{b}, \mathbf{0}\right]                               \\
        q_{a} q_{b} & =\left[s_{a}, \mathbf{0}\right]\left[s_{b}, \mathbf{0}\right] \\
                    & =\left[s_{a} s_{b}, \mathbf{0}\right]
    \end{aligned}
$$
这是另一个实数四元数,这表明它们的行为就像实数:
$$
    [s, \mathbf{0}] \equiv s.
$$
我们已经遇到过包含零虚项的复数:
$$
    a+b i=a, \quad \text { when } b=0
$$

\section{四元数乘以标量}
直觉告诉我们,四元数乘以标量应该遵循以下规则:
$$
    \begin{aligned}
        q         & =[s, \mathbf{v}]                                      \\
        \lambda q & =\lambda[s, \mathbf{v}], \quad \lambda \in \mathbb{R} \\
                  & =[\lambda s, \lambda \mathbf{v}] .
    \end{aligned}
$$
示例:
$$
    \begin{aligned}
        q & =3[2,3 \mathbf{i}+4 \mathbf{j}+5 \mathbf{k}]    \\
          & =[6,9 \mathbf{i}+12 \mathbf{j}+15 \mathbf{k}] .
    \end{aligned}
$$
我们可以通过将一个四元数乘以一个实四元数形式的标量来证实我们的直觉:
$$
    \begin{aligned}
        q         & =[s,  \mathbf{v}]  \\
        \lambda   & =[\lambda, \mathbf{0}]  \\
        \lambda q & =[\lambda, \mathbf{0}  ][s, \mathbf{v}] \\
                  & =[\lambda s,  \lambda \mathbf{v}  ]
    \end{aligned}
$$
这是很好的证明。

\section{纯四元数}
Hamilton 将纯四元数定义为标量项为零的四元数:
$$
    q=x i+y j+z k
$$

只是一个向量,但具有虚数性质。西蒙·奥特曼(Simon Altmann)和其他人认为,Hamilton 将实数为零的四元数称为向量是一个严重的错误。

主要问题是有两种类型的向量:极性和轴向,也称为伪向量。理查德·费曼( Richard Feynman )将极矢量描述为“诚实的”矢量[7],并表示有向直线的日常矢量。然而,轴矢量是由极矢量计算的,例如在矢量积中。然而,这两种类型的向量在转换时的行为并不相同。
例如,给定两个‘诚实’的极性向量 $\mathbf{a}$ 和 $\mathbf{b}$,我们可以计算轴向向量:$\mathbf{c}=\mathbf{a} \times \mathbf{b}$。接下来,如果我们通过原点对$\mathbf{a}$和$\mathbf{b}$进行反转变换,使得$\mathbf{a}$变成$-\mathbf{a}$, $\mathbf{b}$变成$-\mathbf{b}$,并计算它们的外积$(-\mathbf{a}) \times(-\mathbf{b})$,我们仍然得到$\mathbf{c}$! 这意味着轴向量$\mathbf{c}$不能与$\mathbf{a}$和$\mathbf{b}$一起变换。

可以认为,反转变换不是一个“适当的”变换,因为它把一组右旋轴变成了一组左旋轴。但在物理学中,自然法则在这两个系统中都适用。不幸的是, Hamilton 没有意识到这一区别,因为他刚刚发明了向量。然而,在这中间的几年里,很明显, Hamilton 的四元数向量是一个轴向向量,而不是一个极向向量。我们将看到,在三维旋转中,四元数的形式是
$$
    q=\left[\cos \left(\frac{\theta}{2}\right), \sin \left(\frac{\theta}{2}\right) \mathbf{v}\right]
$$
其中$\theta$是旋转角度,$\mathbf{v}$是旋转轴,当我们设置$\theta=180^{\circ}$时,我们得到
$$
    q=[0, \mathbf{v}]
$$
它仍然是一个四元数,即使它只包含一个向量部分。

因此,我们将纯四元数定义为
$$
    q=[0, \mathbf{v}]
$$
两个纯四元数的乘积是
$$
    \begin{aligned}
        q_{a}       & =[0, \mathbf{a}]                                              \\
        q_{b}       & =[0, \mathbf{b}]                                              \\
        q_{a} q_{b} & =[0, \mathbf{a}][0, \mathbf{b}]                               \\
                    & =[-\mathbf{a} \cdot \mathbf{b}, \mathbf{a} \times \mathbf{b}]
    \end{aligned}
$$
这不再是“纯的”,因为一些原始向量信息已经通过点积“隧穿”到实部。

\section{单位四元数}
让我们通过引入一些熟悉的向量符号来进一步进行分析。

给出向量 $\mathbf{v}$, 然后
$$
    \mathbf{v}=\lambda \hat{\mathbf{v}}, \quad \text { 其中 } \lambda=\|\mathbf{v}\| \text { 且 }\|\hat{\mathbf{v}}\|=1
$$
将其与纯四元数的定义结合起来,我们得到:
$$
    \begin{aligned}
        q & =[0, \mathbf{v}]               \\
          & =[0, \lambda \hat{\mathbf{v}}] \\
          & =\lambda[0, \hat{\mathbf{v}}]
    \end{aligned}
$$
并揭示出对象 $[0,\hat{\mathbf{v}}]$,它被称为单位四元数,由一个零标量和一个单位向量组成。 为了方便起见,我们把这个单位四元数称为 $\hat{q}$ :
$$
    \hat{q}=[0, \hat{\mathbf{v}}]
$$
因此,现在我们有了一种类似于向量的符号,即向量 $\mathbf{v}$ 是用它的单位形式来描述的:
$$
    \mathbf{v}=\lambda \hat{\mathbf{v}}
$$
而四元数 $q$ 也可以用其单位形式来描述:
$$
    q=\lambda \hat{q}
$$
注意 $\hat{q}$ 是一个虚数对象,因为它的平方为 -1 :
$$
    \begin{aligned}
        \hat{q}^{2} & =[0, \hat{\mathbf{v}}][0, \hat{\mathbf{v}}]                                           \\
                    & =[-\hat{\mathbf{v}} \cdot \hat{\mathbf{v}}, \hat{\mathbf{v}} \times \hat{\mathbf{v}}] \\
                    & =[-1, \mathbf{0}]                                                                     \\
                    & =-1
    \end{aligned}
$$
考虑到 Hamilton 最初的发明,这并不太令人惊讶!

\section{四元数的加法形式}
现在我们来讨论将四元数拆分为实四元数和纯四元数的问题。同样,直觉告诉我们,我们可以把四元数写成
$$
    \begin{aligned}
        q & =[s, \mathbf{v}]                 \\
          & =[s, \mathbf{0}]+[0, \mathbf{v}]
    \end{aligned}
$$
我们可以通过将这样表示的两个四元数形成代数积来验证这一点:
$$
    \begin{aligned}
        q_{a}       & =\left[s_{a}, \mathbf{0}\right]+[0, \mathbf{a}]                                                                                                                                          \\
        q_{b}       & =\left[s_{b}, \mathbf{0}\right]+[0, \mathbf{b}]                                                                                                                                          \\
        q_{a} q_{b} & =\left(\left[s_{a}, \mathbf{0}\right]+[0, \mathbf{a}]\right)\left(\left[s_{b}, \mathbf{0}\right]+[0, \mathbf{b}]\right)                                                                  \\
                    & =\left[s_{a}, \mathbf{0}\right]\left[s_{b}, \mathbf{0}\right]+\left[s_{a}, \mathbf{0}\right][0, \mathbf{b}]+[0, \mathbf{a}]\left[s_{b}, \mathbf{0}\right]+[0, \mathbf{a}][0, \mathbf{b}] \\
                    & =\left[s_{a} s_{b}, \mathbf{0}\right]+\left[0, s_{a} \mathbf{b}\right]+\left[0, s_{b} \mathbf{a}\right]+[-\mathbf{a} \cdot \mathbf{b}, \mathbf{a} \times \mathbf{b}]                     \\
                    & =\left[s_{a} s_{b}-\mathbf{a} \cdot \mathbf{b}, s_{a} \mathbf{b}+s_{b} \mathbf{a}+\mathbf{a} \times \mathbf{b}\right]
    \end{aligned}
$$
这是正确的,证实了加法形式是有效的。

\section{四元数的二元形式}
在证明了四元数的加法形式有效并发现了单位四元数之后,我们可以将这两个对象连接起来,如下所示:
$$
    \begin{aligned}
        q & =[s, \mathbf{v}]                              \\
          & =[s, \mathbf{0}]+[0, \mathbf{v}]              \\
          & =[s, \mathbf{0}]+\lambda[0, \hat{\mathbf{v}}] \\
          & =s+\lambda \hat{q} .
    \end{aligned}
$$

概括地说,$s$ 是标量,$\lambda$ 是向量项的长度,$\hat{q}$ 是单位四元数$[0, \hat{\mathbf{v}}]$。

看看这个符号与复数多么相似:
$$
    \begin{aligned}
         & z=a+b i             \\
         & q=s+\lambda \hat{q}
    \end{aligned}
$$
其中 $a,b,s,\lambda$ 为标量,$i$ 为单位虚数,$\hat{q}$ 为单位四元数。

\section{四元数的复共轭}
我们已经发现,复数 $z=a+b \mathrm{i}$ 的共轭值如下给出
$$
    z^{*}=a-b \mathrm{i}
$$

在计算 $z$ 的逆时非常有用。四元共轭在计算四元数的逆时也起着类似的作用。因此,给出
$$
    q=[s, \mathbf{v}]
$$

四元数共轭定义为
$$
    q^{*}=[s,-\mathbf{v}]
$$

示例:
$$
    \begin{aligned}
        q     & =[2,3 \mathbf{i}-4 \mathbf{j}+5 \mathbf{k}]  \\
        q^{*} & =[2,-3 \mathbf{i}+4 \mathbf{j}-5 \mathbf{k}]
    \end{aligned}
$$

如果计算乘积 $q q^{*}$,我们会得到
$$
    \begin{aligned}
        q q^{*} & =[s, \mathbf{v}][s,-\mathbf{v}]                                                                             \\
                & =\left[s^{2}-\mathbf{v} \cdot(-\mathbf{v}),-s \mathbf{v}+s \mathbf{v}+\mathbf{v} \times(-\mathbf{v})\right] \\
                & =\left[s^{2}+\mathbf{v} \cdot \mathbf{v}, \mathbf{0}\right]                                                 \\
                & =\left[s^{2}+v^{2}, \mathbf{0}\right] .
    \end{aligned}
$$

让我们证明 $q q^{*}=q^{*} q$ :
$$
    \begin{aligned}
        q^{*} q & =[s,-\mathbf{v}][s, \mathbf{v}]                                                                               \\
                & =\left[s^{2}-(-\mathbf{v}) \cdot \mathbf{v}, s \mathbf{v}-s \mathbf{v}+(-\mathbf{v}) \times \mathbf{v}\right] \\
                & =\left[s^{2}+\mathbf{v} \cdot \mathbf{v}, \mathbf{0}\right]                                                   \\
                & =\left[s^{2}+v^{2}, \mathbf{0}\right]                                                                         \\
                & =q q^{*}
    \end{aligned}
$$

现在证明 $\left(q_{a} q_{b}\right)^{*}=q_{b}^{*} q_{a}^{*}$.
\begin{align}
    q_{a}       & =[s_{a}, \mathbf{a}]  \notag\\
    q_{b}       & =[s_{b}, \mathbf{b}]  \notag\\
    q_{a} q_{b} & =[s_{a}, \mathbf{a}][s_{b}, \mathbf{b}]           \notag\\
                & =[s_{a} s_{b}-\mathbf{a} \cdot \mathbf{b}, s_{a} \mathbf{b}+s_{b} \mathbf{a}+\mathbf{a} \times \mathbf{b}]   \notag\\
    (q_{a} q_{b})^{*} & =[s_{a} s_{b}-\mathbf{a} \cdot \mathbf{b},-s_{a} \mathbf{b}-s_{b} \mathbf{a}-\mathbf{a} \times \mathbf{b}] .
\end{align}

接下来计算 $q_{b}^{*} q_{a}^{*}$


\begin{align}
    q_{a}^{*}           & =[s_{a},-\mathbf{a}] \notag\\
    q_{b}^{*}           & =[s_{b},-\mathbf{b}]   \notag \\
    q_{b}^{*} q_{a}^{*} & =[s_{b},-\mathbf{b}][s_{a},-\mathbf{a}] \notag\\
                        & =[s_{a} s_{b}-\mathbf{a} \cdot \mathbf{b},-s_{a} \mathbf{b}-s_{b} \mathbf{a}-\mathbf{a} \times \mathbf{b}] .
\end{align}

而 (6.10) 等于 (6.11),$(q_{a} q_{b})^{*}=q_{b}^{*} q_{a}^{*}$ 。

\section{四元数的范数}
复数 $z=a+b i$ 的范数定义如下:
$$
    |z|=\sqrt{a^{2}+b^{2}}
$$
这允许我们写出
$$
    z z^{*}=|z|^{2}
$$

同样,四元数 $q$ 的规范定义为
$$
    \begin{aligned}
        q   & =[s, \mathbf{v}]               \\
            & =[s, \lambda \hat{\mathbf{v}}] \\
        |q| & =\sqrt{s^{2}+\lambda^{2}}
    \end{aligned}
$$
其中 $\lambda=\|\mathbf{v}\|$ 可以写成
$$
    q q^{*}=|q|^{2}
$$
示例
$$
    \begin{aligned}
        q   & =[1,4 \mathbf{i}+4 \mathbf{j}-4 \mathbf{k}] \\
        |q| & =\sqrt{1^{2}+4^{2}+4^{2}+(-4)^{2}}          \\
            & =\sqrt{49}                                  \\
            & =7
    \end{aligned}
$$

\section{归一化四元数}
具有单位范数的四元数称为归一化四元数。例如,四元数 $q=[s, \mathbf{v}]$ 通过除以 $|q|$ 而归一化:
$$
    q^{\prime}=\frac{q}{\sqrt{s^{2}+\lambda^{2}}}
$$
我们必须注意不要混淆单位四元数和单位范数四元数。单位四元数是 $[0,\hat{\mathbf{v}}]$,带有单位矢量部分,而单位范数四元数是归一化的,使得 $s^{2}+\lambda^{2}=1$。

我将仔细区分这两个术语,因为包括我自己在内的许多作者都使用单位四元数一词来描述具有单位范数的四元数。例如
$$
    q=[1,4 \mathbf{i}+4 \mathbf{j}-4 \mathbf{k}]
$$
的范数为 7 ,而 $q$ 是通过除以 7 来归一化的:
$$
    q^{\prime}=\frac{1}{7}[1,4 \mathbf{i}+4 \mathbf{j}-4 \mathbf{k}]
$$

The type of unit-norm quaternion we will be using takes the form:

$$
    q=\left[\cos \left(\frac{\theta}{2}\right), \sin \left(\frac{\theta}{2}\right) \hat{\mathbf{v}}\right]
$$

because $\cos ^{2} \theta+\sin ^{2} \theta=1$

\section{四元数乘积}
Having shown that ordered pairs can represent a quaternion and its various manifestations, let's summarise the products we will eventually encounter. To start, we have the product of two normal quaternions:

$$
    \begin{aligned}
        q_{a} q_{b} & =\left[s_{a}, \mathbf{a}\right]\left[s_{b}, \mathbf{b}\right]                                                           \\
                    & =\left[s_{a} s_{b}-\mathbf{a} \cdot \mathbf{b}, s_{a} \mathbf{b}+s_{b} \mathbf{a}+\mathbf{a} \times \mathbf{b}\right] .
    \end{aligned}
$$

\subsection{纯四元数的乘积}
Given two pure quaternions:

$$
    \begin{aligned}
         & q_{a}=[0, \mathbf{a}] \\
         & q_{b}=[0, \mathbf{b}]
    \end{aligned}
$$

their product is

$$
    \begin{aligned}
        q_{a} q_{b} & =[0, \mathbf{a}][0, \mathbf{b}]                                 \\
                    & =[-\mathbf{a} \cdot \mathbf{b}, \mathbf{a} \times \mathbf{b}] .
    \end{aligned}
$$

\subsection{两个归一化四元数的乘积}
Given two unit-norm quaternions:

$$
    \begin{aligned}
         & q_{a}=\left[s_{a}, \mathbf{a}\right] \\
         & q_{b}=\left[s_{b}, \mathbf{b}\right]
    \end{aligned}
$$

where $\left|q_{a}\right|=\left|q_{b}\right|=1$. Their product is another unit-norm quaternion, which is proved as follows.

We assume $q_{c}=\left[s_{c}, \mathbf{c}\right]$ and show that $\left|q_{c}\right|=s_{c}^{2}+c^{2}=1$ where

$$
    \begin{aligned}
        {\left[s_{c}, \mathbf{c}\right] } & =\left[s_{a}, \mathbf{a}\right]\left[s_{b}, \mathbf{b}\right]                                                           \\
                                          & =\left[s_{a} s_{b}-\mathbf{a} \cdot \mathbf{b}, s_{a} \mathbf{b}+s_{b} \mathbf{a}+\mathbf{a} \times \mathbf{b}\right] .
    \end{aligned}
$$

Let's assume the angle between $\mathbf{a}$ and $\mathbf{b}$ is $\theta$, which permits us to write:

$$
    \begin{aligned}
         & s_{c}=s_{a} s_{b}-a b \cos \theta                                                                                        \\
         & \mathbf{c}=s_{a} b \hat{\mathbf{b}}+s_{b} a \hat{\mathbf{a}}+a b \sin \theta(\hat{\mathbf{a}} \times \hat{\mathbf{b}}) .
    \end{aligned}
$$

Fig. 6.2 Geometry for $s_{a} b \hat{\mathbf{b}}+s_{b} a \hat{\mathbf{a}}+a b \sin \theta(\hat{\mathbf{a}} \times$ b)

    \begin{center}
        \includegraphics[max width=\textwidth]{2023_04_20_41f1ceac5a31dc7d1b59g-104}
    \end{center}

    Therefore,

    $$
        \begin{aligned}
            s_{c}^{2} & =\left(s_{a} s_{b}-a b \cos \theta\right)\left(s_{a} s_{b}-a b \cos \theta\right) \\
                      & =s_{a}^{2} s_{b}^{2}-2 s_{a} s_{b} a b \cos \theta+a^{2} b^{2} \cos ^{2} \theta .
        \end{aligned}
    $$

    Figure 6.2 shows the geometry representing $\mathbf{c}$.

    $$
        \begin{aligned}
            d^{2}           & =s_{b}^{2} a^{2}+s_{a}^{2} b^{2}-2 s_{a} s_{b} a b \cos (\pi-\theta)                                                                                                       \\
                            & =s_{b}^{2} a^{2}+s_{a}^{2} b^{2}+2 s_{a} s_{b} a b \cos \theta                                                                                                             \\
            c^{2}           & =d^{2}+a^{2} b^{2} \sin ^{2} \theta                                                                                                                                        \\
                            & =s_{b}^{2} a^{2}+s_{a}^{2} b^{2}+2 s_{a} s_{b} a b \cos \theta+a^{2} b^{2} \sin ^{2} \theta                                                                                \\
            s_{c}^{2}+c^{2} & =s_{a}^{2} s_{b}^{2}-2 s_{a} s_{b} a b \cos \theta+a^{2} b^{2} \cos ^{2} \theta+s_{b}^{2} a^{2}+s_{a}^{2} b^{2}+2 s_{a} s_{b} a b \cos \theta+a^{2} b^{2} \sin ^{2} \theta \\
                            & =s_{a}^{2} s_{b}^{2}+a^{2} b^{2}+s_{b}^{2} a^{2}+s_{a}^{2} b^{2}                                                                                                           \\
                            & =s_{a}^{2}\left(s_{b}^{2}+b^{2}\right)+a^{2}\left(s_{b}^{2}+b^{2}\right)                                                                                                   \\
                            & =s_{a}^{2}+a^{2}                                                                                                                                                           \\
                            & =1
        \end{aligned}
    $$

    Therefore, the product of two unit-norm quaternions is another unit-norm quaternion. Consequently, multiplying a quaternion by a unit-norm quaternion, does not change its norm:

    $$
        \begin{aligned}
            q_{a}                    & =\left[s_{a}, \mathbf{a}\right] \\
            \left|q_{a}\right|       & =1                              \\
            q_{b}                    & =\left[s_{b}, \mathbf{b}\right] \\
            \left|q_{a} q_{b}\right| & =\left|q_{b}\right| .
        \end{aligned}
    $$

    \subsection{四元数的平方}
    The square of a quaternion is given by:

    $$
        \begin{aligned}
            \mathbf{v} & =x \mathbf{i}+y \mathbf{j}+z \mathbf{k}                                                      \\
            q          & =[s, \mathbf{v}]                                                                             \\
            q^{2}      & =[s, \mathbf{v}][s, \mathbf{v}]                                                              \\
                       & =\left[s^{2}-\mathbf{v} \cdot \mathbf{v}, 2 s \mathbf{v}+\mathbf{v} \times \mathbf{v}\right] \\
                       & =\left[s^{2}-\mathbf{v} \cdot \mathbf{v}, 2 s \mathbf{v}\right]                              \\
                       & =\left[s^{2}-x^{2}-y^{2}-z^{2}, 2 s(x \mathbf{i}+y \mathbf{j}+z \mathbf{k})\right] .
        \end{aligned}
    $$

    For example:

    $$
        \begin{aligned}
            q     & =[7,2 \mathbf{i}+3 \mathbf{j}+4 \mathbf{k}]                                             \\
            q^{2} & =\left[7^{2}-2^{2}-3^{2}-4^{2}, \quad 14(2 \mathbf{i}+3 \mathbf{j}+4 \mathbf{k})\right] \\
                  & =[20,28 \mathbf{i}+42 \mathbf{j}+56 \mathbf{k}]
        \end{aligned}
    $$

    The square of a pure quaternion is

    $$
        \begin{aligned}
            \mathbf{v} & =x \mathbf{i}+y \mathbf{j}+z \mathbf{k}                        \\
            q          & =[0, \mathbf{v}]                                               \\
            q^{2}      & =[0, \mathbf{v}][0, \mathbf{v}]                                \\
                       & =[0-\mathbf{v} \cdot \mathbf{v}, \mathbf{v} \times \mathbf{v}] \\
                       & =[0-\mathbf{v} \cdot \mathbf{v}, \mathbf{0}]                   \\
                       & =\left[-\left(x^{2}+y^{2}+z^{2}\right), \mathbf{0}\right]
        \end{aligned}
    $$

    which makes the square of a pure, unit-norm quaternion equal to -1 , and was one of the results, to which some 19th-century mathematicians objected.

    \subsection{四元数乘积的范数}
    In proving that the product of two unit-norm quaternions is another unit-norm quaternion we saw that
    $$
        \begin{aligned}
            q_{a}                  & =\left[s_{a}, \mathbf{a}\right]                                          \\
            q_{b}                  & =\left[s_{b}, \mathbf{b}\right]                                          \\
            q_{c}                  & =q_{a} q_{b}                                                             \\
            \left|q_{c}\right|^{2} & =s_{a}^{2}\left(s_{b}^{2}+b^{2}\right)+a^{2}\left(s_{b}^{2}+b^{2}\right) \\
                                   & =\left(s_{a}^{2}+a^{2}\right)\left(s_{b}^{2}+b^{2}\right)
        \end{aligned}
    $$

    which, if we ignore the constraint of unit-norm quaternions, shows that the norm of a quaternion product equals the product of the individual norms:

    $$
        \begin{aligned}
            \left|q_{a} q_{b}\right|^{2} & =\left|q_{a}\right|^{2}\left|q_{b}\right|^{2} \\
            \left|q_{a} q_{b}\right|     & =\left|q_{a}\right|\left|q_{b}\right|
        \end{aligned}
    $$

    \section{逆四元数}
    An important feature of quaternion algebra is the ability to divide two quaternions $q_{b} / q_{a}$, as long as $q_{a}$ does not vanish.

    By definition, the inverse $q^{-1}$ of $q$ satisfies

    $$
        q q^{-1}=[1,0]=1
    $$

    To isolate $q^{-1}$, we multiply (6.12) by $q^{*}$

    $$
        \begin{aligned}
             & q^{*} q q^{-1}=q^{*} \\
             & |q|^{2} q^{-1}=q^{*}
        \end{aligned}
    $$

    and from (6.13) we can write

    $$
        q^{-1}=\frac{q^{*}}{|q|^{2}}
    $$

    If $q$ is a unit-norm quaternion, then

    $$
        q^{-1}=q^{*}
    $$

    which is useful in the context of rotations.

    Furthermore, as

    $$
        \left(q_{a} q_{b}\right)^{*}=q_{b}^{*} q_{a}^{*}
    $$

    then

    $$
        \left(q_{a} q_{b}\right)^{-1}=q_{b}^{-1} q_{a}^{-1}
    $$

    Note that $q q^{-1}=q^{-1} q$ :

    $$
        \begin{aligned}
             & q q^{-1}=\frac{q q^{*}}{|q|^{2}}=1 \\
             & q^{-1} q=\frac{q^{*} q}{|q|^{2}}=1
        \end{aligned}
    $$

    Thus, we represent the quotient $q_{b} / q_{a}$ as

    $$
        \begin{aligned}
            q_{c} & =\frac{q_{b}}{q_{a}}                            \\
                  & =q_{b} q_{a}^{-1}                               \\
                  & =\frac{q_{b} q_{a}^{*}}{\left|q_{a}\right|^{2}}
        \end{aligned}
    $$

    For completeness let's evaluate the inverse of $q$ where

    $$
        \begin{aligned}
            q                            & =\left[1, \frac{1}{\sqrt{3}} \mathbf{i}+\frac{1}{\sqrt{3}} \mathbf{j}+\frac{1}{\sqrt{3}} \mathbf{k}\right]              \\
            q^{*}                        & =\left[1,-\frac{1}{\sqrt{3}} \mathbf{i}-\frac{1}{\sqrt{3}} \mathbf{j}-\frac{1}{\sqrt{3}} \mathbf{k}\right]              \\
            |q|^{2}                      & =1+\frac{1}{3}+\frac{1}{3}+\frac{1}{3}=2                                                                                \\
            q^{-1}=\frac{q^{*}}{|q|^{2}} & =\frac{1}{2}\left[1,-\frac{1}{\sqrt{3}} \mathbf{i}-\frac{1}{\sqrt{3}} \mathbf{j}-\frac{1}{\sqrt{3}} \mathbf{k}\right] .
        \end{aligned}
    $$

    It should be clear that $q^{-1} q=1$ :

    $$
        \begin{aligned}
            q^{-1} q & =\frac{1}{2}\left[1,-\frac{1}{\sqrt{3}} \mathbf{i}-\frac{1}{\sqrt{3}} \mathbf{j}-\frac{1}{\sqrt{3}} \mathbf{k}\right]\left[1, \frac{1}{\sqrt{3}} \mathbf{i}+\frac{1}{\sqrt{3}} \mathbf{j}+\frac{1}{\sqrt{3}} \mathbf{k}\right] \\
                     & =\frac{1}{2}\left[1+\frac{1}{3}+\frac{1}{3}+\frac{1}{3}, \mathbf{0}\right]                                                                                                                                                     \\
                     & =1
        \end{aligned}
    $$

    \section{矩阵}
    Matrices provide another way to express a quaternion product. For convenience, let's repeat (6.8) again and show it in matrix form:

    $$
        \begin{aligned}
            {\left[s_{a}, \mathbf{a}\right]\left[s_{b}, \mathbf{b}\right]=} & {\left[s_{a} s_{b}-x_{a} x_{b}-y_{a} y_{b}-z_{a} z_{b}\right.}                                                                                                \\
                                                                            & +s_{a}\left(x_{b} \mathbf{i}+y_{b} \mathbf{j}+z_{b} \mathbf{k}\right)+s_{b}\left(x_{a} \mathbf{i}+y_{a} \mathbf{j}+z_{a} \mathbf{k}\right)                    \\
                                                                            & \left.+\left(y_{a} z_{b}-y_{b} z_{a}\right) \mathbf{i}+\left(z_{a} x_{b}-z_{b} x_{a}\right) \mathbf{j}+\left(x_{a} y_{b}-x_{b} y_{a}\right) \mathbf{k}\right] \\
                                                                            & {\left[\begin{array}{rrr}
                            s_{a}-x_{a}-y_{a}-z_{a}                 \\
                            x_{a}       & s_{a}-z_{a} & y_{a}       \\
                            y_{a}       & z_{a}       & s_{a}-x_{a} \\
                            z_{a}-y_{a} & x_{a}       & s_{a}
                        \end{array}\right]\left[\begin{array}{c}
                            s_{b} \\
                            x_{b} \\
                            y_{b} \\
                            z_{b}
                        \end{array}\right] }
        \end{aligned}
    $$

    Let's recompute the product $q_{a} q_{b}$ using the above matrix:

    $$
        \begin{aligned}
             & q_{a}=[1,2 \mathbf{i}+3 \mathbf{j}+4 \mathbf{k}]                                           \\
             & q_{b}=[2,3 \mathbf{i}+4 \mathbf{j}+5 \mathbf{k}]                                           \\
             & q_{a} q_{b}=\left[\begin{array}{rrrr}
                    1 & -2 & -3 & -4 \\
                    2 & 1  & -4 & 3  \\
                    3 & 4  & 1  & -2 \\
                    4 & -3 & 2  & 1
                \end{array}\right]\left[\begin{array}{l}
                    2 \\
                    3 \\
                    4 \\
                    5
                \end{array}\right] \\
             & =\left[\begin{array}{r}
                    -36 \\
                    6   \\
                    12  \\
                    12
                \end{array}\right]                                                   \\
             & =[-36,6 \mathbf{i}+12 \mathbf{j}+12 \mathbf{k}] \text {. }
        \end{aligned}
    $$

    \subsection{正交矩阵}
    We can demonstrate that the unit-norm quaternion matrix is orthogonal by showing that the product with its transpose equals the identity matrix. As we are dealing with matrices, $\mathbf{Q}$ will represent the matrix for $q$ :

    $$
        \begin{aligned}
             & q=[s, x \mathbf{i}+y \mathbf{j}+z \mathbf{k}]                                                                     \\
             & \text { where } 1=s^{2}+x^{2}+y^{2}+z^{2}                                                                         \\
             & \mathbf{Q}=\left[\begin{array}{rrrr}
                    s & -x & -y & -z \\
                    x & s  & -z & y  \\
                    y & z  & s  & -x \\
                    z & -y & x  & s
                \end{array}\right]                                                                \\
             & \mathbf{Q}^{\mathrm{T}}=\left[\begin{array}{rrrr}
                    s  & x  & y  & z  \\
                    -x & s  & z  & -y \\
                    -y & -z & s  & x  \\
                    -z & y  & -x & s
                \end{array}\right]                                                   \\
             & \mathbf{Q} \mathbf{Q}^{\mathrm{T}}=\left[\begin{array}{rrrr}
                    s & -x & -y & -z \\
                    x & s  & -z & y  \\
                    y & z  & s  & -x \\
                    z & -y & x  & s
                \end{array}\right]\left[\begin{array}{rrrr}
                    s  & x  & y  & z  \\
                    -x & s  & z  & -y \\
                    -y & -z & s  & x  \\
                    -z & y  & -x & s
                \end{array}\right] \\
             & =\left[\begin{array}{llll}
                    1 & 0 & 0 & 0 \\
                    0 & 1 & 0 & 0 \\
                    0 & 0 & 1 & 0 \\
                    0 & 0 & 0 & 1
                \end{array}\right]
        \end{aligned}
    $$

    For this to occur, $\mathbf{Q}^{\mathrm{T}}=\mathbf{Q}^{-1}$.

    \section{四元数代数}
    Ordered pairs provide a simple notation for representing quaternions, and allow us to represent the real unit 1 as $[1, \mathbf{0}]$, and the imaginaries $i, j, k$ as $[0, \mathbf{i}],[0, \mathbf{j}],[0, \mathbf{k}]$ respectively. A quaternion then becomes a linear combination of these elements with associated real coefficients. Under such conditions, the elements form the basis for an algebra over the field of reals.

    Furthermore, because quaternion algebra supports division, and obeys the normal axioms of algebra, except that multiplication is non-commutative, it is called a division algebra. The German mathematician Ferdinand Georg Frobenius proved that only three such real associative division algebras exist: real numbers, complex numbers and quaternions [8].

    The Cayley numbers $\mathbb{O}$, constitute a real division algebra, but the Cayley numbers are 8-dimensional and are not associative, i.e. $a(b c) \neq(a b) c$ for all $a, b, c \in \mathbb{O}$.

\section{总结}
四元数与复数非常相似,除了它们有三个虚数项,而不是一个。因此,它们继承了一些与复数有关的性质,如范数、复共轭、单位范数和逆。它们也可以被加、减、乘、除。然而,与复数不同的是,它们在相乘时是反交换的。

\subsection{定义总结}
\begin{tcolorbox}[breakable, enhanced,title = {四元数}]
    $$
        \begin{aligned}
             & q_{a}=\left[s_{a}, \mathbf{a}\right]=\left[s_{a}, x_{a} \mathbf{i}+y_{a} \mathbf{j}+z_{a} \mathbf{k}\right]   \\
             & q_{b}=\left[s_{b}, \mathbf{b}\right]=\left[s_{b}, x_{b} \mathbf{i}+y_{b} \mathbf{j}+z_{b} \mathbf{k}\right] .
        \end{aligned}
    $$
\end{tcolorbox}


\begin{tcolorbox}[breakable, enhanced,title = {加减法}]
    $$
        q_{a} \pm q_{b}=\left[s_{a} \pm s_{b}, \mathbf{a} \pm \mathbf{b}\right] .
    $$
\end{tcolorbox}

\begin{tcolorbox}[breakable, enhanced,title = {乘积}]
    $$
        \begin{aligned}
            q_{a} q_{b} & =\left[s_{a}, \mathbf{a}\right]\left[s_{b}, \mathbf{b}\right]                                                         \\
                        & =\left[s_{a} s_{b}-\mathbf{a} \cdot \mathbf{b}, s_{a} \mathbf{b}+s_{b} \mathbf{a}+\mathbf{a} \times \mathbf{b}\right] \\
                        & =\left[\begin{array}{rrr}
                    s_{a}-x_{a}-y_{a}-z_{a}                 \\
                    x_{a}       & s_{a}-z_{a} & y_{a}       \\
                    y_{a}       & z_{a}       & s_{a}-x_{a} \\
                    z_{a}-y_{a} & x_{a}       & s_{a}
                \end{array}\right]\left[\begin{array}{c}
                    s_{b} \\
                    x_{b} \\
                    y_{b} \\
                    z_{b}
                \end{array}\right] .
        \end{aligned}
    $$
\end{tcolorbox}

\begin{tcolorbox}[breakable, enhanced,title = {平方}]
    $$
        \begin{aligned}
            \mathbf{v} & =x \mathbf{i}+y \mathbf{j}+z \mathbf{k}                                            \\
            q^{2}      & =[s, \mathbf{v}][s, \mathbf{v}]                                                    \\
                       & =\left[s^{2}-x^{2}-y^{2}-z^{2}, 2 s(x \mathbf{i}+y \mathbf{j}+z \mathbf{k})\right]
        \end{aligned}
    $$
\end{tcolorbox}

\begin{tcolorbox}[breakable, enhanced,title = {纯四元数}]
    $$
        \begin{aligned}
            \mathbf{v} & =x \mathbf{i}+y \mathbf{j}+z \mathbf{k}                   \\
            q^{2}      & =[0, \mathbf{v}][0, \mathbf{v}]                           \\
                       & =\left[-\left(x^{2}+y^{2}+z^{2}\right), \mathbf{0}\right]
        \end{aligned}
    $$
\end{tcolorbox}

\begin{tcolorbox}[breakable, enhanced,title = {范数}]
    $$
        \begin{aligned}
            \mathbf{v} & =\lambda \hat{\mathbf{v}}      \\
            q          & =[s, \lambda \hat{\mathbf{v}}] \\
            |q|        & =\sqrt{s^{2}+\lambda^{2}}
        \end{aligned}
    $$
\end{tcolorbox}

\begin{tcolorbox}[breakable, enhanced,title = {单位范数}]
    $$
        |q|=\sqrt{s^{2}+\lambda^{2}}=1 .
    $$
\end{tcolorbox}

\begin{tcolorbox}[breakable, enhanced,title = {共轭}]
    $$
        \begin{aligned}
            q^{*}                        & =[s,-\mathbf{v}]       \\
            \left(q_{a} q_{b}\right)^{*} & =q_{b}^{*} q_{a}^{*} .
        \end{aligned}
    $$
\end{tcolorbox}

\begin{tcolorbox}[breakable, enhanced,title = {逆}]
    $$
        \begin{aligned}
            q^{-1}                        & =\frac{q^{*}}{|q|^{2}} \\
            \left(q_{a} q_{b}\right)^{-1} & =q_{b}^{-1} q_{a}^{-1}
        \end{aligned}
    $$
\end{tcolorbox}

\section{样例}
下面是一些运用了上述想法的进一步实例。在某些情况下,还包括一个测试来确认结果。

\begin{myexample}{四元数的加法和减法}{theoexample}
对下列四元数进行加减运算:
$$
    \begin{aligned}
        q_{a}       & =[2,-2 \mathbf{i}+3 \mathbf{j}-4 \mathbf{k}]  \\
        q_{b}       & =[1,-2 \mathbf{i}+5 \mathbf{j}-6 \mathbf{k}]  \\
        q_{a}+q_{b} & =[3,-4 \mathbf{i}+8 \mathbf{j}-10 \mathbf{k}] \\
        q_{a}-q_{b} & =[1,0 \mathbf{i}-2 \mathbf{j}+2 \mathbf{k}] .
    \end{aligned}
$$
\end{myexample}

\begin{myexample}{四元数范数}{theoexample}
求下列四元数的范数:
$$
    \begin{aligned}
        q_{a}              & =[2,-2 \mathbf{i}+3 \mathbf{j}-4 \mathbf{k}]      \\
        q_{b}              & =[1,-2 \mathbf{i}+5 \mathbf{j}-6 \mathbf{k}]      \\
        \left|q_{a}\right| & =\sqrt{2^{2}+(-2)^{2}+3^{2}+(-4)^{2}}=\sqrt{33}   \\
        \left|q_{b}\right| & =\sqrt{1^{2}+(-2)^{2}+5^{2}+(-6)^{2}}=\sqrt{66} .
    \end{aligned}
$$
\end{myexample}

\begin{myexample}{单位范数四元数}{theoexample}
将这些四元数转换为单位范数形式:
$$
    \begin{aligned}
        q_{a}              & =[2,-2 \mathbf{i}+3 \mathbf{j}-4 \mathbf{k}]                    \\
        q_{b}              & =[1,-2 \mathbf{i}+5 \mathbf{j}-6 \mathbf{k}]                    \\
        \left|q_{a}\right| & =\sqrt{33}                                                      \\
        \left|q_{b}\right| & =\sqrt{66}                                                      \\
        q_{a}^{\prime}     & =\frac{1}{\sqrt{33}}[2,-2 \mathbf{i}+3 \mathbf{j}-4 \mathbf{k}] \\
        q_{b}^{\prime}     & =\frac{1}{\sqrt{66}}[1,-2 \mathbf{i}+5 \mathbf{j}-6 \mathbf{k}]
    \end{aligned}
$$
\end{myexample}

\begin{myexample}{四元数乘积}{theoexample}
计算下列四元数的乘积和反向乘积。
$$
    \begin{aligned}
        q_{a}       & =[2,-2 \mathbf{i}+3 \mathbf{j}-4 \mathbf{k}]                                            \\
        q_{b}       & =[1,-2 \mathbf{i}+5 \mathbf{j}-6 \mathbf{k}]                                            \\
        q_{a} q_{b} & =[2,-2 \mathbf{i}+3 \mathbf{j}-4 \mathbf{k}][1,-2 \mathbf{i}+5 \mathbf{j}-6 \mathbf{k}] \\
                    & =[2 \times 1-((-2) \times(-2)+3 \times 5+(-4) \times(-6)),                              \\
                    & +2(-2 \mathbf{i}+5 \mathbf{j}-6 \mathbf{k})+1(-2 \mathbf{i}+3 \mathbf{j}-4 \mathbf{k})  \\
                    & +(3 \times(-6)-(-4) \times 5) \mathbf{i}-((-2) \times(-6)-(-4) \times(-2)) \mathbf{j}   \\
                    & +((-2) \times 5-3 \times(-2)) \mathbf{k}]                                               \\
                    & =[-41,-6 \mathbf{i}+13 \mathbf{j}-16 \mathbf{k}+2 \mathbf{i}-4 \mathbf{j}-4 \mathbf{k}] \\
                    & =[-41,-4 \mathbf{i}+9 \mathbf{j}-20 \mathbf{k}] .                                       \\
        q_{b} q_{a} & =[1,-2 \mathbf{i}+5 \mathbf{j}-6 \mathbf{k}][2-2 \mathbf{i}+3 \mathbf{j}-4 \mathbf{k}]  \\
                    & =[1 \times 2-((-2) \times(-2)+5 \times 3+(-6) \times(-4)),                              \\
                    & +1(-2 \mathbf{i}+3 \mathbf{j}-4 \mathbf{k})+2(-2 \mathbf{i}+5 \mathbf{j}-6 \mathbf{k})  \\
                    & +(5 \times(-4)-(-6) \times 3) \mathbf{i}-((-2) \times(-4)-(-6) \times(-2)) \mathbf{j}   \\
                    & +((-2) \times 3-5 \times(-2)) \mathbf{k}]                                               \\
                    & =[-41,-6 \mathbf{i}+13 \mathbf{j}-16 \mathbf{k}-2 \mathbf{i}+4 \mathbf{j}+4 \mathbf{k}] \\
                    & =[-41,-8 \mathbf{i}+17 \mathbf{j}-12 \mathbf{k}] .
    \end{aligned}
$$
注意:在这个计算中唯一改变的是轴向量叉乘的符号。
\end{myexample}

\begin{myexample}{四元数平方}{theoexample}
计算这个四元数的平方:
$$
    \begin{aligned}
        q     & =[2,-2 \mathbf{i}+3 \mathbf{j}-4 \mathbf{k}]                                            \\
        q^{2} & =[2,-2 \mathbf{i}+3 \mathbf{j}-4 \mathbf{k}][2,-2 \mathbf{i}+3 \mathbf{j}-4 \mathbf{k}] \\
              & =[2 \times 2-((-2) \times(-2)+3 \times 3+(-4) \times(-4))                               \\
              & +2 \times 2(-2 \mathbf{i}+3 \mathbf{j}-4 \mathbf{k})]                                   \\
              & =[-25,-8 \mathbf{i}+12 \mathbf{j}-16 \mathbf{k}]
    \end{aligned}
$$
\end{myexample}

\begin{myexample}{四元数的逆}{theoexample}
计算这个四元数的逆:
$$
    \begin{aligned}
        q       & =[2,-2 \mathbf{i}+3 \mathbf{j}-4 \mathbf{k}]              \\
        q^{*}   & =[2,2 \mathbf{i}-3 \mathbf{j}+4 \mathbf{k}]               \\
        |q|^{2} & =2^{2}+(-2)^{2}+3^{2}+(-4)^{2}=33                         \\
        q^{-1}  & =\frac{1}{33}[2,2 \mathbf{i}-3 \mathbf{j}+4 \mathbf{k}] .
    \end{aligned}
$$
\end{myexample}



\begin{thebibliography}{99}
    \bibitem{bib6-1} Hamilton, W.R.: On quaternions: or a new system of imaginaries in algebra. Phil. Mag. 3rd ser. $25(1844)$

    \bibitem{bib6-2} Hamilton, W.R.: Lectures on Quaternions. Hodges and Smith, Dublin (1853)

    \bibitem{bib6-3} Hamilton, W.R.: Elements of Quaternions (Jolly, C.J. (ed.) 2 vols.), 2nd edn. Green \& Co., London, Longmans (1899-1901)

    \bibitem{bib6-4} Tait, P.G.: An Elementary Treatise on Quaternions. Cambridge University Press, Cambridge $(1867)$

    \bibitem{bib6-5} Gauss, C.F.: Mutation des Raumes In: Carl Friedrich Gauss Werke, Achter Band, pp. 357-361, König. Gesell. Wissen. Göttingen, 1900 (1819)

    \bibitem{bib6-6} Wilson, E.B.: Vector Analysis. Yale University Press, New Haven (1901)

    \bibitem{bib6-7} Feynman, R.P.: Symmetry and physical laws. In: Feynman Lectures in Physics, vol. 1

    \bibitem{bib6-8} Altmann, S.L.: Rotations, : Quaternions and Double Groups, p. 16. Dover, New York (2005). ISBN-13: 978-0-486-44518-2

\end{thebibliography}
