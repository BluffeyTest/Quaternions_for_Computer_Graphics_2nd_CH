\chap{数集与代数}

\section{介绍}
在这一章中,我们回顾了数集的一些基本概念,以及如何用算术和代数方法来处理它们。我们简要地看一下表达式和方程,以及用于它们的构造和求值的规则。反过来,这些揭示了用所谓的复数来扩展日常数字的必要性。本章的第二部分用于定义群、环和域。

\section{数集}
\subsection{自然数}
自然数是整数$1,2,3,4$,等等,根据定义(DIN $5473)$,自然数和零$\{0,1,2,3,4,\ldots\}$的集合由符号$\mathbb{N}$表示,我们使用:
$$
\mathbb{N}=\{0,1,2,3,4, \ldots\}
$$
语句
$$
k \in \mathbb{N}
$$
表示$k$属于集合$\mathbb{N}$,其中$\in$表示属于,或者换句话说,$k$是一个自然数。我们在本书中使用这种符号,以确保对所使用的数字数量的类型没有混淆。

$\mathbb{N}^{*}$用于表示集合$\{1,2,3,4,\ldots\}$,没有零。

\subsection{实数}
十进制数构成了由 $\mathbb{R}$ 标识的实数集。这些数都是有符号的,可以组织成一条线,在$\pm $无穷之间延伸,并包括零。无穷大的概念很奇怪,德国数学家格奥尔格-康托尔(Georg Cantor,1845-1918 年)对其进行了研究。康托尔还发明了集合论,并证明实数比自然数更多。幸运的是,我们不需要在本书中使用这些概念。

使用实数集 $\mathbb{R}$ 表示维度,其中 $\mathbb{R}^{2}$ 表示二维,$\mathbb{R}^{3}$ 表示三维,$\mathbb{R}^{n}$ 表示 $n$ 维。

\subsection{整数}
整数集合 $\mathbb{Z}$ 包含自然数及其负数:
$$
\mathbb{Z}=\{\ldots,-3,-2,-1,0,1,2,3, \ldots\}
$$

$\mathbb{Z}$ 代表 Zahlen---德语中的 "数字"。

\subsection{有理数}
有理数的集合是 $\mathbb{Q}$ ,包含形式如下的数:
$$
\frac{a}{b}, \quad a, b \in \mathbb{Z}, \quad b \neq 0
$$

\section{算术运算}
我们使用算术运算加法、减法、乘法和除法来处理数字,其结果是否封闭或未定义,取决于底层集合。例如,当我们把两个自然数相加时,结果总是另一个自然数,因此,这个运算是封闭的:
$$
3+4=7
$$
但是,当我们用两个自然数相减时,结果不一定是自然数。例如,虽然
$$
6-2=4
$$
是封闭运算符,但是
$$
2-6=-4
$$
不是封闭的,因为-4 不属于自然数集。

两个自然数的乘积总是一个封闭运算,但是除法会带来一些问题。首先,偶数自然数除以 2 是封闭运算:
$$
\frac{16}{2}=8
$$
而奇数自然数除以偶数自然数得到的是十进制数:
$$
\frac{7}{2}=3.5
$$
并不闭合,因为 3.5 不属于自然数集。在集合语言中,这可以写成
$$
3.5 \notin \mathbb{N}
$$

其中 $\notin$ 表示不属于。

任何数与零相乘的结果都是零---这是一个封闭运算;然而,任何数与零相除的结果都是不确定的,必须排除在外。

实数不存在任何与自然数相关的问题,加、乘、除运算都是封闭的:
$$
\begin{aligned}
a+b & =c, \quad a, b, c \in \mathbb{R} \\
a b & =c, \quad a, b, c \in \mathbb{R} \\
\frac{a}{b} & =c, \quad a, b, c \in \mathbb{R} \text { and } b \neq 0 .
\end{aligned}
$$

\section{公理}
当我们构建代数表达式时,我们会运用被称为公理的特定法则。对于加法和乘法,我们知道数字的分组对最终结果没有影响:例如,$2+(4+6)=(2+4)+6$ 和 $2 \times(3 \times 4)=(2\times  3)\times 4$。这就是联立公理,其表达式为:

$$
\begin{aligned}
a+(b+c) & =(a+b)+c \\
a(b c) & =(a b) c .
\end{aligned}
$$
我们还知道,在进行加法或乘法运算时,顺序对最终结果没有影响:例如,$2+6=6+2$,$2 \times 6=6 \times 2$。这就是交换公理,其表达式为
$$
\begin{aligned}
a+b & =b+a \\
a b & =b a .
\end{aligned}
$$
代数表达式包含了一个实数和一串实数的乘积,它们服从分配律:

$$
\begin{aligned}
a(b+c) & =a b+a c \\
(a+b)(c+d) & =a c+a d+b c+b d .
\end{aligned}
$$

The reason why we have reviewed these axioms is that they should not be regarded as carved in mathematical stone, and apply to everything that is invented. For when we come to quaternions we discover that they do not obey the commutative axiom, which is not that strange. If you have used matrices you will know that matrix multiplication is also non-commutative, but is associative.

\section{表达式}
Using the above axioms we are able to construct all sorts of expressions such as:

$$
\begin{gathered}
a(2+c)-\frac{d}{e}+a-10 \\
\frac{g}{a c-b d}+\frac{h}{d e-f g} .
\end{gathered}
$$

We also employ notation for raising a quantity to some power such as $n^{2}$. This notation introduces another set of observations:

$$
\begin{aligned}
a^{n} a^{m} & =a^{n+m} \\
\frac{a^{n}}{a^{m}} & =a^{n-m} \\
\left(a^{n}\right)^{m} & =a^{n m} \\
\frac{a^{n}}{a^{n}} & =a^{0}=1 \\
\frac{1}{a^{n}} & =a^{-1} \\
a^{\frac{1}{n}} & =\sqrt[n]{a} .
\end{aligned}
$$

Next, we have to include all sorts of functions such as square-roots, sines and cosines, which may seem rather innocent. But we must be wary of them. For example, $\sqrt{16}= \pm 4$, whereas, there is no natural or real number solution for $\sqrt{-16}$. Consequently, the expression $\sqrt{a}$ has no real roots if $a<0$. Similarly, when working with trigonometric functions such as sine and cosine, we must remember that these take on a range of values between -1 and +1 , including 0 , which means that if they are employed as a denominator, the result could be undefined. For example, this expression is undefined if $\sin \alpha=0$

$$
\frac{a}{\sin \alpha}
$$

\section{等式}
Next, we come to equations where we assign the value of an expression to a variable. In most situations the assignment is straightforward and leads to a real result such as

$$
x^{2}-16=0
$$

where $x= \pm 4$. But what is interesting is that just by reversing the sign to

$$
x^{2}+16=0
$$

we create an equation for which there is no real solution. However, there is a complex solution, which is the subject of Chap. 3.

\section{有序对}
An ordered pair or couple $(a, b)$ is an object having two entries, coordinates or projections, where the first or left entry, is distinguishable from the second or right entry. For example, $(a, b)$ is distinguishable from $(b, a)$ unless $a=b$. Perhaps the best example of an ordered pair is $(x, y)$ that represents a point in $\mathbb{R}^{2}$, where the order of the entries is always the $x$-coordinate followed by the $y$-coordinate.

Ordered pairs and ordered triples are widely used in computer graphics to represent points in $\mathbb{R}^{2}:(x, y)$, points in $\mathbb{R}^{3}:(x, y, z)$, and colour values such as $(\mathrm{r}, \mathrm{g}, \mathrm{b})$ and (h, s, v). In these examples, the fields are all real values. There is nothing to stop us from developing an algebra using ordered pairs that behaves like another algebra, and we will do this for complex numbers in Chap. 3 and quaternions in Chap. 6. For the moment, let's explore some ways ordered pairs can be manipulated.

Say we choose to describe a generic ordered pair as

$$
a=\left(a_{1}, a_{2}\right), \quad a_{1}, a_{2} \in \mathbb{R}
$$

We will define the addition of two such objects as

$$
\begin{aligned}
a & =\left(a_{1}, a_{2}\right) \\
b & =\left(b_{1}, b_{2}\right) \\
a+b & =\left(a_{1}+b_{1}, a_{2}+b_{2}\right) .
\end{aligned}
$$

For example:

$$
\begin{aligned}
a & =(2,3) \\
b & =(4,5) \\
a+b & =(6,8) .
\end{aligned}
$$

We will define the product as

$$
a b=\left(a_{1} b_{1}, a_{2} b_{2}\right)
$$

which, using the above values, results in

$$
a b=(8,15)
$$

Remember, we are in charge, and we define the rules.

Another rule will control how an ordered pair responds to scalar multiplication. For example:

$$
\begin{aligned}
\lambda\left(a_{1}, a_{2}\right) & =\left(\lambda a_{1}, \lambda a_{2}\right), \quad \lambda \in \mathbb{R} \\
3(2,3) & =(6,9) .
\end{aligned}
$$

With the above rules, we are in a position to write

$$
\begin{aligned}
\left(a_{1}, a_{2}\right) & =\left(a_{1}, 0\right)+\left(0, a_{2}\right) \\
& =a_{1}(1,0)+a_{2}(0,1)
\end{aligned}
$$

and if we square these unit ordered pairs $(1,0)$ and $(0,1)$ using the product rule, we obtain

$$
\begin{aligned}
& (1,0)^{2}=(1,0) \\
& (0,1)^{2}=(0,1)
\end{aligned}
$$

which suggests that they behave like real numbers, and is not unexpected.

This does not appear to be very useful, but wait and see what happens in the context of complex numbers and quaternions.

\section{群,环和域}
Mathematicians employ a bewildering range of names to identify their inventions, which seemingly, appear on a daily basis. Even the name 'quaternion' is not original, and appears throughout history often in the context of 'a quaternion of soldiers':

The Romans detached a quaternion or four men for a night guard ... [1].

Without becoming too formal, let's explore some more mathematical structures that are relevant to the ideas contained in this book.

\subsection{群}
We have already covered the idea of a set, and what it means to belong to a set. We have also discovered that when we apply certain arithmetic operations to members of a set we can secure closure, non-closure, or the result is undefined.

When combining sets with arithmetic operations, it is convenient to create another entity: a group, which is a set, together with the axioms describing how elements of the set are combined. The set might contain numbers, matrices, vectors, quaternions, polynomials, etc., and are represented below as $a, b$ and $c$.

The axioms employ the ' $\circ$ ' symbol to represent any binary operation such as ,,$+- \times$. And a group is formed from a set and a binary operation. For example, we may wish to form a group of integers under addition: $(\mathbb{Z},+)$, or we may wish to examine whether quaternions form a group under the operation of multiplication: $(\mathbb{H}, \times)$.

To be a group, all the following axioms must hold for the set $S$. In particular, there must be a special identity element $e \in S$, and for each $a \in S$ there must exist an inverse element $a^{-1} \in S$, so that the following axioms are satisfied:

$$
\begin{array}{rlr}
\text { Closure: } & a \circ b \in S, & a, b \in S . \\
\text { Associativity: } & (a \circ b) \circ c=a \circ(b \circ c), & a, b, c \in S . \\
\text { Identity: } & a \circ e=e \circ a=a, & a, e \in S . \\
\text { Inverse: } & a \circ a^{-1}=a^{-1} \circ a=e, & a, a^{-1}, e \in S .
\end{array}
$$

We describe a group as $(S, \circ)$, where $S$ is the set and ' $\circ$ ' the operation. For instance, $(\mathbb{Z},+)$ is the group of integers under the operation of addition, and $(\mathbb{R}, \times)$ is the group of reals under the operation of multiplication.

Let's bring these axioms to life with three examples. $(\mathbb{Z},+)$ : The integers $\mathbb{Z}$ form a group under the operation of addition:

$$
\begin{aligned}
\text { Closure: } & -23+24=1 \\
\text { Associativity: } & (2+3)+4=2+(3+4)=9 \\
\text { Identity: } & 2+0=0+2=2 \\
\text { Inverse: } & 2+(-2)=(-2)+2=0
\end{aligned}
$$

$(\mathbb{Z}, \times)$ : The integers $\mathbb{Z}$ do not form a group under multiplication:

Closure: $-2 \times 4=-8$.

Associativity: $(2 \times 3) \times 4=2 \times(3 \times 4)=24$.

Identity: $2 \times 1=1 \times 2=2$.

Inverse: $2^{-1}=0.5 \quad(0.5 \notin \mathbb{Z})$.

Also, the integer 0 has no inverse.

$(\mathbb{Q}, \times)$ : The group of non-zero rational numbers form a group under multiplication:

$$
\begin{aligned}
\text { Closure: } & \frac{2}{5} \times \frac{2}{3}=\frac{4}{15} . \\
\text { Associativity: } & \left(\frac{2}{5} \times \frac{2}{3}\right) \times \frac{1}{2}=\frac{2}{5} \times\left(\frac{2}{3} \times \frac{1}{2}\right)=\frac{2}{15} . \\
\text { Identity: } & \frac{2}{3} \times \frac{1}{1}=\frac{1}{1} \times \frac{2}{3}=\frac{2}{3} . \\
\text { Inverse: } & \frac{2}{3} \times \frac{3}{2}=\frac{1}{1} \quad\left(\text { where } \frac{3}{2}=\left(\frac{2}{3}\right)^{-1}\right) .
\end{aligned}
$$

\subsection{阿贝尔群}
Lastly, an abelian group, named after the Norwegian mathematician Neils Henrik Abel (1802-1829), is a group where the order of elements does not influence the result. i.e. the group is commutative. Thus there are five axioms: closure, associativity, identity element, inverse element, and commutativity:

$$
\text { Commutativity: } a \circ b=b \circ a, \quad a, b \in S \text {. }
$$

For example, the set of integers forms an abelian group under ordinary addition $(\mathbb{Z},+)$. However, because 3 -D rotations do not generally commute, the set of all rotations in 3-D space forms a non-commutative group.

\subsection{环}
A ring is an extended group, where we have a set of objects which can be added and multiplied together, subject to some precise axioms. There are rings of real numbers, complex numbers, integers, matrices, equations, polynomials, etc. A ring is formally defined as a system where $(S,+)$ and $(S, \times)$ are abelian groups and the distributive axioms:

Additive associativity: $a+(b+c)=(a+b)+c, \quad a, b, c \in S$.

Multiplicative associativity: $a \times(b \times c)=(a \times b) \times c, \quad a, b, c \in S$.

Distributivity: $a \times(b+c)=(a \times b)+(a \times c), \quad$ and $(a+b) \times c=(a \times c)+(b \times c), \quad a, b, c \in S$.

For example, we already know that the integers $\mathbb{Z}$ form a group under the operation of addition, but they also form a ring, as the set satisfies the above axioms:

$$
\begin{aligned}
& 2 \times(3 \times 4)=(2 \times 3) \times 4 \\
& 2 \times(3+4)=(2 \times 3)+(2 \times 4) \\
& (2+3) \times 4=(2 \times 4)+(3 \times 4) .
\end{aligned}
$$

\subsection{域}
Although rings support addition and multiplication, they do not necessarily support division. However, as division is such an important arithmetic operation, the field was created to support it, with one proviso: division by zero is not permitted. Thus we have fields of real numbers $\mathbb{R}$, rational numbers $\mathbb{Q}$, and as we shall see, the complex numbers $\mathbb{C}$. However, we will discover that quaternions do not form a field, but they do form what is called a division ring.

It follows that every field is a ring, but not every ring is a field.

\subsection{除法环}
A division ring or division algebra, is a ring in which every element has an inverse element, with the proviso that the element is non-zero. The algebra also supports non-commutative multiplication. Here is a formal description of the division ring $(S,+, \times)$ : Additive associativity: $(a+b)+c=a+(b+c)$,

Additive commutativity: $a+b=b+a$,

Additive identity $0: 0+a=a+0$,

Additive inverse: $a+(-a)=(-a)+a=0$,

Multiplicative associativity: $(a \times b) \times c=a \times(b \times c)$,

Multiplicative identity $1: 1 \times a=a \times 1$,

Multiplicative inverse: $a \times a^{-1}=a^{-1} \times a=1$,

Distributivity: $a \times(b+c)=(a \times b)+(a \times c)$,

$$
(b+c) \times a=(b \times a)+(c \times a), \quad a, b, c \in S .
$$

In 1877 the German mathematician Ferdinand Georg Frobenius (1849-1917), proved that there are only three associative division algebras: real numbers $\mathbb{R}$, complex numbers $\mathbb{C}$, and quaternions $\mathbb{H}$.

\section{总结}
The objective of this chapter was to remind you of the axiomatic systems underlying algebra, and how the results of arithmetic operations can be open, closed, or undefined. Perhaps some of the ideas of ordered pairs, sets, groups, fields and rings are new, and they have been included as this notation is often used in association with quaternions.

All of these ideas emerge again when we consider the algebra of complex numbers and later on, quaternions.

\subsection{定义总结}
\subsubsection*{有序对}
An object with two distinguishable components: $(a, b)$ such that $(a, b) \neq(b, a)$ unless $a=b$.

\subsubsection*{集合}
Definition: A set is a collection of objects.

Notation: $k \in \mathbb{Z}$ means $k$ belongs to the set $\mathbb{Z}$.

$\mathbb{C}:$ Set of complex numbers

$\mathbb{H}:$ Set of quaternions

$\mathbb{N}:$ Set of natural numbers

$\mathbb{Q}$ : Set of rational numbers $\mathbb{R}:$ Set of real numbers

$\mathbb{Z}:$ Set of integers.

\subsubsection*{群}
Definition: A group $(S, \circ)$ is a set $S$ and a binary operation ' $\circ$ ' and the axioms defining closure, associativity, an identity element, and an inverse element.

$$
\begin{array}{rlr}
\text { Closure: } & a \circ b \in S, & a, b \in S . \\
\text { Associativity: } & (a \circ b) \circ c=a \circ(b \circ c), & a, b, c \in S . \\
\text { Identity: } & a \circ e=e \circ a=a, & a, e \in S . \\
\text { Inverse: } & a \circ a^{-1}=a^{-1} \circ a=e, & a, a^{-1}, e \in S .
\end{array}
$$

\subsubsection*{环}
Definition: A ring is a group whose elements can be added/subtracted and multiplied, using some precise axioms:

Additive associativity: $a+(b+c)=(a+b)+c, \quad a, b, c \in S$.

Multiplicative associativity: $a \times(b \times c)=(a \times b) \times c, \quad a, b, c \in S$.

Distributivity: $a \times(b+c)=(a \times b)+(a \times c), \quad$ and

$$
(a+b) \times c=(a \times c)+(b \times c), \quad a, b, c \in S .
$$

\subsubsection*{域}
Definition: A field is a ring that supports division.

\subsubsection*{除法环}
Every element of a division ring has an inverse element, with the proviso that the element is non-zero. The algebra also supports non-commutative multiplication.

Additive associativity: $(a+b)+c=a+(b+c)$,

$$
\begin{array}{r}
a, b, c \in S . \\
a, b \in S . \\
a, 0 \in S . \\
a,-a \in S . \\
a, b, c \in S . \\
a, 1 \in S .
\end{array}
$$

Multiplicative associativity: $(a \times b) \times c=a \times(b \times c)$,

Multiplicative inverse: $a \times a^{-1}=a^{-1} \times a=1$,

$a, a^{-1} \in S, a \neq 0$

Distributivity: $a \times(b+c)=(a \times b)+(a \times c), \quad$ and $(b+c) \times a=(b \times a)+(c \times a), \quad a, b, c \in S$.


\begin{thebibliography}{99}
    \bibitem{bib2-1} Robinson, E.: Greek and English Lexicon of the New Testament (1825). \href{http://books.google}{http://books.google}. \href{http://co.uk}{co.uk}
\end{thebibliography}