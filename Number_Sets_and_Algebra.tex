\chap{数集与代数}

\section{介绍}
在这一章中,我们回顾了数集的一些基本概念,以及如何用算术和代数方法来处理它们。我们简要地看一下表达式和方程,以及用于它们的构造和求值的规则。反过来,这些揭示了用所谓的复数来扩展日常数字的必要性。本章的第二部分用于定义群、环和域。

\section{数集}
\subsection{自然数}
自然数是整数$1,2,3,4$,等等,根据定义(DIN $5473)$,自然数和零$\{0,1,2,3,4,\ldots\}$的集合由符号$\mathbb{N}$表示,我们使用:
$$
    \mathbb{N}=\{0,1,2,3,4, \ldots\}
$$
语句
$$
    k \in \mathbb{N}
$$
表示$k$属于集合$\mathbb{N}$,其中$\in$表示属于,或者换句话说,$k$是一个自然数。我们在本书中使用这种符号,以确保对所使用的数字数量的类型没有混淆。

$\mathbb{N}^{*}$用于表示集合$\{1,2,3,4,\ldots\}$,没有零。

\subsection{实数}
十进制数构成了由 $\mathbb{R}$ 标识的实数集。这些数都是有符号的,可以组织成一条线,在$\pm $无穷之间延伸,并包括零。无穷大的概念很奇怪,德国数学家格奥尔格-康托尔(Georg Cantor,1845-1918 年)对其进行了研究。康托尔还发明了集合论,并证明实数比自然数更多。幸运的是,我们不需要在本书中使用这些概念。

使用实数集 $\mathbb{R}$ 表示维度,其中 $\mathbb{R}^{2}$ 表示二维,$\mathbb{R}^{3}$ 表示三维,$\mathbb{R}^{n}$ 表示 $n$ 维。

\subsection{整数}
整数集合 $\mathbb{Z}$ 包含自然数及其负数:
$$
    \mathbb{Z}=\{\ldots,-3,-2,-1,0,1,2,3, \ldots\}
$$

$\mathbb{Z}$ 代表 Zahlen---德语中的 "数字"。

\subsection{有理数}
有理数的集合是 $\mathbb{Q}$ ,包含形式如下的数:
$$
    \frac{a}{b}, \quad a, b \in \mathbb{Z}, \quad b \neq 0
$$

\section{算术运算}
我们使用算术运算加法、减法、乘法和除法来处理数字,其结果是否封闭或未定义,取决于底层集合。例如,当我们把两个自然数相加时,结果总是另一个自然数,因此,这个运算是封闭的:
$$
    3+4=7
$$
但是,当我们用两个自然数相减时,结果不一定是自然数。例如,虽然
$$
    6-2=4
$$
是封闭运算符,但是
$$
    2-6=-4
$$
不是封闭的,因为-4 不属于自然数集。

两个自然数的乘积总是一个封闭运算,但是除法会带来一些问题。首先,偶数自然数除以 2 是封闭运算:
$$
    \frac{16}{2}=8
$$
而奇数自然数除以偶数自然数得到的是十进制数:
$$
    \frac{7}{2}=3.5
$$
并不闭合,因为 3.5 不属于自然数集。在集合语言中,这可以写成
$$
    3.5 \notin \mathbb{N}
$$

其中 $\notin$ 表示不属于。

任何数与零相乘的结果都是零---这是一个封闭运算;然而,任何数与零相除的结果都是不确定的,必须排除在外。

实数不存在任何与自然数相关的问题,加、乘、除运算都是封闭的:
$$
    \begin{aligned}
        a+b         & =c, \quad a, b, c \in \mathbb{R}                          \\
        a b         & =c, \quad a, b, c \in \mathbb{R}                          \\
        \frac{a}{b} & =c, \quad a, b, c \in \mathbb{R} \text { and } b \neq 0 .
    \end{aligned}
$$

\section{公理}
当我们构建代数表达式时,我们会运用被称为公理的特定法则。对于加法和乘法,我们知道数字的分组对最终结果没有影响:例如,$2+(4+6)=(2+4)+6$ 和 $2 \times(3 \times 4)=(2\times  3)\times 4$。这就是联立公理,其表达式为:

$$
    \begin{aligned}
        a+(b+c) & =(a+b)+c   \\
        a(b c)  & =(a b) c .
    \end{aligned}
$$
我们还知道,在进行加法或乘法运算时,顺序对最终结果没有影响:例如,$2+6=6+2$,$2 \times 6=6 \times 2$。这就是交换公理,其表达式为
$$
    \begin{aligned}
        a+b & =b+a   \\
        a b & =b a .
    \end{aligned}
$$
代数表达式包含了一个实数和一串实数的乘积,它们服从分配律:

$$
    \begin{aligned}
        a(b+c)     & =a b+a c           \\
        (a+b)(c+d) & =a c+a d+b c+b d .
    \end{aligned}
$$
我们回顾这些公理的原因是,它们不应该被认为是刻在数学石头上的,并适用于所有发明的东西。因为当我们谈到四元数时,我们发现它们不遵守交换公理,这并不奇怪。如果你用过矩阵,你会知道矩阵乘法也是不符合交换律的,但它是符合结合律的。

\section{表达式}
使用上述公理,我们可以构造各种表达式,例如:
$$
    \begin{gathered}
        a(2+c)-\frac{d}{e}+a-10 \\
        \frac{g}{a c-b d}+\frac{h}{d e-f g} .
    \end{gathered}
$$
我们也使用符号来表示一个量的幂,例如$n^{2}$。这个符号引入了另一组观察结果:

$$
    \begin{aligned}
        a^{n} a^{m}            & =a^{n+m}       \\
        \frac{a^{n}}{a^{m}}    & =a^{n-m}       \\
        \left(a^{n}\right)^{m} & =a^{n m}       \\
        \frac{a^{n}}{a^{n}}    & =a^{0}=1       \\
        \frac{1}{a^{n}}        & =a^{-1}        \\
        a^{\frac{1}{n}}        & =\sqrt[n]{a} .
    \end{aligned}
$$
接下来,我们要包括各种各样的函数,比如平方根,正弦和余弦,这些看起来很简单。但我们必须警惕他们。例如,$\sqrt{16}= \pm 4$,而$\sqrt{-16}$没有自然数或实数解。因此,如果$a<0$,表达式$\sqrt{a}$没有实根。类似地,当处理正弦和余弦等三角函数时,我们必须记住,它们的取值范围在-1到+1之间,包括0,这意味着如果它们被用作分母,结果可能是未定义的。例如,如果 $\sin \alpha=0$,下面这个表达式没有定义,

$$
    \frac{a}{\sin \alpha}
$$

\section{等式}
接下来是等式,我们将表达式的值赋给一个变量。 在大多数情况下,赋值都是简单明了的,并会得到一个真实的结果,例如
$$
    x^{2}-16=0
$$
其中 $x= \pm 4$。但有趣的是,只要将符号反转为
$$
    x^{2}+16=0
$$
我们建立了一个没有实数解的方程。不过,有一个复数解,这就是第 3 章的主题。

\section{有序对}
有序对或对偶 $(a,b)$是具有两个条目、坐标或投影的对象,其中第一个条目或左条目可与第二个条目或右条目区分开来。例如,$(a, b)$ 与$(b, a)$ 是可以区分的,除非 $a=b$。$(x, y)$ 表示 $\mathbb{R}^{2}$ 中的一个点,可能是有序对的最好例子,其中条目的顺序总是先是 $x$ 坐标,然后是 $y$ 坐标。

在计算机图形学中,有序对和有序三元组广泛用于表示$\mathbb{R}^{2}:(x, y)$中的点,$\mathbb{R}^{3}:(x, y, z)$中的点,以及诸如$(\mathrm{r}, \mathrm{g}, \mathrm{b})$和$(h, s, v)$等颜色值。在这些示例中,字段都是实数。没有什么可以阻止我们使用有序对开发一个代数,它的行为就像另一个代数,我们将在第三章对复数和第六章对四元数这样做。现在,让我们探索一些可以操作有序对的方法。

假设我们选择将一般有序对描述为
$$
    a=\left(a_{1}, a_{2}\right), \quad a_{1}, a_{2} \in \mathbb{R}
$$
我们将把两个这样的对象相加定义为
$$
    \begin{aligned}
        a   & =\left(a_{1}, a_{2}\right)               \\
        b   & =\left(b_{1}, b_{2}\right)               \\
        a+b & =\left(a_{1}+b_{1}, a_{2}+b_{2}\right) .
    \end{aligned}
$$
举例:
$$
    \begin{aligned}
        a   & =(2,3)   \\
        b   & =(4,5)   \\
        a+b & =(6,8) .
    \end{aligned}
$$
我们将乘积定义为
$$
    a b=\left(a_{1} b_{1}, a_{2} b_{2}\right)
$$
根据上述数值,得出
$$
    a b=(8,15)
$$

记住,我们说了算,规则由我们定义。

另一条规则将控制有序数对如何响应标量乘法。例如
$$
    \begin{aligned}
        \lambda\left(a_{1}, a_{2}\right) & =\left(\lambda a_{1}, \lambda a_{2}\right), \quad \lambda \in \mathbb{R} \\
        3(2,3)                           & =(6,9) .
    \end{aligned}
$$
根据上述规则,我们可以写出
$$
    \begin{aligned}
        \left(a_{1}, a_{2}\right) & =\left(a_{1}, 0\right)+\left(0, a_{2}\right) \\
                                  & =a_{1}(1,0)+a_{2}(0,1)
    \end{aligned}
$$
如果我们用乘积法则将$(1,0)$ 和$(0,1)$ 这两个单位有序对平方,就可以得到
$$
    \begin{aligned}
         & (1,0)^{2}=(1,0) \\
         & (0,1)^{2}=(0,1)
    \end{aligned}
$$
这表明它们的行为与实数类似,并不出人意料。

这似乎不是很有用,但请拭目以待,看看在复数和四元数的背景下会发生什么。

\section{群,环和域}
数学家使用了一系列令人困惑的名称来标识他们的发明,这些发明似乎每天都在出现。就连 "四元数 "这个名字也不是原创的,在历史上经常出现在 "士兵的四人组 "的语境中:

\begin{CJK}{UTF8}{gkai}
    \begin{quotation}
        罗马人派出四人组(quaternion)或四个人进行夜间守卫...... \cite{bib2-1}.
    \end{quotation}
\end{CJK}

在不太正式的情况下,让我们探索一些与本书中包含的思想相关的更多数学结构。

\subsection{群}
我们已经讨论了集合的概念,以及属于集合的含义。我们还发现,当对集合的成员应用某些算术运算时,可以确保闭合、非闭合或结果未定义。

当将集合与算术运算结合在一起时,可以方便地创建另一个实体:一个群,它是一个集合,以及描述集合元素如何组合的公理。该集合可能包含数字、矩阵、向量、四元数、多项式等,并在下面表示为$a, b$和$c$。

该公理使用' $\circ$ '符号来表示任何二进制运算,例如$+- \times$。一个群是由一个集合和一个二进制运算组成的。例如,我们可能希望在加法运算下形成一个整数群:$(\mathbb{Z},+)$,或者我们可能希望检查四元数是否在乘法运算下形成一个群:$(\mathbb{H}, \times)$。

要成为一个群,以下所有公理必须在集合 $S$ 中成立。 特别是,必须有一个特殊的单位元 $e \in S$ ,而且对于每个 $a \in S$ ,必须存在一个逆元 $a^{-1} \in S$中,这样就满足了以下公理:

$$
    \begin{array}{rlr}
        \text { 闭合律: }       & a \circ b \in S,                        & a, b \in S .         \\
        \text { 结合律: } & (a \circ b) \circ c=a \circ(b \circ c), & a, b, c \in S .      \\
        \text { 单位元: }      & a \circ e=e \circ a=a,                  & a, e \in S .         \\
        \text { 逆元: }       & a \circ a^{-1}=a^{-1} \circ a=e,        & a, a^{-1}, e \in S .
    \end{array}
$$

我们把一个群描述为$(S, \circ)$,其中$S$是集合,$\circ$是运算。例如,$(\mathbb{Z},+)$是加法运算下的整数群,$(\mathbb{R}, \times)$是乘法运算下的实数群。

让我们用三个例子来说明这些公理。$(\mathbb{Z},+)$:整数$\mathbb{Z}$在加法运算下组成一个群:
$$
    \begin{array}{rlr}
        \text { 闭合律: }       & -23+24=1.       \\
        \text { 结合律: } & (2+3)+4=2+(3+4)=9. \\
        \text { 单位元: }      & 2+0=0+2=2.         \\
        \text { 逆元: }       & 2+(-2)=(-2)+2=0.
    \end{array}
$$

$(\mathbb{Z}, \times)$:整数$\mathbb{Z}$在乘法下不构成一个群:
$$
    \begin{array}{rlr}
        \text { 闭合律: }       &-2 \times 4=-8.\\
        \text { 结合律: } &(2 \times 3) \times 4=2 \times(3 \times 4)=24.\\
        \text { 单位元: }      & 2 \times 1=1 \times 2=2.\\
        \text { 逆元: }       &2^{-1}=0.5 \quad(0.5 \notin \mathbb{Z}).
\end{array}
$$
同样,整数0没有逆。

$(\mathbb{Q},\times)$:非零有理数组成的乘法下的群:
$$
    \begin{array}{rlr}
        \text { 闭合律: }       & \frac{2}{5} \times \frac{2}{3}=\frac{4}{15} .                                                                                               \\
        \text { 结合律: } & \left(\frac{2}{5} \times \frac{2}{3}\right) \times \frac{1}{2}=\frac{2}{5} \times\left(\frac{2}{3} \times \frac{1}{2}\right)=\frac{2}{15} . \\
        \text { 单位元: }      & \frac{2}{3} \times \frac{1}{1}=\frac{1}{1} \times \frac{2}{3}=\frac{2}{3} .                                                                 \\
        \text { 逆元: }       & \frac{2}{3} \times \frac{3}{2}=\frac{1}{1} \quad\left(\text { 其中 } \frac{3}{2}=\left(\frac{2}{3}\right)^{-1}\right) .
    \end{array}
$$

\subsection{阿贝尔群}
最后,阿贝尔群以挪威数学家尼尔斯-亨里克-阿贝尔(Neils Henrik Abel,1802-1829 年)的名字命名,是一个元素的顺序不会影响结果的群。因此有五个公理:闭合律、结合律、单位元、逆元和交换律:
$$
    \text { 交换律: } a \circ b=b \circ a, \quad a, b \in S \text {. }
$$

例如,整数集合在普通加法$(\mathbb{Z},+)$下构成一个无性群。然而,由于 3 -D 旋转一般不换向,三维空间中所有旋转的集合形成了一个非交换群。

\subsection{环}
环是一个扩展群,其中有一组可以相加和相乘的对象,但要遵守一些精确的公理。有实数环、复数环、整数环、矩阵环、方程环、多项式环等。环被正式定义为一个系统,其中$(S,+)$和$(S,\times)$是非良群和分配公理:

环是一个扩展的群,在这里我们有一集合对象,这些对象可以相加和相乘,服从一些精确的公理。有实数环、复数环、整数环、矩阵环、方程环、多项式环等。一个环被正式定义为一个系统,其中$(S,+)$和$(S, \times)$是阿贝尔群和分配律公理:
$$
    \begin{array}{rlr}
\text{加法结合律:} &a+(b+c)=(a+b)+c, \quad a, b, c \in S.\\
\text{乘法结合律:} &a \times(b \times c)=(a \times b) \times c, \quad a, b, c \in S.\\
\text{分配律:} & a \times(b+c)=(a \times b)+(a \times c), \quad \text{和} \\
 & (a+b) \times c=(a \times c)+(b \times c), \quad a, b, c \in S.
    \end{array}
$$

例如,我们已经知道整数$\mathbb{Z}$在加法运算下构成一个群,但它们也构成一个环,因为该集合满足上述公理:
$$
    \begin{array}{rlr}
         & 2 \times(3 \times 4)=(2 \times 3) \times 4 \\
         & 2 \times(3+4)=(2 \times 3)+(2 \times 4)    \\
         & (2+3) \times 4=(2 \times 4)+(3 \times 4) .
    \end{array}
$$

\subsection{域}
虽然环支持加法和乘法,但它们不一定支持除法。然而,由于除法是如此重要的算术运算,因此创建了该字段来支持除法,但有一个附带条件:不允许除零。因此,我们有实数$\mathbb{R}$的域,有理数$\mathbb{Q}$的域,以及我们将看到的复数$\mathbb{C}$的域。然而,我们会发现四元数并不构成域,但它们确实构成了所谓的除法环。

由此可见,每个域都是一个环,但并非每个环都是一个域。

\subsection{除法环}
除法环或除法代数是一个环,其中每个元素都有一个逆元素,附带条件是该元素非零。代数也支持非交换乘法。下面是除法环$(S,+, \times)$的正式描述:
$$
    \begin{array}{rlr}
        \text{ 加法结合律: } & (a+b)+c=a+(b+c), & a, b, c \in S .\\
        \text{加法交换律: } & a+b=b+a,          & a, b \in S .\\
        \text{加法单位元 0: } & 0: 0+a=a+0,       & a, 0 \in S .\\
        \text{加法逆元: }  &a+(-a)=(-a)+a=0,    & a, -a \in S .\\
        \text{乘法结合律: } &(a \times b) \times c=a \times(b \times c),& a, b, c \in S .\\
        \text{乘法单位元 1: } &1 \times a=a \times 1,& a, 1 \in S .\\
        \text{乘法逆元: } &a \times a^{-1}=a^{-1} \times a=1,& a,a^{-1}\in S ,a\neq 0.\\
        \text{分配律: } & a\times(b+c)=(a \times b)+(a \times c), &\text{和}\\
        &(b+c) \times a=(b \times a)+(c \times a), & a, b, c \in S .
\end{array}
$$

1877 年,德国数学家费迪南德-乔治-弗罗贝纽斯(Ferdinand Georg Frobenius,1849-1917 年)证明了只有三个联立除法代数:实数 $\mathbb{R}$、复数 $\mathbb{C}$、四元数 $\mathbb{H}$。

\section{总结}
本章的目的是提醒大家代数的公理系统,以及算术运算的结果可以是开放的、闭合的或未定义的。也许有些有序对、集合、群、域和环的概念是新的,由于这些符号经常与四元数联系在一起使用,所以也包括在内。

当我们考虑复数代数以及后来的四元数时,所有这些想法都会再次出现。

\subsection{定义总结}
\subsubsection*{有序对}
具有两个可区分组件的对象:$(a, b)$,使得$(a, b) \neq(b, a)$除非$a=b$。
\subsubsection*{集合}
定义: 集合是对象的集合。

符号: $k \in \mathbb{Z}$ 表示 $k$ 属于集合 $\mathbb{Z}$.


$\mathbb{C}:$  复数集

$\mathbb{H}:$  四元数集

$\mathbb{N}:$  自然数集

$\mathbb{Q}:$  有理数集 

$\mathbb{R}:$  实数集

$\mathbb{Z}:$  整数集.

\subsubsection*{群}
定义: 一个群 $(S, \circ)$ 是一个集合 $S$ 和一个二元运算‘$\circ$’以及定义了闭合律、结合律、单位元和逆元的公理。

$$
    \begin{array}{rlr}
        \text { 闭合律: }       & a \circ b \in S,                        & a, b \in S .         \\
        \text { 结合律: } & (a \circ b) \circ c=a \circ(b \circ c), & a, b, c \in S .      \\
        \text { 单位元: }      & a \circ e=e \circ a=a,                  & a, e \in S .         \\
        \text { 逆元: }       & a \circ a^{-1}=a^{-1} \circ a=e,        & a, a^{-1}, e \in S .
    \end{array}
$$

\subsubsection*{环}
定义:环是一个群,它的元素可以加/减/乘,使用一些严格的公理:
$$
    \begin{array}{rlr}
        \text { 加法结合律: } &a+(b+c)=(a+b)+c, & a, b, c \in S.\\
        \text { 乘法结合律: } &a \times(b \times c)=(a \times b) \times c, & a, b, c \in S.\\
        \text { 分配律: } &a \times(b+c)=(a \times b)+(a \times c), & \text{和}\\
        &(a+b) \times c=(a \times c)+(b \times c), & a, b, c \in S .
\end{array}
$$
\subsubsection*{域}
定义:域是支持除法的环。

\subsubsection*{除法环}
除法环的每个元素都有一个逆元,但该逆元非零。代数也支持非交换乘法。
$$
    \begin{array}{rlr}
        \text{ 加法结合律: } & (a+b)+c=a+(b+c), & a, b, c \in S .\\
        \text{加法交换律: } & a+b=b+a,          & a, b \in S .\\
        \text{加法单位元 0: } & 0: 0+a=a+0,       & a, 0 \in S .\\
        \text{加法逆元: }  &a+(-a)=(-a)+a=0,    & a, -a \in S .\\
        \text{乘法结合律: } &(a \times b) \times c=a \times(b \times c),& a, b, c \in S .\\
        \text{乘法单位元 1: } &1 \times a=a \times 1,& a, 1 \in S .\\
        \text{乘法逆元: } &a \times a^{-1}=a^{-1} \times a=1,& a,a^{-1}\in S ,a\neq 0.\\
        \text{分配律: } & a\times(b+c)=(a \times b)+(a \times c), &\text{和}\\
        &(b+c) \times a=(b \times a)+(c \times a), & a, b, c \in S .
\end{array}
$$

\begin{thebibliography}{99}
    \bibitem{bib2-1} Robinson, E.: Greek and English Lexicon of the New Testament (1825). \href{http://books.google}{http://books.google}. \href{http://co.uk}{co.uk}
\end{thebibliography}